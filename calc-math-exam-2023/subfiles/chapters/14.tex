\section{Среднеквадратичная аппроксимация (непрерывный случай). Понятие ортогональности.}\label{sec:ch14}
Теперь обратимся к варианту непрерывного задания $f(x)$ на $[a, b]$
\begin{equation}
    \rho^2 = \int_a^b \left( Q(x) - f(x) \right)^2 dx \rightarrow \min \label{eq:approx_continuity}
\end{equation}
СЛАУ относительно $a_k$ сохранит прежний вид, только вместо сумм будут интегралы
\begin{equation}
    a_0 \int_{a}^{b} \varphi_0 (x) \cdot \varphi_k (x) dx + a_1 \int_{a}^{b} \varphi_1 (x) \cdot \varphi_k (x) dx + \dots + a_m \int_{a}^{b} \varphi_m (x) \cdot \varphi_k (x) dx = \int_{a}^{b} f(x) \cdot \varphi_k (x) dx \label{eq:approx_eqsys_cont}
\end{equation}

Аналогично дискретному случаю критерий $\displaystyle \rho^2$ может быть обобщен введением
положительной весовой функции $p(x)$
\begin{equation}
    \rho^2 = \int_a^b p(x) \left( Q(x) - f(x) \right)^2 dx \rightarrow \min \label{eq:approx_with_p_cont}
\end{equation}
Все формулы сохранят прежний вид, а под знаком интеграла появится $p(x)$
\begin{equation}
    a_0 \int_{a}^{b} p(x) \cdot \varphi_0 (x) \cdot \varphi_k (x) dx + \dots + a_m \int_{a}^{b} p(x) \cdot \varphi_m (x) \cdot \varphi_k (x) dx = \int_{a}^{b} p(x) \cdot f(x) \cdot \varphi_k (x) dx \label{eq:approx_eqsys_with_p_cont}
\end{equation}

Решение системы~\eqref{eq:approx_eqsys_with_p_cont} значительно упрощается, если вместо произвольных
линейно независимых функций $\left\{ \varphi_k (x) \right\}$ воспользоваться ортогональными функциями
$\left\{ g_k(x) \right\}$. Последовательность функций $\left\{ g_k(x) \right\}$ является ортогональной
на промежутке $[a, b]$ с весом $p(x)$, если выполняются следующие условия
\begin{equation*}
    \int_{a}^{b} p(x)g_k(x)g_i(x)dx =
    \begin{cases}
        0\text{, если } i \neq k\\
        A > 0\text{, если } i = k
    \end{cases}
\end{equation*}
Если в дополнение $A = 1$, то такие функции называются \emph{ортонормированными}. Для ортогональных
функций все интегралы, кроме одного, в левой части~\eqref{eq:approx_eqsys_with_p_cont} равные нулю,
матрица этой системы оказывается диагональной, и каждое уравнение дает готовое выражение для
коэффициента
\begin{equation}
    a_k = \frac{\displaystyle \int_{a}^{b} p(x)f(x)g_k(x)dx}{\displaystyle \int_{a}^{b} p(x)g_k^2(x)dx}
\end{equation}
Если исходный базис не является ортогональным, его можно сделать таковым, используя процедуру Грама-Шмидта.
