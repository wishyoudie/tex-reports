\documentclass[polytech/stats/exam-2023/stats-exam-2023.tex]{subfiles}
\graphicspath{{polytech/stats/exam-2023/images/}}

\begin{document}
    \section{Предмет теории вероятностей, элементарные исходы, случайные события, виды случайных событий.}
    Теория вероятностей занимается установлением макрозакономерностей, которым подчиняются
    массовые однородные случайные события. Теория вероятностей не стремится предсказать
    единичное событие.
    \begin{definition}[Однородные события]
        Это события, которые происходят при осуществлении одних и тех же условий $S$ и подчиняются
        определенным макрозакономерностям независимо от природы событий.
    \end{definition}
    Условия $S$ необходимо подробно и тщательно описывать в каждом конкретном случае при постановке
    задачи исследования случайных событий.

    \begin{definition}[Случайное событие]
        Событие называется \textit{случайным}, если в результате испытаний при осуществлении некоторой совокупности
        условий $S$ оно может \textit{произойти} или \textit{не произойти}. Под <<испытанием>> может пониматься
        проявление какого-либо природного явления или спланированный исследователем эксперимент над рукотворным
        или природным объектом.
    \end{definition}
    \begin{definition}[Элементарный исход]
        Это результат одного испытания в условиях $S$. Обозначение: $\omega_i$
    \end{definition}
    \begin{theorem}[Признак элементарных исходов]
        Элементарные исходы взаимно исключают друг друга, и в результате каждого испытания может произойти
        только один из элементарных исходов.
    \end{theorem}
    Примеры элементарных исходов:
    \begin{itemize}
        \item Результат бросания монеты на идеальную плоскость;
        \item Результат бросания игральной кости и выпадение на верхней грани какого-либо числа;
        \item Результат одновременного бросания нескольких игральных костей и выпадение на верхних
        гранях всех костей определенной комбинации цифр.
    \end{itemize}
    Все элементарные исходы, возможные при условиях $S$, образуют пространство элементарных исходов
    $\Omega: \forall \, i \ \omega_i \in \Omega$. Каждый элементарный исход влечет за собой появление какого-либо
    события. В общем случае событие $A$ может произойти при появлении элементарных исходов, принадлежащих
    некоторому подмножеству $\Omega_A \subset \Omega$.

    Пусть в целях некоторого исследования сформулированы условия $S$ и события $A_1$ и $A_2$, которые могут
    произойти в результате испытаний при появлении элементарных исходов, принадлежащих подмножествам
    $\Omega_{A_1} \subset \Omega, \ \Omega_{A_2} \subset \Omega$. Запишем это сопоставление событий и элементарных
    исходов в виде
    \begin{equation*}
        A_1 : \left(\omega \in \Omega_{A_1}\right), \qquad A_2 : \left(\omega \in \Omega_{A_2}\right)
    \end{equation*}
    \begin{definition}[Объединение событий]
        Пусть в этих же условиях определено событие $B$ следующим образом: <<Событие $B$ происходит или при осуществлении
        события $A_1$, или при осуществлении события $A_2$>>. При такой формулировке говорят, что событие $B$ является
        \textit{объединением событий} $A_1$ и $A_2$ и записывают: $B = A_1 \cup A_2$. В этом случае подмножество элементарных
        исходов, влекущих за собой событие $B$, есть объединение подмножеств $\Omega_{A_1}$ и $\Omega_{A_2}$:
        \begin{equation*}
            B: \left(\omega \in \Omega_{B}\right), \text{ где } \Omega_B = \Omega_{A_1} \cup \Omega_{A_2}
        \end{equation*}
    \end{definition}
    \begin{definition}[Пересечение событий]
        Если в этих же условиях принято, что событие $B$ происходит, когда события $A_1$ и $A_2$ осуществляются одновременно,
        то говорят, что событие $B$ есть \textit{пересечение событий} $A_1$ и $A_2$, и записывают этот факт в виде $B = A_1 \cap A_2$,
        при чем в этом случае 
        \begin{equation*}
            B: \left(\omega \in \Omega_{B}\right), \text{ где } \Omega_B = \Omega_{A_1} \cap \Omega_{A_2}
        \end{equation*}
    \end{definition}
    Выделяют виды случайных событий:
    \begin{enumerate}
        \item \textit{Достоверное событие} $T: \left(\omega \in \Omega_T, \, \Omega_T = \Omega\right)$ -- событие, которое непременно
        происходит при появлении любого элементарного исхода в условиях $S$;
        \item \textit{Невозможное событие} $\varnothing$ -- событие, которое не может произойти ни при одном элементарном исходе
        из пространства $\Omega$ при условиях $S$.
        \item События $A$ и $B$ называют \textit{несовместными}, если появление одного из них исключает появление другого. Для них
        $A \subset \Omega, \, B \subset \Omega\, A \cap B = \varnothing$;
        \item События $A$ и $B$ называют \textit{противоположными}, если они несовместны и $A \cup B = T$, в этом случае пользуются
        обозначениями $B = \overline{A}$ или $A = \overline{B}$;
        \item События $\displaystyle A_i \vert_1^n$ образуют \textit{полную группу попарно несовместных событий}, если при условиях $S$
        осуществляется только одно из этих событий и 
        \begin{equation*}
            \bigcup_{i = 1}^n A_i = T, \ A_i \cap A_j = \varnothing \ \forall \, i \neq j.
        \end{equation*}
    \end{enumerate}
    \newpage
\end{document}
