\documentclass[../../calc-math-exam-2023.tex]{subfiles}
\begin{document}
    \section{Выбор узлов интерполирования. Интерполяционный полином Ньютона для равно и неравноотстоящих узлов.}\label{sec:ch07}
    \subsection{Выбор узлов интерполирования}
    Для уменьшения погрешности интерполирования обратимся к теореме об остаточном члене
    полинома Лагранжа при заданной степени полинома $m$. Поскольку величиной $\displaystyle f^{(m+1)}(\eta)$ трудно управлять,
    и возможна лишь оценка пределов ее изменения, задача уменьшения погрешности сводится
    к управлению величиной $|\omega(x)|$ за счет выбора узлов интерполирования. Рассмотрим
    два типичных на практике случая.

    \emph{Случай 1}. Задана степень полинома $m$ и имеется таблица достаточно большой длины.
    Точка $x^{*}$, в которой вычисляется значение полинома, заранее известна. Требуется
    выбрать $m+1$ узел так, чтобы величина $|\omega(x^{*})|$ была бы минимальна.

    Результат очевиден. Нужно выбирать узлы интерполирования из таблицы,
    \emph{ближайшие} к $x^{*}$. Использование любого другого узла вместо ближайшего
    неизбежно увеличивает значение
    \begin{equation*}
        |\omega(x^{*})| = |\left( x^{*} - x_0 \right)\left( x^{*} - x_1 \right)\dots\left( x^{*} - x_m \right)|
    \end{equation*}

    \emph{Случай 2}. Заданы степень полинома $m$ и промежуток интерполирования $[a, b]$.
    Точка $x^{*}$, в которой вычисляется значение полинома, заранее не известна.
    Требуется выбрать узлы интерполирования так, чтобы в самом неблагоприятном случае
    расположения $x^{*}$ погрешность была бы минимальна (т.н. \emph{минимаксный критерий})
    \begin{equation*}
        \max_{[a, b]} |\omega(x^{*})| \rightarrow \min
    \end{equation*}
    Интуитивно напрашивающееся предложение о равномерном задании узлов на промежутке
    оказывается ошибочным. Значения $|\omega(x)|$ в узлах интерполирования равны нулю,
    график напоминает <<колокольчики>>, максимум которых достигается между узлами
    интерполирования. При выборе равноотстоящих узлов погрешность для $x^{*}$, близких
    к центру промежутка интерполирования оказывается небольшой, однако ближе к концам
    она сильно возрастает. Узлы интерполирования нужно симметрично сместить ближе к
    концам промежутка. Тогда высота центрального <<колокольчика>> увеличится, в то
    время как высота крайних уменьшится. Оптимальный выбор узлов интерполирования
    отвечает нулям так называемых ортогональных полиномов Чебышева, когда все
    <<колокольчики>> будут одинаковыми по высоте.


\end{document}