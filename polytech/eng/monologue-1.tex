\documentclass[a4paper,12pt]{article}
%%% Default imports
\usepackage{listings} % Code listings
\usepackage{graphicx} % Images
\usepackage{booktabs} % Better tables
\usepackage{makecell}
\usepackage{enumitem} % Lists
\usepackage{dsfont}
\usepackage{geometry} % Page geometry
\usepackage[utf8]{inputenc} % Encoding
\usepackage[T2A]{fontenc} % Font
\usepackage[english, russian]{babel} % Multi-language support
\usepackage{titling} % Better titles
\usepackage{textcomp} % Old-style numbers? Check difference
\usepackage{mathtext} % Russian text in math expressions? Check difference
\usepackage{amsmath, amsfonts, amssymb, amsthm, mathtools} % Mathematics
\usepackage{bm} % Bold math symbols
\usepackage{icomma} % Better comma in numbers within math mode
\usepackage{xifthen} % Better if-expressions
\usepackage{transparent} % Transparent colors
\usepackage{caption}    % }
\usepackage{subcaption} % } Captioning figures
\usepackage[table,xcdraw]{xcolor} % Colors
\usepackage{textpos} % Absolute positioning
\usepackage{upgreek} % Cool greek letters

\usepackage{fancyvrb}
\usepackage{fvextra}
\usepackage{chngcntr}

%%% Page geometry
% \setlength\parindent{0pt} % No indentation in paragraphs
\setlist{noitemsep} % No spacing between list items

\usepackage{float}
\usepackage{multirow}

\geometry{
    paper=a4paper,
    top=2.5cm,
    bottom=3cm,
    left=2.5cm,
    right=2.5cm,
    headheight=0.75cm,
    footskip=1.5cm,
    headsep=0.75cm,
}

%%% Numeration
\newcommand{\RNumb}[1]{\uppercase\expandafter{\romannumeral #1\relax}}
\newcommand{\thesec}{\arabic{section}}
\renewcommand\thesection{\arabic{section}}
\renewcommand\thesubsection{\thesection.\arabic{subsection}}
\renewcommand\thesubsubsection{\RNumb{\arabic{subsubsection}}}
\renewcommand{\sectionmark}[1]{\markright{\thesection\ #1}}
\renewcommand{\bf}{\textbf}
\renewcommand{\it}{\textit}
\def\hash{\texttt{\#}}
\def\cpp{\C\texttt{++}}

\counterwithin{figure}{section}
\counterwithin{table}{section}
\renewenvironment{titlepage}{\thispagestyle{empty}} % Include titlepage into page numeration

%%% Headers and footers
\usepackage{setspace}
\usepackage{fancyhdr}
\usepackage{lastpage}


%%% Graphics
% Custom colors
\definecolor{myblue}{RGB}{72, 184, 178}
\definecolor{myblue1}{RGB}{0, 109, 167}
\definecolor{commentgreen}{RGB}{2,112,10}
\definecolor{mauve}{rgb}{0.58,0,0.82}
\definecolor{amethyst}{RGB}{153, 102, 203}

\usepackage{pgfplots} % Plots
\usepackage{tikz}
\usetikzlibrary{3d,perspective,decorations.text}
\usetikzlibrary{animations}
\usetikzlibrary{positioning}
\usetikzlibrary{matrix}
\usepackage{tikz-cd}
\usetikzlibrary{cd}
\usetikzlibrary{karnaugh}
\pgfplotsset{width=6cm,compat=newest}
\usepackage{color}

\usepackage[framemethod=TikZ]{mdframed}
\newcommand{\definebox}[2]{\newcounter{#1}\newenvironment{#1}[1][]{\stepcounter{#1}\mdfsetup{frametitle={\tikz[baseline=(current bounding box.east),outer sep=0pt]\node[anchor=east,rectangle,fill=white]{\strut \MakeUppercase#1~\csname the#1\endcsname\ifstrempty{##1}{}{:~##1}};}}\mdfsetup{innertopmargin=1pt,linecolor=#2,linewidth=3pt,topline=true,frametitleaboveskip=\dimexpr-\ht\strutbox\relax,}\begin{mdframed}[]\relax}{\end{mdframed}}}
\definebox{definition}{black!90}
\definebox{theorem}{myblue1!90}
\definebox{demonstration}{amethyst!90}

\newcounter{Theorem}
\def\themytheorem{\thesection.\arabic{Theorem}}
\usepackage[most]{tcolorbox}
\tcbuselibrary{theorems}
\newtcbtheorem{Theorem}{Theorem}
{colframe=myblue!90,coltitle=black,colback=white,fonttitle=\bfseries}{Th}

%%% Code listings
\usepackage{matlab-prettifier}

\lstset{
    extendedchars=\true,
}
\lstdefinestyle{cpp} {
    language=C++,
    frame=tb,
    tabsize=4,
    showstringspaces=false,
    numbers=left,
    captionpos=b,
    columns=flexible,
    upquote=true,
    commentstyle=\color{commentgreen},
    keywordstyle=\color{blue},
    stringstyle=\color{commentgreen},
    basicstyle=\small\ttfamily,
    emph={int,char,double,float,unsigned,void,bool,size\_t},
    emphstyle={\color{blue}},
    escapechar=\&,
    classoffset=1,
    otherkeywords={>,<,.,;,-,!,=,~},
    morekeywords={>,<,.,;,-,!,=,~},
    keywordstyle=\color{black},
    classoffset=0,
}
\lstdefinestyle{py} {
    language=Python,
    frame=tb,
    tabsize=4,
    showstringspaces=false,
    numbers=left,
    captionpos=b,
    columns=flexible,
    upquote=true,
    commentstyle=\color{commentgreen},
    keywordstyle=\color{blue},
    stringstyle=\color{commentgreen},
    basicstyle=\small\ttfamily,
    emph={and,as,assert,break,class,continue,def,del,elif,else,except,False,finally,for,from,global,if,import,in,%
    is,lambda,None,nonlocal,not,or,pass,raise,return,True,try,while,with,yield},
    emphstyle={\color{blue}},
    classoffset=1,
    otherkeywords={>,<,.,;,-,!,=,~},
    morekeywords={>,<,.,;,-,!,=,~},
    keywordstyle=\color{black},
    classoffset=0,
}
\lstdefinestyle{def} {
    frame=tb,
    tabsize=4,
    showstringspaces=false,
    numbers=left,
    captionpos=b,
    columns=flexible,
    upquote=true,
    commentstyle=\color{black},
    keywordstyle=\color{black},
    stringstyle=\color{black},
    basicstyle=\small\ttfamily,
    emph={int,char,double,float,unsigned,void,bool,size\_t},
    emphstyle={\color{black}},
    escapechar=\&,
    classoffset=1,
    otherkeywords={>,<,.,;,-,!,=,~},
    morekeywords={>,<,.,;,-,!,=,~},
    keywordstyle=\color{black},
    classoffset=0,
}

%%% Other
\usepackage[normalem]{ulem} % }
\useunder{\uline}{\ul}{}    % } Underline text
\usepackage[colorlinks,urlcolor=blue,filecolor=blue,citecolor=blue,linkcolor = blue,unicode=true]{hyperref}
\usepackage{titlesec}
\titlelabel{\thetitle.\quad}
\usepackage{secdot}
\sectiondot{subsection}
\usepackage{kvmap} % Karnaugh-maps for logic functions
\usepackage{}
\newcommand{\projectname}[3]{
    \begin{center}
        \Large
        \textbf{#1}\\[10pt]
        \textbf{#2}\\[10pt]
        \normalsize
        #3
        \rule{\linewidth}{0.4pt}
    \end{center}
}

\newcommand{\hfconfiguration}[3]{
    \pagestyle{fancy}
    \fancyhead[LE,RO]{}
    \fancyhead[LO,RE]{#1}
    \renewcommand{\footrulewidth}{0.4pt}
    \fancyfoot[C]{\thepage/\pageref*{LastPage}}
    \fancyfoot[LO,RE]{#2}
    \fancyfoot[LE,RO]{#3}
}

\newcommand{\filename}[2]{
    \pagebreak
    \titleformat{\section}
    [display]
    {\bfseries\Large}
    {}
    {0ex}
    {
        \vspace{-4.5ex}
        % \rule{\textwidth}{1pt}
        #1 \centering
    % \vspace{1ex}
    }
    [
        \normalfont\large
        #2
        \rule{\textwidth}{0.4pt}
        \normalsize
    ]
}

\newcommand{\project}[6]{
    \projectname{#1}{#2}{#3}
    \pagestyle{empty}
    \tableofcontents
    \newpage
    \hfconfiguration{#4}{#5}{#6}
}
%%% Final touch
\usepackage{subfiles}


% NB!! use 'texcount monologue-1.tex -total' to see the word count 

\begin{document}
    \hfconfiguration{Ильин Владимир, гр. 3530901/10005}{}{}
    \section{The Digital Age}
    % \subsection{What is the Digital Age? What features make it different from other ages?}
    
    The first topic I would like to talk about is The Digital Age, also called the Computer Age, the Information Age and the New Media Age.
    It is coupled tightly with the advent of personal computers, but many computer historians
    trace its beginnings to the work of the American mathematician Claude E. Shannon. At age
    32 he proposed that information can be quantitatively encoded as a series of ones and zeroes.
    He also suggested a way of transmitting such information without any errors. 

    By the 1970s, computer technologies have made a huge progress. Development of Internet, fiber optic cables
    and faster microprocessors accelerated the transmission and processing of information. Digitalization of information,
    for example, text messages or photos, turned the World Wide Web into an interactive consumer exchange for goods and information,
    allowing people to chat, share knowledge and spend their free time so-called 'surfing' the Web. Near-instant exchange of information
    undoubtedly had a profound impact on people's lives. If Digital Age didn't appear, I wouldn't be able to hand this monologue in without
    spending money on paper to print it out.

    % \subsection{What are the most important milestones in the history of the Digital Age?}

    \section{Smart machines}
    % \subsection{What are smart machines? Speak about positive and negative impact smart machine may have on society}

    The next subject in my speech is called 'Smart machines'. A smart machine is a device embedded with machine-to-machine (M2M) technologies
    or a cognitive computing system, e.g. artificial intelligence. Smart machines use this appliances to problem-solve, make decisions and take action.
    They are mainly used to automate some human tasks, which are unnecessary to be stealing human's time.
    Smart machines are often recalled as digital disruptors because of the positive and negative impact they have on society. Although smart machines can
    provide technological advantages and ensure more efficient manufacturing process, they will therefore displace workers and take away their working positions.

    My favourite smart machine are perharps artificial intelligence models, such as popular nowadays ChatGPT. I prefer AI over others because of it's
    universality and wide range of tasks it can solve. Also, another reason for my choice is AI's current imperfection. It is still a relatively new concept
    and it definitely has a lot of unsolved issues, which means a lot of interesting ideas and inventions coming next in our future.

    % \subsection{Describe a smart machine of your choice. What features make it smart?}

    \section{Smart cities and homes}
    % \subsection{Define what a smart city is. What technologies it uses. Features of a smart city. How a smart city works.}
    The last topic I am going to mention is Smart cities. A smart city is a concept of a city with intelligent systems for transport, energy and more.
    It uses information and communication technology to improve operational efficiency, share information with the public and provide a better quality of
    government service and citizen welfare. A number if technologies a smart city typically uses is insane. Among them are Wi-Fi deployment, 5G network,
    Internet of Things and many more. That technology stack easily allows for smart city to have an immense amount of features. For example, smart parking
    can help drivers find a parking space and also allow for digital payment. Another example would be energy conservation and environmental efficiencies,
    such as streetlights that dim when the roads are empty. Smart cities follow four steps to improve quality of life and enable economic growth through a
    network of connected IoT devices and other technologies. These steps are: collection -- smart sensors gather real-time data, analysis -- the data is analyzed
    to gain insights into the operation of city services and operations, communication -- the results of the data analysis are communicated to decision makers, and,
    finally, action -- action is taken to improve operations, manage assets and improve the quality of city life for the residents.

    % \subsection{Smart homes. What are they like? Would you like to live in a smart home? What smart appliances/gadgets will it be equipped with?}
    A mini-version of a smart city is a smart home. In addition, smart home concept is something much more achievable for yourself, since almost everyone can make
    their home smart. This concept includes wireless control over lights, more precise temperature supporting devices and so on. Personally, I would love to live
    in such a house. I already have smart lights at my place, which I can control using my phone or automate the process by making the lights depend on current sun
    state. Moreover, I have made some programs to increase the comfort of watching movies from my bed. For example, I no longer need to get up to skip to the next
    episode of a TV show: all I need to do is to say 'go next' and my computer will follow my instructions.
\end{document}