\section{Адаптивные квадратурные формулы. Подпрограмма \textbf{QUANC8}.}\label{sec:ch11}
Рассмотренный ранее простой алгоритм вычисления интеграла на основе составных квадратурных формул со сравнением
результата для $N$ и $2N$, отличается достаточной надежностью, но его быстродействие может быть заметно повышено.
Например, если функция в основном ведет себя стабильно, а на каком-то малом участке резко меняется, то имеет смысл
требовать подсчет с малым шагом только на этом участке, а не на всем рассматриваемом промежутке.

Целесообразным представляется построение алгоритма, который был бы способен \emph{адаптироваться} к виду функции и
выбирать достаточно малый шаг там, где функция меняется быстро и характеризуется большими производными, и относительно
большой шаг там, где функция меняется медленно. На этом пути возможны два варианта: минимизировать погрешность при
заданном объеме вычислений или минимизировать объем вычислений при заданных требованиях к погрешности. В рассматриваемой
программе реализован второй подход.

В основу положена квадратурная формула Ньютона-Котеса с девятью узлами, т.е. восемью промежутками между ними, что и
оправдывает название программы. Ее составная формула имеет погрешность вида~\eqref{eq:quad_eps} для $p=10$.

Рассмотрим промежуток длиной $h_k$ внутри $[a, b]$ и введем для него следующие обозначения:

$I_k$ -- точное значение интеграла на этом промежутке

$P_k$ -- значение интеграла, вычисленное по квадратурной формуле с девятью узлами

$Q_k$ -- значение интеграла, вычисленное по той же формуле, примененной к двум половинам этого промежутка (по сути
используется составная формула с вдвое большим значением $N$)

Учитывая вид~\eqref{eq:quad_eps} для $p = 10$, для погрешностей $P_k$ и $Q_k$ последовательно имеем
\begin{equation*}
    I_k - P_k \approx 2^p \left( I_k - Q_k \right); \qquad I_k \approx \frac{2^p Q_k - P_k}{2^p - 1}; \qquad I_k - Q_k \approx \frac{Q_k - P_k}{2^p - 1} = \frac{Q_k - P_k}{1023}
\end{equation*}
Обозначая требуемую абсолютную погрешность вычисления интеграла на всем промежутке $[a, b]$ за $\varepsilon_A$, считаем промежуток
$h_k$ <<принятым>>, а интеграл на нем вычисленным, если выполняется неравенство
\begin{equation}
    \left| \frac{Q_k - P_k}{1023} \right| \leq \frac{h_k}{b-a}\varepsilon_A\label{eq:adaptive_quad_eps_1}
\end{equation}
Множитель $\displaystyle \frac{h_k}{b-a}$ является весовым коэффициентом и отражает вклад погрешности на промежутке $h_k$
в общую погрешность для всего промежутка. Возможно также использование и относительной погрешности $\varepsilon_R$
\begin{equation}
    \left| \frac{Q_k - P_k}{1023} \right| \leq \frac{h_k}{b-a}\varepsilon_R \left| \tilde{I}_k \right|\label{eq:adaptive_quad_eps_2}
\end{equation}
где $\displaystyle \tilde{I}_k$ -- оценка вычисления интеграла по всему промежутку. Следует, однако, помнить, что
использование критерия относительной погрешности усложняется, если значение $\displaystyle \tilde{I}_k$ оказывается
нулевым или близким к нулю. В программе \verb|QUANC8| пользователю предоставляется возможность использовать один из трех
вариантов контроля погрешности на основе формул~\eqref{eq:adaptive_quad_eps_1} и~\eqref{eq:adaptive_quad_eps_2}
\begin{equation}
    \left| \frac{Q_k - P_k}{1023} \right| \leq \frac{h_k}{b-a}\max\left( \varepsilon_A ; \varepsilon_R \left| \tilde{I}_k \right| \right)\label{eq:adaptive_quad_eps}
\end{equation}
\begin{enumerate}
    \item $\varepsilon_R = 0, \quad \varepsilon_A \neq 0$ -- контроль абсолютной погрешности
    \item $\varepsilon_R \neq 0, \quad \varepsilon_A = 0$ -- контроль относительной погрешности
    \item $\varepsilon_R \neq 0, \quad \varepsilon_A \neq 0$ -- контроль <<смешанной>> погрешности
\end{enumerate}
В последнем случае делается попытка избежать упомянутых неприятных ситуаций со значениями $\displaystyle \tilde{I}_k$,
близкими к нулю.

Адаптация программы к виду функции и реализация переменного шага интегрирования реализуются в соответствии со следующим
алгоритмом. Вычисляются $P_k$ и $Q_k$ применительно ко всему промежутку. Если погрешность еще достаточно велика,
промежуток делится пополам, значения подынтегральной функции $f(x)$, вычисленные на правой половине промежутка,
запоминаются, и все повторяется для левой половины промежутка. Такое обращение каждый раз к левой половине текущего
промежутка продолжается до тех пор, пока крайний слева промежуток не будет принят. После этого обрабатывается ближайший
к нему правый промежуток. Запоминание значений $f(x)$ повышает быстродействие алгоритма.

В программе реализовано два ограничения сверху на объем вычислений. Во-первых, деление промежутка пополам продолжается
не более 30 раз. По достижении этой величины соответствующий интеграл на нем считается вычисленным, а промежуток
<<принятым>>, независимо от условия~\eqref{eq:adaptive_quad_eps}. Число таких промежутков, принятых с нарушением
условия~\eqref{eq:adaptive_quad_eps}, содержится в целой части выходного значения переменной \verb|FLAG|. Длина каждого
такого промежутка крайне мала ($\displaystyle \approx 10^{-9}$), и подобная ситуация, как правило, связана с разрывами
подынтегральной функции или ее <<зашумлением>> вычислительной погрешностью. Во-вторых, вводится ограничение сверху на
количество вычислений подынтегральной функции $f(x)$. Если этот предел достигнут, то информация о точке
$\displaystyle x^{*}$, где возникла трудность, отражена в дробной части выходного значения переменной \verb|FLAG|.

Например, для $\verb|FLAG| = 91.21$, число промежутков равно 91, а
\begin{equation*}
    \frac{b-x^{*}}{b-a} = 0.21; \qquad x^{*} = b - 0.21(b-a)
\end{equation*}

Программа имеет следующие параметры

\bf{FUN} -- имя подпрограммы-функции, вычисляющей значение подынтегральной функции $f(x)$;

\bf{A, B} -- нижний и верхний пределы интегрирования;

\bf{ABSERR, RELERR} -- границы абсолютной $\varepsilon_A$ и относительной $\varepsilon_R$ погрешностей. Остальные параметры -- выходные со
следующим смыслом:

\bf{RESULT} -- значение интеграла, определенное программой;

\bf{ERREST} -- оценка погрешности, выполненная программой и удовлетворяющая~\eqref{eq:adaptive_quad_eps};

\bf{NOFUN} -- количество вычислений подынтегральной функции $f(x)$, использованных для получения результата;

\bf{FLAG} -- индикатор надежности результата. Нулевое значение этой переменной отвечает относительной надежности
результата, а ненулевое, как уже отмечалось, свидетельствует об отклонениях от нормального хода выполнения программы.
