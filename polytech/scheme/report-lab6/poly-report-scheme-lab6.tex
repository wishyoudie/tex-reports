\documentclass[a4paper,12pt]{article}
%%% Default imports
\usepackage{listings} % Code listings
\usepackage{graphicx} % Images
\usepackage{booktabs} % Better tables
\usepackage{makecell}
\usepackage{enumitem} % Lists
\usepackage{dsfont}
\usepackage{geometry} % Page geometry
\usepackage[utf8]{inputenc} % Encoding
\usepackage[T2A]{fontenc} % Font
\usepackage[english, russian]{babel} % Multi-language support
\usepackage{titling} % Better titles
\usepackage{textcomp} % Old-style numbers? Check difference
\usepackage{mathtext} % Russian text in math expressions? Check difference
\usepackage{amsmath, amsfonts, amssymb, amsthm, mathtools} % Mathematics
\usepackage{bm} % Bold math symbols
\usepackage{icomma} % Better comma in numbers within math mode
\usepackage{xifthen} % Better if-expressions
\usepackage{transparent} % Transparent colors
\usepackage{caption}    % }
\usepackage{subcaption} % } Captioning figures
\usepackage[table,xcdraw]{xcolor} % Colors
\usepackage{textpos} % Absolute positioning
\usepackage{upgreek} % Cool greek letters

\usepackage{fancyvrb}
\usepackage{fvextra}
\usepackage{chngcntr}

%%% Page geometry
% \setlength\parindent{0pt} % No indentation in paragraphs
\setlist{noitemsep} % No spacing between list items

\usepackage{float}
\usepackage{multirow}

\geometry{
    paper=a4paper,
    top=2.5cm,
    bottom=3cm,
    left=2.5cm,
    right=2.5cm,
    headheight=0.75cm,
    footskip=1.5cm,
    headsep=0.75cm,
}

%%% Numeration
\newcommand{\RNumb}[1]{\uppercase\expandafter{\romannumeral #1\relax}}
\newcommand{\thesec}{\arabic{section}}
\renewcommand\thesection{\arabic{section}}
\renewcommand\thesubsection{\thesection.\arabic{subsection}}
\renewcommand\thesubsubsection{\RNumb{\arabic{subsubsection}}}
\renewcommand{\sectionmark}[1]{\markright{\thesection\ #1}}
\renewcommand{\bf}{\textbf}
\renewcommand{\it}{\textit}
\def\hash{\texttt{\#}}
\def\cpp{\C\texttt{++}}

\counterwithin{figure}{section}
\counterwithin{table}{section}
\renewenvironment{titlepage}{\thispagestyle{empty}} % Include titlepage into page numeration

%%% Headers and footers
\usepackage{setspace}
\usepackage{fancyhdr}
\usepackage{lastpage}


%%% Graphics
% Custom colors
\definecolor{myblue}{RGB}{72, 184, 178}
\definecolor{myblue1}{RGB}{0, 109, 167}
\definecolor{commentgreen}{RGB}{2,112,10}
\definecolor{mauve}{rgb}{0.58,0,0.82}
\definecolor{amethyst}{RGB}{153, 102, 203}

\usepackage{pgfplots} % Plots
\usepackage{tikz}
\usetikzlibrary{3d,perspective,decorations.text}
\usetikzlibrary{animations}
\usetikzlibrary{positioning}
\usetikzlibrary{matrix}
\usepackage{tikz-cd}
\usetikzlibrary{cd}
\usetikzlibrary{karnaugh}
\pgfplotsset{width=6cm,compat=newest}
\usepackage{color}

\usepackage[framemethod=TikZ]{mdframed}
\newcommand{\definebox}[2]{\newcounter{#1}\newenvironment{#1}[1][]{\stepcounter{#1}\mdfsetup{frametitle={\tikz[baseline=(current bounding box.east),outer sep=0pt]\node[anchor=east,rectangle,fill=white]{\strut \MakeUppercase#1~\csname the#1\endcsname\ifstrempty{##1}{}{:~##1}};}}\mdfsetup{innertopmargin=1pt,linecolor=#2,linewidth=3pt,topline=true,frametitleaboveskip=\dimexpr-\ht\strutbox\relax,}\begin{mdframed}[]\relax}{\end{mdframed}}}
\definebox{definition}{black!90}
\definebox{theorem}{myblue1!90}
\definebox{demonstration}{amethyst!90}

\newcounter{Theorem}
\def\themytheorem{\thesection.\arabic{Theorem}}
\usepackage[most]{tcolorbox}
\tcbuselibrary{theorems}
\newtcbtheorem{Theorem}{Theorem}
{colframe=myblue!90,coltitle=black,colback=white,fonttitle=\bfseries}{Th}

%%% Code listings
\usepackage{matlab-prettifier}

\lstset{
    extendedchars=\true,
}
\lstdefinestyle{cpp} {
    language=C++,
    frame=tb,
    tabsize=4,
    showstringspaces=false,
    numbers=left,
    captionpos=b,
    columns=flexible,
    upquote=true,
    commentstyle=\color{commentgreen},
    keywordstyle=\color{blue},
    stringstyle=\color{commentgreen},
    basicstyle=\small\ttfamily,
    emph={int,char,double,float,unsigned,void,bool,size\_t},
    emphstyle={\color{blue}},
    escapechar=\&,
    classoffset=1,
    otherkeywords={>,<,.,;,-,!,=,~},
    morekeywords={>,<,.,;,-,!,=,~},
    keywordstyle=\color{black},
    classoffset=0,
}
\lstdefinestyle{py} {
    language=Python,
    frame=tb,
    tabsize=4,
    showstringspaces=false,
    numbers=left,
    captionpos=b,
    columns=flexible,
    upquote=true,
    commentstyle=\color{commentgreen},
    keywordstyle=\color{blue},
    stringstyle=\color{commentgreen},
    basicstyle=\small\ttfamily,
    emph={and,as,assert,break,class,continue,def,del,elif,else,except,False,finally,for,from,global,if,import,in,%
    is,lambda,None,nonlocal,not,or,pass,raise,return,True,try,while,with,yield},
    emphstyle={\color{blue}},
    classoffset=1,
    otherkeywords={>,<,.,;,-,!,=,~},
    morekeywords={>,<,.,;,-,!,=,~},
    keywordstyle=\color{black},
    classoffset=0,
}
\lstdefinestyle{def} {
    frame=tb,
    tabsize=4,
    showstringspaces=false,
    numbers=left,
    captionpos=b,
    columns=flexible,
    upquote=true,
    commentstyle=\color{black},
    keywordstyle=\color{black},
    stringstyle=\color{black},
    basicstyle=\small\ttfamily,
    emph={int,char,double,float,unsigned,void,bool,size\_t},
    emphstyle={\color{black}},
    escapechar=\&,
    classoffset=1,
    otherkeywords={>,<,.,;,-,!,=,~},
    morekeywords={>,<,.,;,-,!,=,~},
    keywordstyle=\color{black},
    classoffset=0,
}

%%% Other
\usepackage[normalem]{ulem} % }
\useunder{\uline}{\ul}{}    % } Underline text
\usepackage[colorlinks,urlcolor=blue,filecolor=blue,citecolor=blue,linkcolor = blue,unicode=true]{hyperref}
\usepackage{titlesec}
\titlelabel{\thetitle.\quad}
\usepackage{secdot}
\sectiondot{subsection}
\usepackage{kvmap} % Karnaugh-maps for logic functions
\usepackage{}
\newcommand{\projectname}[3]{
    \begin{center}
        \Large
        \textbf{#1}\\[10pt]
        \textbf{#2}\\[10pt]
        \normalsize
        #3
        \rule{\linewidth}{0.4pt}
    \end{center}
}

\newcommand{\hfconfiguration}[3]{
    \pagestyle{fancy}
    \fancyhead[LE,RO]{}
    \fancyhead[LO,RE]{#1}
    \renewcommand{\footrulewidth}{0.4pt}
    \fancyfoot[C]{\thepage/\pageref*{LastPage}}
    \fancyfoot[LO,RE]{#2}
    \fancyfoot[LE,RO]{#3}
}

\newcommand{\filename}[2]{
    \pagebreak
    \titleformat{\section}
    [display]
    {\bfseries\Large}
    {}
    {0ex}
    {
        \vspace{-4.5ex}
        % \rule{\textwidth}{1pt}
        #1 \centering
    % \vspace{1ex}
    }
    [
        \normalfont\large
        #2
        \rule{\textwidth}{0.4pt}
        \normalsize
    ]
}

\newcommand{\project}[6]{
    \projectname{#1}{#2}{#3}
    \pagestyle{empty}
    \tableofcontents
    \newpage
    \hfconfiguration{#4}{#5}{#6}
}
%%% Final touch
\usepackage{subfiles}

\graphicspath{{polytech/scheme/report-lab6/subfiles/images/}}

\begin{document}
    \begin{titlepage}
		\begin{center}
			\large Санкт-Петербургский политехнический университет Петра Великого\\
			\large Институт компьютерных наук и технологий \\
			\large Кафедра компьютерных систем и программных технологий\\[6cm]


		\huge КУРСОВАЯ РАБОТА\\[0.5cm]
			\large по дисциплине <<Вычислительная математика>>\\[0.1cm]
			\large\textbf{Исследование уравнения движения пузырьков}\\[5cm]
		\end{center}


		\begin{flushright}
			\begin{minipage}{0.25\textwidth}
				\begin{flushleft}

					\large\textbf{Работу выполнил:}\\
					\large Ильин В.П.\\
					\large {Группа:} 3530901/10005\\

					\large \textbf{Преподаватель:}\\
					\large Куляшова З.В.

				\end{flushleft}
			\end{minipage}
		\end{flushright}

		\vfill

		\begin{center}
			\large Санкт-Петербург\\
			\large \the\year
		\end{center}
	\end{titlepage}

	\vfill
	\newpage


    \tableofcontents
    \hfconfiguration{Лабораторная работа №6}{}{}

    \section{Цель работы}
    Исследование счетчиков, построенных по различной архитектуре, и типовых функциональных устройств с их использованием.
    \section{Исходные данные}
    Вариант задания -- 8. Число тетрад = 3.
    \section{Ход работы}
    \subsection{Исследование счетчика с последовательным переносом}
    \begin{figure}[H]
        \centering
        \includegraphics[width=0.9\linewidth]{scheme1}
        \caption{Разработанная схема}
    \end{figure}

    \begin{figure}[H]
        \centering
        \includegraphics[width=0.6\linewidth]{compile1}
        \caption{Аппаратные затраты}
    \end{figure}

    \begin{figure}[H]
        \centering
        \includegraphics[width=0.6\linewidth]{fmax1}
        \caption{Максимальная тактовая частота}
    \end{figure}

    \begin{figure}[H]
        \centering
        \includegraphics[width=0.7\linewidth]{timings1}
        \caption{Задержки появления сигналов}
    \end{figure}
    Сумма Тсо = 37,546.
    \begin{figure}[H]
        \centering
        \includegraphics[width=0.9\linewidth]{tmv1}
        \caption{Technology Map Viewer}
    \end{figure}

    Проведем несколько временных тестов. Для начала используем частоту, меньшую максимальной.

    \begin{figure}[H]
        \centering
        \includegraphics[width=\linewidth]{wave_1__60}
        \caption{$T = 60 \text{ нс} < T_\textit{max}$ -- счетчик работает нормально.}
    \end{figure}
    \begin{figure}[H]
        \centering
        \includegraphics[width=\linewidth]{wave_1__30}
        \caption{$T = 30 \text{ нс} \approx T_\textit{max}$ -- счетчик работает нормально.}
    \end{figure}
    \begin{figure}[H]
        \centering
        \includegraphics[width=\linewidth]{wave_1__5}
        \caption{$T = 5 \text{ нс} > T_\textit{max}$ -- счетчик работает с опозданием.}
    \end{figure}
    \begin{figure}[H]
        \centering
        \includegraphics[width=\linewidth]{wave_1__3}
        \caption{$T = 3 \text{ нс} >> T_\textit{max}$ -- счетчик ломается.}
    \end{figure}
    \subsection{Исследование счетчика, реализованного на основе мегафункции}
    \begin{figure}[H]
        \centering
        \includegraphics[width=0.9\linewidth]{scheme2}
        \caption{Разработанная схема}
    \end{figure}

    \begin{figure}[H]
        \centering
        \includegraphics[width=\linewidth]{wave_2_1}
        \caption{Временная диаграмма работы счетчика в режиме сложения}
    \end{figure}

    \begin{figure}[H]
        \centering
        \includegraphics[width=\linewidth]{wave_2_2}
        \caption{Временная диаграмма работы счетчика в режиме вычитания}
    \end{figure}

    \begin{figure}[H]
        \centering
        \includegraphics[width=\linewidth]{wave_2_3}
        \caption{Временная диаграмма работы счетчика в режиме синхронной загрузки}
    \end{figure}
    \subsection{Делитель частоты на число}
    \begin{figure}[H]
        \centering
        \includegraphics[width=\linewidth]{scheme_div}
        \caption{Разработанная схема}
    \end{figure}
    \begin{figure}[H]
        \centering
        \includegraphics[width=\linewidth]{wave_div}
        \caption{Временная диаграмма}
    \end{figure}
    Видно, что cout принимает значение 1 раз в data тактов (в данном случае data = 15). Если перевести
    счетчик в режим сложения и подать на вход число (256 - data) = 241, то cout станет единицей только при
    числе 255.
    \subsection{Устройство фиксации коротких импульсов}
    \begin{figure}[H]
        \centering
        \includegraphics[width=\linewidth]{scheme_filter}
        \caption{Разработанная схема}
    \end{figure}
    \subsection{Двоично-десятичный счетчик}
    \begin{figure}[H]
        \centering
        \includegraphics[width=\linewidth]{scheme_5}
        \caption{Разработанная схема}
    \end{figure}
    \begin{figure}[H]
        \centering
        \includegraphics[width=\linewidth]{wave_5}
        \caption{Временная диаграмма}
    \end{figure}
    \subsection{Преобразователь из двоичного кода в двоично-десятичный}
    \begin{figure}[H]
        \centering
        \includegraphics[width=\linewidth]{scheme_6}
        \caption{Разработанная схема}
    \end{figure}
    \subsection{Накапливающий сумматор}
    \begin{figure}[H]
        \centering
        \includegraphics[width=\linewidth]{scheme_last}
        \caption{Разработанная схема}
    \end{figure}
    \begin{figure}[H]
        \centering
        \includegraphics[width=\linewidth]{wave_last}
        \caption{Временная диаграмма}
    \end{figure}
    \section{Вывод}
    В ходе работы получены навыки исследования двоичных счетчиков, построенных по различной архитектуре,
    такие как счетчики с последовательным переносом и счетчики, основанные на мегафункции. Было проведено
    исследование зависимости максимальной частоты работы счетчика от количества регистров. Были исследованы
    различные устройства на основе счетчиков, такие как делитель частоты на число, устройство фиксации коротких
    импульсов, генератор треугольного сигнала, двоично-десятичный счетчик, преобразователь из двоичного кода в
    двоично-десятичный и накапливающий сумматор.
\end{document}
