\section{Разделенные разности и их связь с конечными разностями.}\label{sec:ch04}
Для равноотстоящих узлов таблицы конечные разности являются хорошей характеристикой изменения функции, аналогичной
производной для непрерывного случая. При произвольном расположении узлов таблицы целесообразно ввести понятие
\emph{разделенной разности}.
\begin{definition}[Разделенная разность]
    Разделенные разности нулевого порядка совпадают со значениями функции, а разности первого порядка определяются
    равенством
    \begin{equation}
        f(x_{n-1};x_n) = \frac{f(x_n) - f(x_{n-1})}{x_n - x_{n-1}}
    \end{equation}
\end{definition}
Аналогично строятся разделенные разности высших порядков. При этом разности $k$-го порядка определяются через разности
$(k-1)$-го порядка по формуле
\begin{equation}
    f(x_0;x_1;\dots;x_k) = \frac{f(x_1; x_2; \dots ; x_k) - f(x_0 ; x_1 ; \dots ; x_{k-1})}{x_k - x_0}
\end{equation}
Подобно конечным разностям, разделенные тоже можно выразить через значения функции в различных точках. По индукции можно
доказать следующее равенство:
\begin{equation}
    f(x_i;x_{i+1};\dots;x_{i+k}) = \sum_{j=i}^{i+k} \frac{f(x_j)}{\displaystyle \prod_{j \neq i} (x_j - x_i)}
\end{equation}
Отсюда следует важное свойство разделенных разностей: они являются симметричными функциями своих аргументов.
\begin{equation*}
    f(x_n; x_{n-1}) = f(x_{n-1}; x_n)
\end{equation*}
Если в исходной таблице узлы равноотстоящие, то для описания поведения функции можно использовать как конечные разности,
так и разделенные. Установим связь между ними. Обобщенную формулу можно доказать по индукции.
\begin{equation*}
    f(x_i; x_{i+1}) = \frac{f(x_{i+1}) - f(x_i)}{x_{i+1} - x_i} = \frac{\Delta f_i}{h} \quad \dots
\end{equation*}
\vspace{-20pt}
\begin{theorem}[Связь конечных и разделенных разностей]
    \begin{equation}
        f(x_i; x_{i+1}; \dots ; x_{i+k-1}; x_{i+k}) = \frac{\Delta^k f_i}{k!\,h^k}
    \end{equation}
\end{theorem}
