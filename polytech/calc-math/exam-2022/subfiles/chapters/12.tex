\section{Задача численного дифференцирования. Влияние вычислительной погрешности.}\label{sec:ch12}

\subsection{Задача численного дифференцирования.}
Предлагаемая задача ставится следующим образом. Для таблично заданной функции $f(x)$ требуется оценить значения
производной функции в узлах таблицы.

Идея, лежащая в основе численного дифференцирования, крайне проста и уже использовалась при получении квадратурных
формул. Исходная функция аппроксимируется интерполяционным полиномом
\begin{equation*}
    f(x) = Q_m(x) + R_m(x)
\end{equation*}
и производная от полинома дает формулу численного дифференцирования
\begin{equation*}
    \frac{df(x_k)}{dx} \approx \frac{dQ_m(x_k)}{dx}
\end{equation*}
а производная от остаточного члена позволяет оценить погрешность этой операции
\begin{equation*}
    \varepsilon = \frac{dR_m(x_k)}{dx}
\end{equation*}
Ограничимся случаем, когда узлы таблицы будут равноотстоящими с шагом $h = x_{k+1} - x_k$. Начнем с полинома первой
степени, построенного по двум узлам $x_k$ и $x_{k+1}$
\begin{equation*}
    Q_1(x) = \frac{x - x_{k+1}}{x_k - x_{k+1}}f(x_k) + \frac{x - x_k}{x_{k+1} - x_k}f(x_{k+1})
\end{equation*}
Дифференцируя его и полагая последовательно $x = x_k$ и $x = x_{k+1}$, получаем
\begin{flalign}
    &\frac{df(x_k)}{dx} \approx \frac{f_{k+1} - f_k}{h} \label{eq:num_diff_1} \\
    &\frac{df(x_{k+1})}{dx} \approx \frac{f_{k+1} - f_k}{h} \label{eq:num_diff_2}
\end{flalign}
Хотя правые части обоих выражений равны, формулы получились принципиально различными. Выполним аналогичные операции для
полинома второй степени
\begin{flalign*}
    \displaystyle
    Q_2(x) = &\frac{(x - x_{k+1})(x-x_{k+2})}{(x_k - x_{k+1})(x_k - x_{k+2})}f(x_k) + \frac{(x-x_k)(x-x_{k+2})}{(x_{k+1} - x_k)(x_{k+1}-x_{k+2})}f(x_{k+1}) +\\
    &+\frac{(x-x_k)(x-x_{k+1})}{(x_{k+2} - x_k)(x_{k+1} - x_{k+2})}f(x_{k+2})
\end{flalign*}
Последовательно полагая $x = x_k$, $x = x_{k+1}$ и $x = x_{k+2}$, получаем
\begin{flalign}
    &\frac{df(x_k)}{dx} \approx \frac{-3f_k + 4f_{k+1} - f_{k+2}}{2h} \label{eq:num_diff_3} \\
    &\frac{df(x_{k+1})}{dx} \approx \frac{f_{k+2} - f_k}{2h} \label{eq:num_diff_4} \\
    &\frac{df(x_{k+2})}{dx} \approx \frac{3f_{k+2} -4f_{k+1} + f_k}{2h} \label{eq:num_diff_5}
\end{flalign}
Для оценки погрешности всех формул необходимо продифференцировать остаточный член $R_m(x)$. Для полинома первой степени
он имеет вид
\begin{equation*}
    R_1(x) = \frac{(x-x_k)(x-x_{k+1})}{2!}f''(\eta)
\end{equation*}
\begin{equation*}
    \frac{dR_1(x)}{dx} = \frac{x-x_{k+1}}{2!}f''(\eta) + \frac{x-x_k}{2!}f''(\eta) + \frac{(x-x_k)(x-x_{k+1})}{2!}f'''(\eta)\eta'(x)
\end{equation*}
Спасает ситуацию то, что оценивать погрешность нужно в узлах интерполирования. Так как при $x = x_k$ и $x = x_{k+1}$ два
слагаемых из трех в этой формуле обращаются в ноль
\begin{flalign}
    &\varepsilon_1(x_k) = \frac{dR_1(x_k)}{dx} = -\frac{h}{2}f''(\eta)\\ \label{eq:num_diff_extr1}
    &\varepsilon_1(x_{k+1}) = \frac{dR_1(x_{k+1})}{dx} = \frac{h}{2}f''(\eta)
\end{flalign}
Эти два выражения задают погрешность численного дифференцирования для формул~\eqref{eq:num_diff_1} и~\eqref{eq:num_diff_2}
соответственно. Аналогично продифференцируем погрешность $R_2(x)$
\begin{equation*}
    R_2(x) = \frac{(x-x_k)(x-x_{k+1})(x-x_{k+2})}{3!}f'''(\eta)
\end{equation*}
Последовательно подставляя в результат $x = x_k$, $x = x_{k+1}$ и $x = x_{k+2}$, получаем
\begin{flalign}
    &\varepsilon_2(x_k) = \frac{dR_2(x_k)}{dx} = \frac{h^2}{3}f'''(\eta)\\
    &\varepsilon_2(x_{k+1}) = \frac{dR_2(x_{k+1})}{dx} = -\frac{h^2}{6}f'''(\eta)\\
    &\varepsilon_2(x_{k+2}) = \frac{dR_2(x_{k+2})}{dx} = \frac{h^2}{3}f'''(\eta)
\end{flalign}

\emph{Заметим, что} на меньшую погрешность можно рассчитывать, используя формулу~\eqref{eq:num_diff_4}, которая и
является наиболее популярной на практике. Формулы~\eqref{eq:num_diff_3} и~\eqref{eq:num_diff_5} используются для
дифференцирования в начале и в конце таблицы соответственно.
\vspace{10pt}

Если интерполяционный полином второй степени продифференцировать дважды, то получается простейшая формула для второй
производной
\begin{equation}
    \frac{d^2 f(x_{k+1})}{dx^2} \approx \frac{f_{k+2} - 2f_{k+1} + f_k}{h^2}
\end{equation}
\vspace{10pt}

Практически важным является вопрос о выборе шага $h$ для формул численного дифференцирования. Ограничение сверху
накладывается величиной погрешности $\varepsilon_2$, а снизу -- точностью задания табличных данных для $f(x)$.

\subsection{Влияние погрешности задания функции на точность.}
В качестве примера вновь обратимся к простейшей формуле для первой производной~\eqref{eq:num_diff_1}. Пусть в ней
значение $f_{k+1}$ определено с погрешностью $\Delta_{k+1}$, а значение $f_k$ -- $\Delta_k$. Тогда общая погрешность $\varepsilon(h)$
складывается из двух погрешностей
\begin{equation*}
    \varepsilon(h) = \varepsilon_1(h) + \varepsilon_2(h)
\end{equation*}
первая из которых задается формулой~\eqref{eq:num_diff_extr1} и примерно линейно убывает с уменьшением шага $h$, а вторая
зависит от $\Delta_k$ и $\Delta_{k+1}$. Оценивая полную погрешность сверху, получаем
\begin{equation}
    \left| \varepsilon(h) \right| \leq \left| \varepsilon_1(h) \right| + \left| \varepsilon_2(h) \right| = \frac{h}{2} \left| f''(\eta) \right| + \frac{\left| \Delta_{k+1} - \Delta_k \right|}{h} \label{eq:num_diff_extr_full}
\end{equation}
Оптимальное значение шага $h_{opt}$ отвечает ситуации, когда оба слагаемых равны друг другу. На практике в точном
определении $h_{opt}$ нет необходимости, важно лишь знать о характере зависимости~\eqref{eq:num_diff_extr_full}.
