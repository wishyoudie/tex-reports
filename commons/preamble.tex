\documentclass[a4paper,12pt]{article}
\usepackage[utf8x]{inputenc}
\usepackage[T1,T2A]{fontenc}
\usepackage[russian]{babel}

\usepackage[left=2cm,right=2cm, top=2cm,bottom=2cm,bindingoffset=0cm]{geometry}
\usepackage{amsmath}
\usepackage{enumitem}

\bibliographystyle{ugost2008ls}

\usepackage{tikz}

%% Images
\usepackage{graphicx} % Inputing images
\usepackage{xcolor} % Colors
\usepackage{subcaption} % Captioning images

%% Нумерация картинок по секциям
\usepackage{chngcntr}
\counterwithin{figure}{section}
\counterwithin{table}{section}

%%Точки нумерации заголовков
\usepackage{titlesec}
\titlelabel{\thetitle.\quad}
\usepackage[dotinlabels]{titletoc}

\usepackage[normalem]{ulem}
\useunder{\uline}{\ul}{}

\setlist{noitemsep} % No spacing between list items

%% Внесение titlepage в учёт счётчика страниц
\renewenvironment{titlepage}{\thispagestyle{empty}}

\usepackage{fancyhdr}
\usepackage{float}
\usepackage{multirow}
\usepackage{upgreek} % Cool greek letters
\usepackage{bigfoot}
\usepackage{fancyvrb}
\usepackage{fvextra}

\usepackage[colorlinks,urlcolor = blue, filecolor=blue,citecolor=blue, linkcolor = blue, unicode=true]{hyperref}

\usepackage[framemethod=TikZ]{mdframed}
\newcommand{\definebox}[2]{%
  \newcounter{#1}
  \newenvironment{#1}[1][]{%
    \stepcounter{#1}%
    \mdfsetup{%
        frametitle={%
            \tikz[baseline=(current bounding box.east),outer sep=0pt]
            \node[anchor=east,rectangle,fill=white]
            {\strut \MakeUppercase#1~\csname the#1\endcsname\ifstrempty{##1}{}{:~##1}};}}%
    \mdfsetup{innertopmargin=1pt,linecolor=#2,%
        linewidth=3pt,topline=true,
        frametitleaboveskip=\dimexpr-\ht\strutbox\relax,}%
    \begin{mdframed}[]\relax%
    }{\end{mdframed}}%
}
\definebox{definition}{orange!90}

% Листинги
\usepackage{listings} % для самих листингов
\definecolor{commentgreen}{RGB}{2,112,10}
\lstdefinestyle{cpp} {
    language=C++,
    frame=tb,
    tabsize=4,
    showstringspaces=false,
    numbers=left,
    captionpos=b,
    columns=flexible,
    upquote=true,
    commentstyle=\color{commentgreen},
    keywordstyle=\color{blue},
    stringstyle=\color{commentgreen},
    basicstyle=\small\ttfamily,
    emph={int,char,double,float,unsigned,void,bool,size\_t},
    emphstyle={\color{blue}},
    escapechar=\&,
    classoffset=1,
    otherkeywords={>,<,.,;,-,!,=,~},
    morekeywords={>,<,.,;,-,!,=,~},
    keywordstyle=\color{black},
    classoffset=0,
}
\lstdefinestyle{def} {
    frame=tb,
    tabsize=4,
    showstringspaces=false,
    numbers=left,
    captionpos=b,
    columns=flexible,
    upquote=true,
    commentstyle=\color{black},
    keywordstyle=\color{black},
    stringstyle=\color{black},
    basicstyle=\small\ttfamily,
    emph={int,char,double,float,unsigned,void,bool,size\_t},
    emphstyle={\color{black}},
    escapechar=\&,
    classoffset=1,
    otherkeywords={>,<,.,;,-,!,=,~},
    morekeywords={>,<,.,;,-,!,=,~},
    keywordstyle=\color{black},
    classoffset=0,
}

\lstdefinelanguage[RISC-V]{Assembler}
{
    alsoletter={.}, % allow dots in keywords
    alsodigit={0x}, % hex numbers are numbers too!
    morekeywords=[1]{ % instructions
    lb, lh, lw, lbu, lhu,
    sb, sh, sw,
    sll, slli, srl, srli, sra, srai,
    add, addi, sub, lui, auipc,
    xor, xori, or, ori, and, andi,
    slt, slti, sltu, sltiu,
    beq, bne, blt, bge, bltu, bgeu,
    j, jr, jal, jalr, ret,
    scall, break, nop
},
    morekeywords=[2]{ % sections of our code and other directives
    .align, .ascii, .asciiz, .byte, .data, .double, .extern,
    .float, .globl, .half, .kdata, .ktext, .set, .space, .text, .word
},
    morekeywords=[3]{ % registers
    zero, ra, sp, gp, tp, s0, fp,
    t0, t1, t2, t3, t4, t5, t6,
    s1, s2, s3, s4, s5, s6, s7, s8, s9, s10, s11,
    a0, a1, a2, a3, a4, a5, a6, a7,
    ft0, ft1, ft2, ft3, ft4, ft5, ft6, ft7,
    fs0, fs1, fs2, fs3, fs4, fs5, fs6, fs7, fs8, fs9, fs10, fs11,
    fa0, fa1, fa2, fa3, fa4, fa5, fa6, fa7
},
    morecomment=[l]{;},   % mark ; as line comment start
    morecomment=[l]{\#},  % as well as # (even though it is unconventional)
    morestring=[b]",      % mark " as string start/end
    morestring=[b]'       % also mark ' as string start/end
}

% usage example:

% define some basic colors
\definecolor{mauve}{rgb}{0.58,0,0.82}

\lstdefinestyle{riscv} {
    basicstyle=\small\ttfamily,                    % very small code
    breaklines=true,                              % break long lines
    commentstyle=\itshape\color{green!50!black},  % comments are green
    keywordstyle=[1]\color{blue!80!black},        % instructions are blue
    keywordstyle=[2]\color{orange!80!black},      % sections/other directives are orange
    keywordstyle=[3]\color{red!50!black},         % registers are red
    stringstyle=\color{mauve},                    % strings are from the telekom
    identifierstyle=\color{teal},                 % user declared addresses are teal
    frame=l,                                      % black line on the left side of code
    captionpos=b,
    language=[RISC-V]Assembler,                   % all code is RISC-V
    tabsize=4,                                    % indent tabs with 4 spaces
    showstringspaces=false                        % do not replace spaces with weird underlines
}
%% Final touch
\usepackage{subfiles}
