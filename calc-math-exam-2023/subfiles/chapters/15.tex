\section{Ортогонализация по Шмидту. Примеры ортогональных полиномов.}\label{sec:ch15}

\subsection{Процедура ортогонализации Грама-Шмидта.}
Для начала обозначим задачу. Пусть задан набор линейно независимых функций $\left\{ \varphi_k(x) \right\}$. Требуется построить
набор ортогональных функций $\left\{ g_k(x) \right\}$, которые будут являться линейной комбинацией функций
$\left\{ \varphi_k(x) \right\}$. Аппроксимация, таким образом, будет выполняться в том же классе функций.

Введем следующее обозначение: $\tilde{g}_k(x)$ -- функции ортогональные, но еще не нормированные. Очередная функция
$\tilde{g}_k(x)$ строится так, чтобы она была ортогональна всем $\tilde{g}_i(x)$, построенным до нее.
\vspace{10pt}

Шаг 1.

$\displaystyle \tilde{g}_0(x) = \varphi_0(x)$. Нормируем ее.
\begin{equation*}
    \int_{a}^{b} p(x) \tilde{g}_0^2 dx = \alpha_0^2, \quad g_0(x) = \frac{\tilde{g}_0}{\alpha_0}
\end{equation*}

Шаг $m$.

Функция $\tilde{g}_m(x)$ строится с привлечением новой $\varphi_m(x)$ и добавлением линейной комбинации функций
$\displaystyle g_k(x)$, построенных на предыдущих шагах.
\begin{equation*}
    \tilde{g}_m(x) = \varphi_m(x) - \sum_{k=0}^{m-1} C_{m,k} \cdot g_k(x)
\end{equation*}

Из условия ортогональности $\displaystyle \tilde{g}_m(x)$ и $\displaystyle g_i(x)$ получаем выражения для коэффициентов
$C_{m, i}$
\begin{equation*}
    \int_{a}^{b} p(x) \cdot \tilde{g}_m(x) \cdot g_i(x)dx = \int_{a}^{b} p(x) \cdot \varphi_m(x) \cdot g_i(x) dx - \sum_{k=0}^{m-1} C_{m, k} \cdot \int_{a}^{b} p(x) \cdot g_k(x) \cdot g_i(x)dx = 0
\end{equation*}
Все интегралы под знаком суммы, кроме одного, равны нулю и для $\displaystyle C_{m, i}$ имеем
\begin{equation*}
    C_{m,i} = \frac{\displaystyle \int_{a}^{b} p(x) \cdot \varphi_m(x) \cdot g_i(x) dx}{\displaystyle \int_{a}^{b} p(x) \cdot g_i^2(x)dx}, \quad \int_{a}^{b} p(x) \tilde{g}_m^2(x) dx = \alpha_m^2, \quad g_m(x) = \frac{\tilde{g}_m}{\alpha_m}
\end{equation*}

\subsection{Примеры ортогональных полиномов.}
Неотъемлемыми атрибутами понятия ортогональности являются промежуток интегрирования и весовая функция. Проблема различных
промежутков, возникающих на практике, решается легко. Полиномы для стандартных промежутков ($[-1, 1]$, $[0, 1]$)
приводятся в справочниках и учебниках, а к произвольному промежутку переходят обычной заменой переменных. Примером
является следующая замена
\begin{equation*}
    x = \frac{a + b}{2} + \frac{b - a}{2}t, \qquad x \in [a, b], \quad t \in [-1, 1]
\end{equation*}
В приводимых примерах остановимся на промежутке $[-1, 1]$. Тогда главной отличительной особенностью различных полиномов
будет весовая функция $p(x)$.
\begin{enumerate}
    \item Ортогональные полиномы Лежандра
    \vspace{5pt}

    Для этих полиномов весовая функция имеет популярный вид: $p(x) \equiv 1$ и сами они могут быть вычислены по формуле
    \begin{equation}
        \displaystyle L_n(x) = \frac{(-1)^n}{n!\, 2^n} \frac{d^n}{dx^n} \left[ \left( 1 - x^2 \right)^n \right], \qquad x \in [-1, 1] \label{eq:legendre}
    \end{equation}
    \emph{Заметим, что} $ \displaystyle L_0(x) = 1, L_1(x) = x$, а следующие полиномы можно найти при помощи рекуррентной
    формулы:
    \begin{equation*}
    (n+1)
        L_{n+1}(x) - (2n+1)xL_n(x) + nL_{n-1}(x) = 0
    \end{equation*}
    Применив ее, получаем
    \begin{equation*}
        L_2(x) = \frac{3x^2-1}{2}, \quad L_3(x) = \frac{5x^3-3x}{2}, \quad L_4(x) = \frac{35x^4-30x^2+3}{8}\dots
    \end{equation*}
    В такой форме полиномы Лежандра ортогональны, но не нормированы. Квадрат нормы:
    \begin{equation*}
        \int_{-1}^{1} L_n^2(x)dx = \frac{2}{2n+1}
    \end{equation*}

    \item Ортогональные полиномы Чебышева
    \vspace{5pt}

    Для этих полиномов весовая функция выглядит следующим образом:
    \begin{equation*}
        p(x) = \frac{1}{\sqrt{1-x^2}}.
    \end{equation*}
    При $x \in [-1, 1]$ они могут быть вычислены по формуле
    \begin{equation}
        T_0(x) = 1, \quad T_1(x) = x, \quad T_n(x) = \cos \left( n \cdot \arccos(x) \right) \label{eq:chebyshev}
    \end{equation}
    Как и для полиномов Лежандра, здесь имеет место следующее рекуррентное соотношение:
    \begin{equation*}
        T_{n+1}(x) = 2xT_n(x) - T_{n-1}(x)
    \end{equation*}
    Применив ее, получаем
    \begin{equation*}
        T_2(x) = 2x^2 - 1, \quad T_3(x) = 4x^3 -3x, \quad T_4(x) = 8x^4 -8x^2 + 1\dots
    \end{equation*}
    Полиномы Чебышева могут быть представлены и в другом виде:
    \begin{equation}
        T_n(x) = \frac{(-2)^n n!}{(2n)!}\sqrt{1-x^2}\frac{d^n}{dx^n}\left[ \left( 1 - x^2 \right)^{n-1/2} \right] \label{eq:chebyshev2}
    \end{equation}
    Квадраты нормы:
    \begin{equation*}
    (T_0, T_0)
        = \pi, \qquad (T_n, T_n) = \int_{-1}^{1} \frac{T_n^2(x)}{\sqrt{1-x^2}}dx = \frac{\pi}{2}
    \end{equation*}
\end{enumerate}
