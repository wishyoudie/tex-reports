\section{Метод последовательных приближений для решения линейных систем.}\label{sec:ch21}
\emph{Итерационные методы} (еще одно название -- \emph{методы последовательных приближений}) дают возможность для
системы~\eqref{eq:nonlinear_algebraic_equation} строить последовательность векторов $\bf{x}_0, \bf{x}_1, \dots$,
пределом которой должно быть точное решение $\bf{x}^{*}$
\begin{equation*}
    \bf{x}^{*} = \lim_{n \rightarrow \infty} \bf{x}_n
\end{equation*}
На практике построение последовательности обрывается как только достигается желаемая точность. Чаще всего для достаточно
малого значения $\varepsilon > 0$ контролируется выполнение оценки $\displaystyle \left| \bf{x}^* - \bf{x}_n \right| < \varepsilon$. Метод
последовательных приближений может быть построен, например, по следующей схеме. Эквивалентными преобразованиями приведем
систему~\eqref{eq:nonlinear_algebraic_equation} к виду
\begin{equation}
    \bf{x} = \bf{C} \bf{x} + \bf{d} \label{eq:nonlinear_algebraic_equation_iter}
\end{equation}
Под эквивалентными преобразованиями будем понимать преобразования, сохраняющие решение системы (т.е.
решения~\eqref{eq:nonlinear_algebraic_equation} и~\eqref{eq:nonlinear_algebraic_equation_iter} совпадают). Точное решение
$\bf{x}^*$ системы имеет вид
\begin{equation*}
    \bf{x}^* = \left( \bf{E} - \bf{C} \right)^{-1} \bf{d}
\end{equation*}
Вместо~\eqref{eq:nonlinear_algebraic_equation_iter} будем решать следующую систему разностных уравнений
\begin{equation}
    \bf{x}_{n+1} = \bf{C}\bf{x}_n + \bf{d} \label{eq:nonlinear_sub_iter}
\end{equation}
пошаговым методом. При этом необходимо решить целый ряд вопросов. Сходится ли итерационный процесс~\eqref{eq:nonlinear_sub_iter}?
Если сходится, что является пределом последовательности, и какова скорость сходимости?

Ранее было показано, что решение системы~\eqref{eq:nonlinear_sub_iter} записывается в виде
\begin{equation*}
    \bf{x}_n = \bf{C}^n \bf{x}_0 + \left( \bf{E} - \bf{C}^n \right)\left( \bf{E} - \bf{C} \right)^{-1} \bf{d}
\end{equation*}
Вычитания из него точное решение, получаем
\begin{equation*}
    \bf{x}_n - \bf{x}^* = \bf{C}^n \bf{x}_0 - \bf{C}^n \left( \bf{E} - \bf{C} \right)^{-1}\bf{d} = \bf{C}^n \left( \bf{x}_0 - \bf{x}^* \right)
\end{equation*}
Чтобы обеспечить условие сходимости, все элементы матрицы $\bf{C}^n$ должны стремиться к нулю при $n \rightarrow \infty$.
Для этого, в свою очередь, необходимо и достаточно, чтобы \emph{все} собственные значения матрицы \bf{C} были по модулю
меньше единицы
\begin{equation*}
    |\uplambda_k| < 1
\end{equation*}
Поскольку нахождение всех собственных значений доставляет значительные трудности, вместо этого условия можно использовать
достаточное условие сходимости
\begin{equation*}
    \left\| \bf{C} \right\| < 1
\end{equation*}
которое справедливо для любой канонической нормы.

Количество итераций по формуле~\eqref{eq:nonlinear_sub_iter} будет тем меньше, чем меньше по модулю собственные значения
матрицы \bf{C} и чем ближе к $\bf{x}^*$ выбрано начальное приближение $\bf{x}_0$. На практике при реализации на
компьютере процесс~\eqref{eq:nonlinear_sub_iter} прерывается либо заданием максимального числа итераций, либо условием
$\displaystyle \left\| \bf{x}_{n+1} - \bf{x}_n \right\| < \varepsilon$. Таким образом, основным неформальным моментом
является такое приведение системы~\eqref{eq:nonlinear_algebraic_equation} к виду~\eqref{eq:nonlinear_algebraic_equation_iter},
чтобы выполнялось условие ограниченности собственных значений. В общем случае универсальный способ такого перехода с
малой трудоемкостью отсутствует, и поэтому часто используется специфика решаемой задачи. Рассмотрим следующий пример.

Пусть диагональные элементы матрицы \bf{A} в~\eqref{eq:nonlinear_algebraic_equation} значительно превышают по модулю
остальные элементы в соответствующих строках. Разделим каждое уравнение на соответствующий диагональный элемент и получим
\begin{equation*}
    \tilde{\bf{A}} \bf{x} = \tilde{\bf{b}}, \qquad \bf{x} = \left( \bf{E} - \tilde{\bf{A}} \right) \bf{x} + \tilde{\bf{b}}
\end{equation*}
На главной диагонали у матрицы $\tilde{\bf{A}}$ стоят единицы, а у матрицы $\displaystyle \left( \bf{E} - \tilde{\bf{A}} \right)$
расположены нули. Вне главной диагонали у обеих матриц находятся малые по модулю элементы, что позволяет, выбрав
$\bf{C} = \bf{E} - \tilde{\bf{A}}$, легко обеспечить условие ограниченности нормы и быструю сходимость итерационного
процесса.
