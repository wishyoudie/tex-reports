\section{Выбор узлов интерполирования. Интерполяционный полином Ньютона для равно и неравноотстоящих узлов.}\label{sec:ch07}

\subsection{Выбор узлов интерполирования}
Для уменьшения погрешности интерполирования обратимся к теореме об остаточном члене полинома Лагранжа при заданной
степени полинома $m$. Поскольку величиной $\displaystyle f^{(m+1)}(\eta)$ трудно управлять, и возможна лишь оценка пределов
ее изменения, задача уменьшения погрешности сводится к управлению величиной $|\omega(x)|$ за счет выбора узлов
интерполирования. Рассмотрим два типичных на практике случая.

\emph{Случай 1}. Задана степень полинома $m$ и имеется таблица достаточно большой длины. Точка $x^{*}$, в которой
вычисляется значение полинома, заранее известна. Требуется выбрать $m+1$ узел так, чтобы величина $|\omega(x^{*})|$ была бы
минимальна.

Результат очевиден. Нужно выбирать узлы интерполирования из таблицы, \emph{ближайшие} к $x^{*}$. Использование любого
другого узла вместо ближайшего неизбежно увеличивает значение
\begin{equation*}
    |\omega(x^{*})| = |\left( x^{*} - x_0 \right)\left( x^{*} - x_1 \right)\dots\left( x^{*} - x_m \right)|
\end{equation*}

\emph{Случай 2}. Заданы степень полинома $m$ и промежуток интерполирования $[a, b]$. Точка $x^{*}$, в которой
вычисляется значение полинома, заранее не известна. Требуется выбрать узлы интерполирования так, чтобы в самом
неблагоприятном случае расположения $x^{*}$ погрешность была бы минимальна (т.н. \emph{минимаксный критерий})
\begin{equation*}
    \max_{[a, b]} |\omega(x^{*})| \rightarrow \min
\end{equation*}
Интуитивно напрашивающееся предложение о равномерном задании узлов на промежутке оказывается ошибочным. Значения $|\omega(x)|$
в узлах интерполирования равны нулю, график напоминает <<колокольчики>>, максимум которых достигается между узлами
интерполирования. При выборе равноотстоящих узлов погрешность для $x^{*}$, близких к центру промежутка интерполирования
оказывается небольшой, однако ближе к концам она сильно возрастает. Узлы интерполирования нужно симметрично сместить
ближе к концам промежутка. Тогда высота центрального <<колокольчика>> увеличится, в то время как высота крайних
уменьшится. Оптимальный выбор узлов интерполирования отвечает нулям так называемых ортогональных полиномов Чебышева,
когда все <<колокольчики>> будут одинаковыми по высоте.

\subsection{Интерполяционный полином Ньютона для равно и неравноотстоящих узлов.}
Оценка погрешности на основе формулы
\begin{equation*}
    R_m(x) = \frac{f^{(m+1)}(\eta)}{(m+1)!}\omega(x)
\end{equation*}
выполняется крайне редко из-за известных трудностей, связанных с оценкой производной $\displaystyle f^{(m+1)}(\eta)$
особенно для таблично заданной функции. Поэтому на практике о величине погрешности принято судить, сравнивая в заданной
точке $x^{*}$ значения интерполяционных полиномов соседних степеней $Q_m(x^{*}) \text{ и } Q_{m+1}(x^{*})$. При
недостаточной точности последовательно повышают степень полинома. Но для такой процедуры использование полинома Лагранжа
оказывается неэффективным. При переходе к полиному следующей степени всю работу приходится выполнять заново.
Целесообразно записать полином в таком виде, чтобы расчеты сводились к появлению лишь еще одного слагаемого в дополнение
к ранее вычисленному $Q_m(x)$. С этой целью запишем первую разделенную разность
\begin{equation*}
    f(x; x_0) = \frac{f(x) - f(x_0)}{x - x_0}
\end{equation*}
и выразим из нее $f(x)$
\begin{equation}
    f(x) = f(x_0) + (x - x_0)f(x; x_0)\label{eq:newton_proof_1}
\end{equation}
\emph{Заметим, что} первое слагаемое в первой части это интерполяционный полином нулевой степени, а второе слагаемое --
погрешность полинома. Теперь запишем вторую разделенную разность
\begin{equation*}
    f(x; x_0; x_1) = \frac{f(x; x_0) - f(x_0; x_1)}{x - x_1}
\end{equation*}
выразим из нее первую разность через вторую и поставим в формулу~\eqref{eq:newton_proof_1}
\begin{equation}
    f(x) = f(x_0) + (x - x_0)f(x_1; x_0) + (x - x_0)(x - x_1)f(x; x_0; x_1)
\end{equation}
Продолжая этот процесс, получаем
\begin{flalign*}
    f(x) = &f(x_0) + (x - x_0)f(x_1; x_0) + (x - x_0)(x - x_1)f(x_0; x_1; x_2) + \dots + \\
    &+(x-x_0)(x-x_1)\dots(x-x_{m-1})f(x_0;x_1;x_2;\dots x_m) + \\
    &+(x-x_0)(x-x_1)\dots(x-x_m)f(x;x_0;x_1;x_2;\dots x_m) = \\
    &= Q_m(x) + \omega(x)f(x;x_0;x_1;x_2;\dots x_m)
\end{flalign*}
Сумма первых $k$ слагаемых порождает интерполяционный полином степени $k - 1$, а последнее
слагаемое является погрешностью интерполяционного полинома степени $m$. При этом структура полинома такова, что полином
степени $m$ получается как полином степени $m - 1$ с добавлением еще одного слагаемого.
\begin{definition}[Интерполяционный полином Ньютона]
    Это интерполяционный полином в форме
    \begin{equation}
        Q_m(x) = Q_{m-1}(x) + (x- x_0)(x - x_1) \dots (x - x_{m-1})f(x_0; x_1; x_2;\dots x_m)\label{eq:newton_polynom}
    \end{equation}
\end{definition}
\vspace{20pt}

На практике вычисление разделенных разностей производится в рамках следующей таблицы, где появление новой разделенной
разности более высокого порядка связано с построением еще одной диагонали
\begin{table}[H]
    \centering
    \begin{tabular}{|c|c|c|c|c|c|}
        \hline
        $x_0$ & $f(x_0)$ & $f(x_0;x_1)$ & $f(x_0;x_1;x_2)$ & $f(x_0;x_1;x_2;x_3)$ & $f(x_0;x_1;x_2;x_3;x_4)$ \\ \hline
        $x_1$ & $f(x_1)$ & $f(x_1;x_2)$ & $f(x_1;x_2;x_3)$ & $f(x_1;x_2;x_3;x_4)$ &                          \\ \hline
        $x_2$ & $f(x_2)$ & $f(x_2;x_3)$ & $f(x_2;x_3;x_4)$ &                      &                          \\ \hline
        $x_3$ & $f(x_3)$ & $f(x_3;x_4)$ &                  &                      &                          \\ \hline
        $x_4$ & $f(x_4)$ &              &                  &                      &                          \\ \hline
    \end{tabular}
\end{table}

Пускай узлы таблицы задания функции являются равноотстоящими. В этом случае разделенные разности можно заменить на
конечные, тем самым избежав деления на разность значений аргумента. Используя
\begin{equation*}
    \frac{x-x_k}{h} = \frac{x-x_0-kh}{h} = \frac{x-x_0}{h} -k = t - k
\end{equation*}
и формулу связи разделенных и конечных разностей
\begin{equation*}
    f(x_i; x_{i+1}; \dots; x_{i+k-1}; x_{i+k}) = \frac{\displaystyle \Delta^k f_i}{\displaystyle k!\, h^k}
\end{equation*}
получим версию полинома Ньютона для равноотстоящих узлов
\begin{equation}
    Q_m(x_0 + ht) = f(x_0) + \frac{t}{1!}\Delta f(x_0) + \frac{t(t-1)}{2!}\Delta^2 f(x_0) + \dots + \frac{t(t-1)\dots (t - m + 1)}{m!}\Delta^m f(x_0)\label{eq:newton_polynom_2}
\end{equation}

Новая независимая переменная $t$ принимает в узлах таблицы целые значения, а вычисление конечных разностей реализуется
подобно разделенным разностям по следующей таблице
\begin{table}[H]
    \centering
    \begin{tabular}{|c|c|c|c|c|c|}
        \hline
        $x_0$ & $f(x_0)$ & $\Delta f_0$ & $\Delta^2 f_0$ & $\Delta^3 f_0$ & $\Delta^4 f_0$ \\ \hline
        $x_1$ & $f(x_1)$ & $\Delta f_1$ & $\Delta^2 f_1$ & $\Delta^3 f_1$ &                \\ \hline
        $x_2$ & $f(x_2)$ & $\Delta f_2$ & $\Delta^2 f_2$ &                &                \\ \hline
        $x_3$ & $f(x_3)$ & $\Delta f_3$ &                &                &                \\ \hline
        $x_4$ & $f(x_4)$ &              &                &                &                \\ \hline
    \end{tabular}
\end{table}