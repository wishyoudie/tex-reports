\documentclass[polytech/stats/exam-2023/stats-exam-2023.tex]{subfiles}
\graphicspath{{polytech/stats/exam-2023/images/}}
\begin{document}
\section{Формула полной вероятности. Формула Байеса и ее практическое значение.}
    Пусть в условиях $S$ в $\Omega$ возможно событие $A$ и в $\Omega$ определена
    полная группа попарно непересекающихся событий $H_i$:
    \begin{equation*}
        H_i \in \Omega: H_i \bigcap_{i \neq j} H_j = \varnothing, \ \bigcup_{i = 1}^n H_i = T
    \end{equation*}

    Тогда, поскольку $H_i$ заполняют все пространство $\Omega$, $A = \bigcup_{i = 1}^n \left(A \cap H_i \right)$.
    Воспользовавшись аксиомой Колмогорова о счетой аддитивности вероятностной меры и математическим определением
    условных вероятностей, получим:
    \begin{equation*}
        P(A) = P\left(\bigcup_{i=1}^n \left(A \cap H_i\right) \right) = \sum_{i=1}^n P(A \cap H_i) = \sum_{i=1}^n P(A)P(H_i | A) =%
        \sum_{i=1}^n P(H_i) P(A | H_i)
    \end{equation*}
    Таким образом получена формула т.н. \textit{полной вероятности}, в которой $H_i$ называют \textit{гипотезами}:
    \begin{theorem}[Формула полной вероятности]
        \begin{equation*}
            P(A) = \sum_{i = 1}^n P(H_i) P(A | H_i)
        \end{equation*}
    \end{theorem}

    Далее, воспользовавшись формулой условной вероятности и формулой полной вероятности, получим формулу \textit{Байеса}:
    \begin{theorem}[Формула Байеса]
        \begin{equation*}
            P(H_i | A) = \frac{P(H_i \cap A)}{P(A)} = \frac{P(H_i) P(A | H_i)}{\sum_{i=1}^n P(H_i) P(A | H_i)}
        \end{equation*}
    \end{theorem}
    Формула Байеса эффективно используется при исследованиях природных явлений, при исследованиях и испытаниях рукотворных объектов
    в условиях неопределенности математической модели исследуемых объектов и действия мешающих случайных воздействий. 
    В этих условиях события, происходящие при исследованиях, неоднозначно связаны со свойствами и параметрами объектов.

    Пусть $H_i$ -- гипотезы (предположения) исследователя о свойствах или параметрах исследуемого объекта, рукотворного или природного.
    Эти гипотезы могут иметь одинаковый или различный приоритет, который выражается путем задания значений вероятностей $P(H_i)$.
    Эти вероятности в данной ситуации суть \textit{априорные вероятности} гипотез $H_i$.
    В результате эксперимента или исследования событие A происходит с той или иной вероятностью. Это событие исследователь фиксирует, и по
    нему он должен вынести суждение об оправданности того или иного априорного предположения (гипотезы). В силу действия случайных факторов и
    неопределенности математической модели объекта однозначные причинно-следственные связи между предположениями (гипотезами) 
    исследователя и результатами испытаний размыты. После выполнения эксперимента (испытания) фиксируется событие $A$.
    В этой ситуации можно оценить условные вероятности $P(A | H_i)$ возможности реализации события $A$ при справедливости каждой
    из гипотез. Таким образом после эксперимента правая часть формулы Байеса может быть рассчитана, и формула Байеса
    дает возможность оценить апостериорную вероятность той или иной гипотезы при условии, что результатом эксперимента оказалось событие $A$.

    Естественно принять в качестве наиболее правдоподобного то предположение (гипотезу), апостериорная вероятность которого окажется
    наибольшей. Такое правило принятия решения, которое основано на применении формулы Байеса, называется байесовским. Этой же
    формулой порожден принцип максимума апостериорной вероятности, который часто и эффективно используется в теории и практике
    систем автоматического регулирования, при математической обработке результатов измерений, при идентификации объектов.
    \newpage
\end{document}