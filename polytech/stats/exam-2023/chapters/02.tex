\documentclass[polytech/stats/exam-2023/stats-exam-2023.tex]{subfiles}
\graphicspath{{polytech/stats/exam-2023/images/}}
\begin{document}
\section{Предмет теории вероятностей, основные понятия и определения теории вероятностей, вероятность событий, аксиоматика и различные определения вероятности.}
    \textit{Вероятность} события $A$ -- это числовая характеристика возможности случайного события при условиях $S$. Если $\omega \in \Omega_A$,
    то вероятность события $A$ есть вероятностная мера множества $\Omega_A$, обозначается $P(A)$.
    
    Для вероятности существует несколько математических определений.
    \begin{definition}[Классическое определение вероятности]
        $P(A)$ есть отношение количества случаев, благоприятствующих появлению события $A$ к общему числу испытаний.
        \begin{equation*}
            P(A) = \frac{N_A}{N}
        \end{equation*}
    \end{definition}
    \begin{definition}[Частотное определение]
        \begin{equation*}
            P(A) = \lim_{n \to \infty } \frac{m}{n},
        \end{equation*}
        где $n$ -- общее число выполненных испытаний, $m$ -- количество случаев появления события $A$ при этих испытаниях.
    \end{definition}
    \begin{definition}[Современная аксиоматика]
        $P(A)$ -- неотрицательная монотонная счетно-аддитивная мера возможности случайного события, такая, что $P(T) = 1$.
        \vspace{0.5cm}
        
        Пояснения:
        \begin{itemize}
            \item \textit{неотрицательность}: $P(A) \geq 0$;
            \item \textit{монотонность}: если при наступлении события $A$ обязательно наступает событие $B$, но обратное
            необязательно, то есть, если $A \subset B$, то $P(A) \leq P(B)$.
            \item \textit{счетная аддитивность}: если условия $S$ определены, события $A_i$ попарно несовместны, то есть
            $A_i \cap A_j = \varnothing \ \forall \, i \neq j$, то
            \begin{equation*}
                P\left(\bigcup_{i=1}^n A_i\right) = \sum_{i=1}^n P\left(A_i\right).
            \end{equation*}
        \end{itemize}
    \end{definition}
    
    Следствия из аксиоматики Колмогорова:
    \begin{enumerate}
        \item $0 \leq P(A) \leq 1$;
        \item $T = T \cup \varnothing$;
        \item $P(T) = P(T) + P(\varnothing) = 1 + P(\varnothing) \Rightarrow P(\varnothing) = 0$;
        \item Пусть $A \subset \Omega, B \subset \Omega$, и $A$ и $B$ противоположны. Тогда $P(A \cup B) = P(A) + P(B) = P(T) = 1$,
        откуда $P(A) = 1 - P(B)$, то есть
        \begin{itemize}
            \item[] $P(A) = 1 - P(\overline{A})$,
            \item[] $P(B) = 1 - P(\overline{B})$;
        \end{itemize}
    \end{enumerate}

    В современной теории вероятностей вероятностная мера определяется на классах событий. Классы событий образуют таким образом,
    чтобы они давали возможность определить вероятностную меру вначале на простейших событиях, а затем распространить ее на события
    любой сложности. Для этого класс событий должен содержать в себе не только сходящиеся в этом классе последовательности событий,
    но также их пределы. Обозначим класс событий $\mathfrak{R}$. 
    
    \begin{definition}[Алгебра событий]
        Если в условиях $S$ события $A_i$ принадлежат классу событий $\mathfrak{R}$, и счетное объединение и пересечение этих событий также
        принадлежат этому классу, то этот класс называется \textit{алгеброй* событий}.
        \vspace{0.5cm}
        
        Если в этих же условиях
        \begin{equation*}
            A_i \in \mathfrak{R}, \quad \bigcup_{i=1}^{\infty} A_i \in \mathfrak{R}, \quad \bigcap_{i=1}^{\infty} A_i \in \mathfrak{R},
        \end{equation*}
        то такой класс событий называется \textit{сигма-алгеброй}.
        \vspace{0.5cm}
        
        *Пример алгебры: многоугольники на плоскости, мерой для которых является их площадь.
    \end{definition}
    \newpage
\end{document}