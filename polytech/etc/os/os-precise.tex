\documentclass[a4paper,12pt]{article}
%%% Default imports
\usepackage{listings} % Code listings
\usepackage{graphicx} % Images
\usepackage{booktabs} % Better tables
\usepackage{makecell}
\usepackage{enumitem} % Lists
\usepackage{dsfont}
\usepackage{geometry} % Page geometry
\usepackage[utf8]{inputenc} % Encoding
\usepackage[T2A]{fontenc} % Font
\usepackage[english, russian]{babel} % Multi-language support
\usepackage{titling} % Better titles
\usepackage{textcomp} % Old-style numbers? Check difference
\usepackage{mathtext} % Russian text in math expressions? Check difference
\usepackage{amsmath, amsfonts, amssymb, amsthm, mathtools} % Mathematics
\usepackage{bm} % Bold math symbols
\usepackage{icomma} % Better comma in numbers within math mode
\usepackage{xifthen} % Better if-expressions
\usepackage{transparent} % Transparent colors
\usepackage{caption}    % }
\usepackage{subcaption} % } Captioning figures
\usepackage[table,xcdraw]{xcolor} % Colors
\usepackage{textpos} % Absolute positioning
\usepackage{upgreek} % Cool greek letters

\usepackage{fancyvrb}
\usepackage{fvextra}
\usepackage{chngcntr}

%%% Page geometry
% \setlength\parindent{0pt} % No indentation in paragraphs
\setlist{noitemsep} % No spacing between list items

\usepackage{float}
\usepackage{multirow}

\geometry{
    paper=a4paper,
    top=2.5cm,
    bottom=3cm,
    left=2.5cm,
    right=2.5cm,
    headheight=0.75cm,
    footskip=1.5cm,
    headsep=0.75cm,
}

%%% Numeration
\newcommand{\RNumb}[1]{\uppercase\expandafter{\romannumeral #1\relax}}
\newcommand{\thesec}{\arabic{section}}
\renewcommand\thesection{\arabic{section}}
\renewcommand\thesubsection{\thesection.\arabic{subsection}}
\renewcommand\thesubsubsection{\RNumb{\arabic{subsubsection}}}
\renewcommand{\sectionmark}[1]{\markright{\thesection\ #1}}
\renewcommand{\bf}{\textbf}
\renewcommand{\it}{\textit}
\def\hash{\texttt{\#}}
\def\cpp{\C\texttt{++}}

\counterwithin{figure}{section}
\counterwithin{table}{section}
\renewenvironment{titlepage}{\thispagestyle{empty}} % Include titlepage into page numeration

%%% Headers and footers
\usepackage{setspace}
\usepackage{fancyhdr}
\usepackage{lastpage}


%%% Graphics
% Custom colors
\definecolor{myblue}{RGB}{72, 184, 178}
\definecolor{myblue1}{RGB}{0, 109, 167}
\definecolor{commentgreen}{RGB}{2,112,10}
\definecolor{mauve}{rgb}{0.58,0,0.82}
\definecolor{amethyst}{RGB}{153, 102, 203}

\usepackage{pgfplots} % Plots
\usepackage{tikz}
\usetikzlibrary{3d,perspective,decorations.text}
\usetikzlibrary{animations}
\usetikzlibrary{positioning}
\usetikzlibrary{matrix}
\usepackage{tikz-cd}
\usetikzlibrary{cd}
\usetikzlibrary{karnaugh}
\pgfplotsset{width=6cm,compat=newest}
\usepackage{color}

\usepackage[framemethod=TikZ]{mdframed}
\newcommand{\definebox}[2]{\newcounter{#1}\newenvironment{#1}[1][]{\stepcounter{#1}\mdfsetup{frametitle={\tikz[baseline=(current bounding box.east),outer sep=0pt]\node[anchor=east,rectangle,fill=white]{\strut \MakeUppercase#1~\csname the#1\endcsname\ifstrempty{##1}{}{:~##1}};}}\mdfsetup{innertopmargin=1pt,linecolor=#2,linewidth=3pt,topline=true,frametitleaboveskip=\dimexpr-\ht\strutbox\relax,}\begin{mdframed}[]\relax}{\end{mdframed}}}
\definebox{definition}{black!90}
\definebox{theorem}{myblue1!90}
\definebox{demonstration}{amethyst!90}

\newcounter{Theorem}
\def\themytheorem{\thesection.\arabic{Theorem}}
\usepackage[most]{tcolorbox}
\tcbuselibrary{theorems}
\newtcbtheorem{Theorem}{Theorem}
{colframe=myblue!90,coltitle=black,colback=white,fonttitle=\bfseries}{Th}

%%% Code listings
\usepackage{matlab-prettifier}

\lstset{
    extendedchars=\true,
}
\lstdefinestyle{cpp} {
    language=C++,
    frame=tb,
    tabsize=4,
    showstringspaces=false,
    numbers=left,
    captionpos=b,
    columns=flexible,
    upquote=true,
    commentstyle=\color{commentgreen},
    keywordstyle=\color{blue},
    stringstyle=\color{commentgreen},
    basicstyle=\small\ttfamily,
    emph={int,char,double,float,unsigned,void,bool,size\_t},
    emphstyle={\color{blue}},
    escapechar=\&,
    classoffset=1,
    otherkeywords={>,<,.,;,-,!,=,~},
    morekeywords={>,<,.,;,-,!,=,~},
    keywordstyle=\color{black},
    classoffset=0,
}
\lstdefinestyle{py} {
    language=Python,
    frame=tb,
    tabsize=4,
    showstringspaces=false,
    numbers=left,
    captionpos=b,
    columns=flexible,
    upquote=true,
    commentstyle=\color{commentgreen},
    keywordstyle=\color{blue},
    stringstyle=\color{commentgreen},
    basicstyle=\small\ttfamily,
    emph={and,as,assert,break,class,continue,def,del,elif,else,except,False,finally,for,from,global,if,import,in,%
    is,lambda,None,nonlocal,not,or,pass,raise,return,True,try,while,with,yield},
    emphstyle={\color{blue}},
    classoffset=1,
    otherkeywords={>,<,.,;,-,!,=,~},
    morekeywords={>,<,.,;,-,!,=,~},
    keywordstyle=\color{black},
    classoffset=0,
}
\lstdefinestyle{def} {
    frame=tb,
    tabsize=4,
    showstringspaces=false,
    numbers=left,
    captionpos=b,
    columns=flexible,
    upquote=true,
    commentstyle=\color{black},
    keywordstyle=\color{black},
    stringstyle=\color{black},
    basicstyle=\small\ttfamily,
    emph={int,char,double,float,unsigned,void,bool,size\_t},
    emphstyle={\color{black}},
    escapechar=\&,
    classoffset=1,
    otherkeywords={>,<,.,;,-,!,=,~},
    morekeywords={>,<,.,;,-,!,=,~},
    keywordstyle=\color{black},
    classoffset=0,
}

%%% Other
\usepackage[normalem]{ulem} % }
\useunder{\uline}{\ul}{}    % } Underline text
\usepackage[colorlinks,urlcolor=blue,filecolor=blue,citecolor=blue,linkcolor = blue,unicode=true]{hyperref}
\usepackage{titlesec}
\titlelabel{\thetitle.\quad}
\usepackage{secdot}
\sectiondot{subsection}
\usepackage{kvmap} % Karnaugh-maps for logic functions
\usepackage{}
\newcommand{\projectname}[3]{
    \begin{center}
        \Large
        \textbf{#1}\\[10pt]
        \textbf{#2}\\[10pt]
        \normalsize
        #3
        \rule{\linewidth}{0.4pt}
    \end{center}
}

\newcommand{\hfconfiguration}[3]{
    \pagestyle{fancy}
    \fancyhead[LE,RO]{}
    \fancyhead[LO,RE]{#1}
    \renewcommand{\footrulewidth}{0.4pt}
    \fancyfoot[C]{\thepage/\pageref*{LastPage}}
    \fancyfoot[LO,RE]{#2}
    \fancyfoot[LE,RO]{#3}
}

\newcommand{\filename}[2]{
    \pagebreak
    \titleformat{\section}
    [display]
    {\bfseries\Large}
    {}
    {0ex}
    {
        \vspace{-4.5ex}
        % \rule{\textwidth}{1pt}
        #1 \centering
    % \vspace{1ex}
    }
    [
        \normalfont\large
        #2
        \rule{\textwidth}{0.4pt}
        \normalsize
    ]
}

\newcommand{\project}[6]{
    \projectname{#1}{#2}{#3}
    \pagestyle{empty}
    \tableofcontents
    \newpage
    \hfconfiguration{#4}{#5}{#6}
}
%%% Final touch
\usepackage{subfiles}

\graphicspath{{polytech/etc/os/subfiles/images/}}

\begin{document}
    \begin{titlepage}
    \begin{center}
        \large Санкт-Петербургский политехнический университет Петра Великого\\
        \large Институт компьютерных наук и технологий \\
        \large Кафедра компьютерных систем и программных технологий\\[6cm]


        \huge Р Е Ф Е Р А Т\\[0.5cm]
        \large по дисциплине <<Основы операционных систем>>\\[0.1cm]
        \large\textbf{Файловая система FAT}\\[5cm]
    \end{center}


    \begin{flushright}
        \begin{minipage}{0.25\textwidth}
            \begin{flushleft}

                \large\textbf{Работу выполнил:}\\
                \large Калашников О.Ю.\\
                \large {Группа:} 35300901/10005\\

                \large \textbf{Преподаватель:}\\
                \large Малышев И.А.

            \end{flushleft}
        \end{minipage}
    \end{flushright}

    \vfill

    \begin{center}
        \large Санкт-Петербург\\
        \large \the\year
    \end{center}
\end{titlepage}

\vfill
\newpage
    
    \hfconfiguration{Файловая система FAT}{}{} 
    
    \tableofcontents
    
    \section{Вступление}

    Файловая система FAT изначально была разработана в 1970-х годах Марком МакДональдом и Биллом Гейтсом для использования на дискетах, и стала стандартом в MS-DOS и ранних версиях Windows. Несмотря на свой довольно большой возраст, FAT до сих пор используется на флэшках и некоторых других твердотельных накопителях. За всю свою историю она подверглась множеству изменений, однако сохранила свои ключевые идеи.

    \section{Сравнение с NTFS}

    Стандартом в Windows сейчас является современная файловая система NTFS, которая по многим характеристикам заметно обгоняет наиболее популярные версии FAT. Разница в характеристиках представлена в таблице~\ref{tab:ntfs}
    
    \begin{table}[H]
        \centering
        \begin{tabular}{|c|c|c|c|c|}
        \hline
                                                                              & NTFS                                                                      & FAT32                                                                    & FAT16                                                                    & FAT12                                                                    \\ \hline
        \begin{tabular}[c]{@{}c@{}}Максимальный\\ размер раздела\end{tabular} & 2ТБ                                                                       & 32ГБ                                                                     & 4ГБ                                                                      & 16Мб                                                                     \\ \hline
        \begin{tabular}[c]{@{}c@{}}Максимальный\\ размер файла\end{tabular}   & 16ТБ                                                                      & 4ГБ                                                                      & 2ГБ                                                                      & $<$16Мб                                                                  \\ \hline
        Размер блока                                                          & 4КБ                                                                       & $4 - 32$ КБ                                                              & $2 - 64$ КБ                                                              & $0.5 - 4$ КБ                                                             \\ \hline
        Отказоустойчивость                                                    & Да                                                                        & Нет                                                                      & Нет                                                                      & Нет                                                                      \\ \hline
        Сжатие                                                                & Да                                                                        & Нет                                                                      & Нет                                                                      & Нет                                                                      \\ \hline
        Совместимость                                                         & \begin{tabular}[c]{@{}c@{}}Windows\\ 10/8/7/XP/\\ Vista/2000\end{tabular} & \begin{tabular}[c]{@{}c@{}}Windows\\ ME/2000/\\ XP/7/8.1/10\end{tabular} & \begin{tabular}[c]{@{}c@{}}Windows\\ ME/2000/\\ XP/7/8.1/10\end{tabular} & \begin{tabular}[c]{@{}c@{}}Windows\\ ME/2000/\\ XP/7/8.1/10\end{tabular} \\ \hline
        \end{tabular}
        \caption{Сравнение FAT и NTFS}
        \label{tab:ntfs}
    \end{table}
    
    \section{Устройство файловой системы}

    Первоначально следует отметить, что система FAT обладает высокой надежностью и эффективностью в хранении данных благодаря своей основной идее -- представлению каждого файла в виде списка связанных блоков. Эта концепция позволяет оптимально распределить информацию в файловой системе, обеспечивая быстрый доступ к данным, а также возможность быстрого обнаружения и устранения ошибок при их возникновении.

    В связи с этим, важной ролью в системе FAT играет таблица размещения файлов, которая содержит информацию о последовательности соединенных блоков. Таким образом, каждый файл имеет свой уникальный адрес, который помогает системе быстро находить его и обеспечивать управление им.
    \begin{figure}[H]
        \centering
        \includegraphics[width=0.8\linewidth]{img2}
        \caption{Пример работы таблицы размещения}
    \end{figure}
    В целях расширения функциональности системы разработчики добавили в нее таблицу каталогов, которая содержит информацию о том, какой блок в файле является первым. Эта таблица выступает важной связующей позицией между таблицей размещения файлов и фактическим расположением данных на диске.
    \begin{figure}[H]
        \centering
        \includegraphics[width=0.8\linewidth]{img3}
        \caption{Пример работы таблицы каталогов}
    \end{figure}
    Однако при использовании системы FAT возникает некоторое количество вопросов, которые требуют дополнительных разъяснений. К примеру, как система определяет адрес root-директории? В данном случае стоит отметить, что адрес root-директории является зарезервированным и всегда известен системе. Это позволяет обеспечить корректную работу всей файловой системы, гарантированный доступ к данным и защиту от ошибок.
    
    Следовательно, использование системы FAT - это надежный и эффективный способ хранения информации, который обеспечивает быстрый доступ к данным, а также возможность быстрого обнаружения и устранения ошибок. Однако для полного понимания ее работы стоит изучить особенности ее устройства и принципов функционирования.

    \section{Практическая часть}
    
    Устройство файловой системы можно просмотреть на живом примере с помощью специализированных hex-редакторов.
    
    \begin{figure}[H]
        \centering
        \includegraphics[width=\linewidth]{img1}
    \end{figure}

    С помощью ASCII-дешифратора справа от таблицы байтов можно сразу увидеть знакомые слова. В самой первой строчке присутствует запись «MSDOS5.0», которая обозначает операционную систему, на которой проводилось форматирование файловой системы. В целях сохранения обратной совместимости, обычно в этом поле указываются более старые системы, как, например, в данном случае, несмотря на то, что последнее форматирование проводилось на Windows 10. На строке $50$ видно обозначение файловой системы — FAT32.
    
    В байтах $0B_{16}$ и $0C_{16}$ записан размер сектора системы в байтах. В данном случае — $200_{16}$ = $512_{10}$. В байтах $0E_{16}$ и $0F_{16}$ содержится информация о количестве секторов, выделенных под резервную область файловой системы — $0A68_{16}$ = $2664_{10}$. С помощью этой информации можно узнать местонахождение самой файловой таблица, которая будет находится в секторе под номером $A68_{16}$. Прежде чем перейти к самой таблице, стоит обратить внимание ещё на некоторые значения: в байтах $32$ и $33$ хранится информация о местонахождении резервной копии метаданных — $0006$.
    
    \begin{figure}[H]
        \centering
        \includegraphics[width=\linewidth]{img0}
    \end{figure}
    
    Перейдя в сектор с номером $6$ можно действительно увидеть копию той информации, которая только что рассматривалась.
    
    \section{Список использованных источников}

    \begin{itemize}
        \item \url{https://www.udacity.com/course/gt-refresher-advanced-os--ud098}
        \item \url{https://shorturl.at/hnFWZ}
        \item \url{https://www.youtube.com/watch?v=FQ_xeY0eCpA&ab_channel=AlekOS}
    \end{itemize}

\end{document}