\section{Метод Ньютона в неявных алгоритмах решения дифференциальных уравнений.}\label{sec:ch28}
Рассмотрим, как решается проблема неявного задания $\bf{x}_{n+1}$, например, в неявном методе ломаных Эйлера. Решение
этого уравнения относительно $\bf{x}_{n+1}$ может быть сведено к решению следующей системы
\begin{equation}
    \bf{F}(\bf{z}) = \bf{z} - \bf{x}_n - h\bf{f}(t_{n+1}, \bf{z}) = 0 \label{eq:hard6}
\end{equation}
методом Ньютона
\begin{equation*}
    \frac{\partial \bf{F}}{\partial \bf{z}}\left( \bf{z}^{(k)} \right)\left( \bf{z}^{(k+1)} - \bf{z}^{(k)} \right) = -\bf{F}\left( \bf{z}^{(k)} \right); \qquad
    \frac{\partial \bf{F}}{\partial \bf{z}} = \bf{E} - h \frac{\partial \bf{f}}{\partial \bf{z}}
\end{equation*}
где $\bf{z}^{(k)}$ -- $k$-е приближение к значению $\bf{x}_{n+1}$. Здесь весьма эффективен модифицированный метод Ньютона,
когда матрица $\displaystyle \frac{\partial \bf{F}}{\partial \bf{z}}$ вычисляется в точке $\bf{x}_0$, раскладывается в произведение
треугольных матриц программой \verb|DECOMP|, и на последующих итерациях используется только программа \verb|SOLVE|.
Матрица вновь вычисляется только тогда, когда метод Ньютона перестает сходиться за три итерации. Даже если матрица Якоби
$\displaystyle \frac{\partial \bf{f}}{\partial \bf{z}}$ исходной системы уравнений плохо обусловлена, обращение матрицы
$\displaystyle \frac{\partial \bf{F}}{\partial \bf{z}}$, как правило, не вызывает затруднений, так как она значительно
лучше обусловлена, чем $\displaystyle \frac{\partial \bf{f}}{\partial \bf{z}}$.

В итоге, применение метода Ньютона в неявных алгоритмах может быть описано по следующей схеме.
\begin{enumerate}
    \item В некоторой точке вычисляем матрицу Якоби по аналитическим формулам для ее элементов или с помощью формул
    численного дифференцирования, а затем производим ее разложение с помощью программы \verb|DECOMP|.

    \item По начальному условию $\displaystyle \bf{z}^{(0)}$, рассчитанному с помощью явного метода Эйлера,
    выполняем итерации метода Ньютона для получения $\displaystyle \bf{x}_{n+1}$.

    \item После одной-двух итераций по методу Ньютона при достижении сходимости переходим к шагу 2. Возвращение к
    шагу 1 проводится только в том случае, если метод Ньютона перестает сходиться за три итерации.
\end{enumerate}
Такая организация вычислений приводит к малому объему работы на одном шаге, а сам метод Ньютона с хорошим начальным
приближением сходится за одну-две итерации.

По аналогичной схеме для решения жестких систем используются и другие неявные методы. Методы с областью устойчивости,
пригодной для решения жестких систем, почти всегда являются неявными, хотя, разумеется, далеко не все неявные методы
такую область имеют.
