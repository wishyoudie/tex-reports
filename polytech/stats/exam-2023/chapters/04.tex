\documentclass[polytech/stats/exam-2023/stats-exam-2023.tex]{subfiles}
\graphicspath{{polytech/stats/exam-2023/images/}}
\begin{document}
\section{Зависимые события. Условные вероятности. Вывод формулы. Признак независимости случайных событий.}
    \begin{definition}[Условная вероятность]
        Пусть $A \subset \Omega, B \subset \Omega$ и сформулированы условия $S$.

        \textit{Условной вероятностью} называют вероятность осуществления одного из событий, при условии, что
        другое событие состоялось.
        \begin{equation*}
            P(B|A) = \frac{P(A \cap B)}{P(A)}
        \end{equation*}
    \end{definition}

    Из определений $P(B|A)$ и $P(A|B)$ получаем следующий вывод:
    \begin{equation*}
        P(A \cap B) = P(B|A) \cdot P(A) = P(A|B) \cdot P(B).
    \end{equation*}
    События $A$ и $B$ независимы, когда $P(B|A) = P(B)$ и $P(A|B) = P(A) \Rightarrow$
    \begin{theorem}[Признак независимости событий]
        События $A$ и $B$ независимы тогда и только тогда, когда $P(A \cap B) = P(A) \cdot P(B)$.

        \textit{Доказательство.}
        \vspace{0.5cm}
        В самом деле, при таком соотношении
        \begin{equation*}
            P(B|A) = \frac{P(A \cap B)}{P(A)} = \frac{P(A)P(B)}{P(A)} = P(B)
        \end{equation*}
    \end{theorem}
    Если $A$ и $B$ связаны взаимно-однозначно, то 
    \begin{equation*}
        P(A \cap B) = P(A) = P(B), \quad P(B|A) = 1, \quad P(A|B) = 1.
    \end{equation*}
    \newpage
\end{document}