\documentclass[../../calc-math-exam-2023.tex]{subfiles}
\begin{document}
    \section{Задача Коши решения обыкновенных дифференциальных уравнений. Явный и неявный методы ломаных Эйлера, метод трапеций.}\label{sec:ch24}
    Как известно, в практических приложениях решения дифференциального уравнения или системы уравнений описывают
    динамику разнообразных явлений и процессов (например, движение совокупности взаимодействующих материальных точек,
    химическую кинетику, процессы в электрических цепях и т.п.). Однако интегрируемых в явном виде дифференциальных
    уравнений крайне мало. Поэтому столь важны численные методы.

    Задача Коши (или задача с начальными условиями) из множества решений для системы
    \begin{equation}
        \frac{d\bf{x}(t)}{dt} = \bf{f}(t, \bf{x}(t)) \label{eq:koshi_1}
    \end{equation}
    где $t$ -- независимая переменная, $\bf{x}(t) = \left( x^{(1)}, \ldots, x^{(m)} \right)^T$ -- вектор искомых функций,
    удовлетворяющих уравнению, и $\bf{f}(t, \bf{x})$ -- вектор заданных, нужное число раз дифференцируемых функций,
    выделяет одно решение, проходящее через начальную точку $(t_0, \bf{x}_0)$. Аналогично ставится задача и для
    дифференциального уравнения $m$-го порядка, разрешенного относительно старшей производной, которое сводится к
    системе~\eqref{eq:koshi_1} из $m$ уравнений первого порядка.

    Если правые части $\bf{f}(t, \bf{x})$, а также элементы матрицы Якоби $\frac{\partial \bf{f}}{\partial \bf{x}}$
    непрерывны и ограничены в некоторй окрестности точки $t_0, \bf{x}_0$, то задача Коши имеет единственное решение.
    Первоначально, исключительно для простоты рассуждений, будем полагать, что~\eqref{eq:koshi_1} представляет собой
    одно уравнение. Вместе с тем, абсолютно все излагаемые в настоящем разделе методы сохраняют свой внешний вид и для
    случая, когда \bf{x} и \bf{f} являются векторами и~\eqref{eq:koshi_1} является системой уравнений.

    Общий подход к решению~\eqref{eq:koshi_1} заключается в приближенном сведении дифференциального уравнения к
    некоторому разностному уравнению, которое, в свою очередь, решается затем пошаговым методом. С этой целью
    выполним дискретизацию независимой переменной: $t_n = t_0 + nh$, где $h$ -- шаг интегрирования (шаг дискретности), а
    значения решения и его производной в этих точках кратко обозначим как $\bf{x}_n = \bf{x}(t_n)$ и $\bf{f}_n = \bf{f}(t, \bf{x}_n)$.
    Интегрируя~\eqref{eq:koshi_1} на промежутке $[t_n, t_{n+1}]$, получаем формулу
    \begin{equation}
        \bf{x}_{n+1} = \bf{x}_n + \int_{t_n}^{t_{n+1}} \bf{f}(\tau, \bf{x}(\tau))d\tau \label{eq:koshi_2}
    \end{equation}
    которую можно считать базовой для построения большей части известных разностных схем. Различные методы при этом
    отличаются способом вычисления интеграла в равенстве~\eqref{eq:koshi_2}.

    Использование квадратурных формул левых и правых прямоугольников, а также формулы трапеций, приводит соответственно
    к следующим численным методам:
    \begin{flalign}
        &\bf{x}_{n+1} = \bf{x}_n + h\bf{f}(t_n, \bf{x}_n) \label{eq:diff_left_triangles} \\
        &\bf{x}_{n+1} = \bf{x}_n + h\bf{f}(t_{n+1}, \bf{x}_{n+1}) \label{eq:diff_right_triangles} \\
        &\displaystyle \bf{x}_{n+1} = \bf{x}_n + \frac{h}{2}\left( \bf{f}(t_n, \bf{x}_n) + \bf{f}(t_{n+1}, \bf{x}_{n+1}) \right) \label{eq:diff_traps}
    \end{flalign}
    которые получили название \emph{явного метода ломаных Эйлера}, \emph{неявного метода ломаных Эйлера} и \emph{неявного
    метода трапеций}. Разностные уравнения~\eqref{eq:diff_right_triangles} и~\eqref{eq:diff_traps} неявно задают значения
    $\bf{x}_{n+1}$ и требуют решения нелинейных уравнений на каждом шаге интегрирования.

    \emph{Заметим, что} для остальных известных нам квадратурных формул требуется вычисление значения функции в точках
    между узлами, в то время как решение $\bf{x}(t)$ в этих точках неопределено. То есть, подобный способ <<превращения>>
    квадратурных формул в формулы для решения дифференциальных уравнений здесь не пройдет.
\end{document}