\section{Конечные разности и их свойства. Таблица конечных разностей.}\label{sec:ch01}

\subsection{Конечные разности и их свойства.}
\begin{definition}[Конечная разность]
    Пусть значения некоторой функции $f(x)$ известны лишь для дискретного множества значений независимой переменной
    $x \in \{x_0\dots x_m\}$. Выражение
    \begin{equation}
        \Delta_h f\left(x_k\right) = f(x_k + h) - f(x_k) = f\left(x_0 + (k + 1)h\right) - f\left(x_0 + kh \right)
    \end{equation}
    называют \emph{конечной разностью} (\emph{разностным оператором}) первого порядка.
\end{definition}
Поскольку величины $x_0 \text{ и } h$ постоянны для рассматриваемого множества, целесообразно, не умаляя общности,
перейти к новой переменной $k = \frac{x_k - x_0}{h}$, которая принимает целые значения $0 \dots m-1$. Тогда функция $f(x)$
становится функцией целочисленной переменной $f(k)$, и можно будет опустить индекс $h = const$.
\begin{equation*}
    \Delta f(k) = \Delta f_k = f(k + 1) - f(k) = f_{k+1} - f_k
\end{equation*}
Теперь обратимся к некоторым свойствам конечных разностей, отмечая тесную связь между ними и свойствами производных, что
является основой большинства конечно-разностных выражений.
\begin{enumerate}
    \item $\displaystyle \alpha = const \Rightarrow \Delta \alpha = 0$
    \begin{equation*}
        \alpha = const \Rightarrow \forall k \quad f(k+1) - f(k) = \alpha - \alpha = 0
    \end{equation*}
    \item $\displaystyle \Delta \left( \alpha f(k) \right) = \alpha \Delta f(k)$
    \item $\displaystyle \Delta \left( f(k) \pm g(k) \right) = \Delta f(k) \pm \Delta g(k)$
    \item $\displaystyle \Delta \left( f(k) \cdot g(k) \right) = \Delta f(k) \cdot g(k + 1) + f(k) \Delta g(k)$
    \begin{gather*}
        \Delta \left( f(k) \cdot g(k) \right) = f(k + 1)g(k + 1) - f(k)g(k) =\\
        f(k+1)g(k+1) - f(k)g(k) + f(k)g(k+1) - f(k)g(k+1) = \\
        \Delta f(k) \cdot g(k+1) + f(k)\Delta g(k)
    \end{gather*}
    \emph{Заметим, что} аналогичными преобразованиями можно было получить и другой вид, в котором функции идут в другом
    порядке:
    \[
        \Delta \left(g(k) \cdot f(k) \right) = \Delta g(k) \cdot f(k + 1) + g(k) \Delta f(k)
    \]
    \item Конечная разность от полинома степени $s$ равна полиному степени $s-1$.
    \begin{equation*}
        \Delta k^s = (k+1)^s - k^s = sk^{s-1} + \frac{s(s-1)}{2}k^{s-2} + \dots
    \end{equation*}
    \item Конечная разность высокого порядка.

    Подобно дифференциалам и производным высокого порядка, соответствующие конечные разности строятся на основе
    рекуррентных соотношений. Так конечная разность порядка $s+1$ строится следующим образом:
    \begin{equation*}
        \Delta^{s+1} f_k = \Delta \left(\Delta^s f_k \right) = \Delta^s f_{k+1} - \Delta^s f_k
    \end{equation*}
    По индукции можно доказать следующее утверждение:
\end{enumerate}
\begin{theorem}[О конечных разностях высокого порядка]
    \begin{equation}
        \Delta^s f_k = \sum_{i=0}^s (-1)^i C_s^i f_{k+s-i}
    \end{equation}
\end{theorem}

\subsection{Таблица конечных разностей.}
Аналогично тому, как в непрерывном случае строилась таблица производных, рассмотрим конечные разности для наиболее
популярных функций.
\begin{enumerate}
    \item $\displaystyle \Delta a^k = a^{k+1} - a^k = a^k(a - 1)$

    \emph{Заметим, что} число $2$ в условиях конечных разностей играет роль, схожую с экспонентой в непрерывном случае:
    $\left(e^x\right)' = e^x$.
    \item $\displaystyle \Delta \sin(k) = \sin(k+1) - \sin(k) = 2\sin \left(\frac{1}{2}\right) \cos \left(k + \frac{1}{2}\right)$
    \item $\displaystyle \Delta \cos(k) = \cos(k+1) - \cos(k) = -2\sin\left(\frac{1}{2}\right)\sin\left(k+\frac{1}{2}\right)$
    \item $\displaystyle \Delta \log(k) = \log(k+1) - \log(k) = \log\left(1 + \frac{1}{k}\right)$
\end{enumerate}
