\documentclass[../../calc-math-exam-2023.tex]{subfiles}
\begin{document}
    \begin{enumerate}
        \item Конечные разности и их свойства. Таблица конечных разностей.
        \item Суммирвоание функций. Формула Абеля суммирования по частям.
        \item Разностное уравнение, его порядок. Линейные разностные уравнения первого порядка и порядка выше первого.
        \item Разделенные разности и их связь с конечными разностями.
        \item Аппроксимация функций. Задача интерполирования.
        \item Интерполяционный полином Лагранжа. Остаточный член полинома Лагранжа.
        \item Выбор узлов интерполирования. Интерполяционный полином Ньютона для равно- и неравноотстоящих узлов.
        \item Сплайн-интерполяция. Подпрограммы \textbf{SPLINE} и \textbf{SEVAL}. Интерполирование по Эрмиту. Обратная задача интерполирования.
        \item Квадратурные формулы левых, правых и средних прямоугольников, трапеций, Симпсона. Малые и составные формулы, их остаточные члены.
        \item Общий подход к построению квадратурных формул. Квадратурные формулы Ньютона-Котеса, Чебышева, Гаусса.
        \item Адаптивные квадратурные формулы. Подпрограмма \textbf{QUANC8}.
        \item Задача численного дифференцирования. Влияние вычислительной погрешности.
        \item Среднеквадратичная аппроксимация (дискретный случай). Понятие веса.
        \item Среднеквадратичная аппроксимация (непрерывный случай). Понятие ортогональности.
        \item Ортогонализация по Шмидту. Прмиеры ортогональных полиномов.
        \item Обратная матрица, собственные числа и векторы. Задачи на матрицы. Норма матрицы, сходимость матричного степенного ряда, функции от матрицы.
        \item 7 теорем о матричных функциях.
        \item Решение систем линейных дифференциальных и разностных уравнений с постоянной матрицей.
        \item Устойчивость решений дифференциальных и разностных уравнений.
        \item Метод Гаусса и явление плохой обусловленности. \textbf{LU}-разложение матрицы. Подпрограммы \textbf{DECOMP} и \textbf{SOLVE}.
        \item Метод последовательных приближений для решения линейных систем.
        \item Методы бисекции, секущих, обратной параболической интерполяции для решения нелинейных уравнений. Подпрограмма \textbf{ZEROIN}.
        \item Методы последовательных приближений и Ньютона для решения нелинейных уравнений и систем.
        \item Задача Коши решения обыкновенных дифференциальных уравнений. Явный и неявный методы ломаных Эйлера, метод трапеций.
        \item Методы Адамса. Локальная и глобальная погрешности, степень метода.
        \item Методы Рунге-Кутты. Подпрограмма \textbf{RKF45}.
        \item Глобальная погрешность. Устойчивость метода. Ограничение на шаг. Явление жесткости и методы решения жестких систем.
        \item Метод Ньютона в неявных алгоритмах решения дифференциальных уравнений.
        \item Сведение дифференциального уравнения высокого порядка к системе уравнений первого порядка. Метод стрельбы для решения краевых задач.
    \end{enumerate}
\end{document}