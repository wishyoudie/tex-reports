\section{Суммирование функций. Формула Абеля суммирования по частям.}\label{sec:ch02}

\subsection{Суммирование функций.}
Обратимся к уравнению
\begin{equation}
    \Delta F(k) = \varphi(k)\label{eq:main_theme_equation}
\end{equation}
До сих пор мы занимались прямой задачей: по заданной функции $F(k)$ необходимо определить функцию $\varphi(k)$. Теперь
обратимся к обратной задаче: по заданной функции $\varphi(k)$ необходимо восстановить функцию $F(k)$. Ситуация подобна
нахождению функции по ее производной в непрерывных терминах. В этом случае появляется возможность ее решения при помощи
интеграла
\begin{equation*}
    \int_a^b h(x)dx = \int_a^b f'(x)dx = f(b) - f(a)
\end{equation*}

Аналогично, решение обратной задачи~\eqref{eq:main_theme_equation} позволяет, в свою очередь, успешно
решать задачу суммирования функции $\varphi(k)$. Запишем уравнение~\eqref{eq:main_theme_equation} последовательно для
$k = m, m+1, \dots, N - 1$ и результаты просуммируем.
\begin{flalign*}
    &F(m+1) - F(m) = \varphi(m)\\
    &F(m+2) - F (m + 1) = \varphi(m + 1)\\
    &F(m+3) - F(m+2) = \varphi(m+2)\\
    & \dots\\
    &F(N) - F(N-1) = \varphi(N-1)\\
    &F(N) - F(m) = \sum_{k=m}^{N-1} \varphi(k)\\
\end{flalign*}

Иными словами

\begin{equation}
    \sum_{k=m}^{N-1} \varphi(k) = \sum_{k=m}^{N-1} \Delta F(k) = F(N) - F(m)\label{eq:newton_leibniz}
\end{equation}

Выражение~\eqref{eq:newton_leibniz} является дискретным аналогом формулы Ньютона-Лейбница. В дополнение следует
заметить, что она выводилась в предположении, что $N > m$.

\vspace{10pt}

Рассмотрим некоторые примеры суммирования функций. Результаты отдаленно напоминают таблицу интегралов для непрерывных
функций, а оператор суммы сопоставляется определенному интегралу:
\begin{equation*}
    \sum_{k=0}^{N-1} \leftrightarrow \int_0^N
\end{equation*}

\begin{enumerate}
    \item $\displaystyle \text{Найти } \sum_{k=0}^{N-1} a^k$

    Функция $F(k)$, удовлетворяющая условию $\displaystyle \Delta F(k) = a^k$ легко находится из таблицы конечных разностей:
    $\displaystyle F(k) = \frac{a^k}{a-1}$. Тогда
    \[
        \sum_{k=0}^{N-1} a^k = \frac{\left(a^N - 1\right)}{a - 1} - \frac{a^0 - 1}{a - 1} = \frac{1 - a^N}{1 - a}
    \]
    \item $\displaystyle \text{Найти } \sum_{k=0}^{N-1} \cos\left(k + \frac{1}{2}\right)$

    Аналогичным образом найдем функцию из таблицы конечных разностей.

    \[
        \sum_{k=0}^{N-1} \cos\left(k + \frac{1}{2}\right) = \sum_{k=0}^{N-1} \frac{\sin(k)}{2\sin\left(\frac{1}{2}\right)} =
        \frac{\sin(N)}{2\sin\left(\frac{1}{2}\right)}
    \]
    \item $\displaystyle \text{Найти } \sum_{k=0}^{N-1} k^2$

    Воспользуемся свойством №5 конечных разностей: полином степени $2$ является конечной разностью полинома степени $3$.
    Рассмотрим
    \begin{flalign*}
        &\Delta k^3 = (k+1)^3 - k^3 = 3k^2 + 3k + 1\\
        &\Delta k^2 = (k+1)^2 - k^2 = 2k + 1\\
        &\Delta k = (k+1) - k = 1
    \end{flalign*}

    Используя также свойства №1 и 2, получаем
    \begin{equation*}
        \Delta \left(\frac{k^3}{3} - \frac{k^2}{2} + \frac{k}{6}\right) =
        \frac{1}{3}\left(3k^2 + 3k + 1\right) - \frac{1}{2}\left(2k + 1\right) + \frac{1}{6} \cdot 1 =
        k^2 + k + \frac{1}{3} - k - \frac{1}{2} + \frac{1}{6} = k^2
    \end{equation*}

    \vspace{15pt}

    Получаем
    \begin{flalign*}
        &F(k) = \frac{1}{3}k^3 - \frac{1}{2}k^2 + \frac{1}{6}k\\
        &\sum_{k=0}^N k^2 = \frac{N^3}{3} - \frac{N^2}{2} + \frac{N}{6}
    \end{flalign*}
\end{enumerate}

\subsection{Суммирование по частям.}
Суммирование по частям вводится как прием, аналогичный интегрированию по частям в непрерывном случае. Для вывода формулы
запишем уравнение интегрирования по частям в несколько другом виде. Введем три функции: $\displaystyle u(t), v(t), U(t)$
, где $\displaystyle U(t) = \int_{0}^{t}u(\tau)d\tau$. Рассмотрим производную выражения $U(t)v(t)$:

$\displaystyle \frac{d}{dt}\left(U(t)v(t)\right) = \frac{dU(t)}{dt}v(t) + U(t)\frac{dv(t)}{dt} = u(t)v(t) + U(t)\frac{dv(t)}{dt}$
\vspace{5pt}

Перенесем второе слагаемое из правой части равенства в левую и проинтегрируем
\begin{equation*}
    \int_a^b u(t)v(t)dt = \int_a^b \frac{d}{dt}\left(U(t)v(t)\right)dt - \int_a^b U(t) \frac{dv(t)}{dt}dt = U(t)v(t) \biggr\rvert_{t=a}^{t=b} - \int_a^b U(t)\frac{dv(t)}{dt}dt
\end{equation*}

Теперь обратимся к суммированию. Аналогично введем три функции: $\displaystyle u(k), v(k), U(k)$, где
$\displaystyle U(k) = \sum_{i=0}^k u(i)$. $\displaystyle \Delta U(k) = U(k+1) - U(k) = u(k+1)$.
Воспользуемся ранее полученной формулой конечной разности для произведения:

$\displaystyle \Delta \left(U(k)v(k)\right) = v(k+1) \Delta U(k) + U(k)\Delta v(k) = v(k+1)u(k+1) + U(k)\Delta v(k)$

Перенесем второе слагаемое из правой части равенства в левую и просуммируем обе его части:

\begin{theorem}[Формула Абеля суммирования по частям]
    \begin{equation}
        \sum_{k=p}^N u(k+1)v(k+1) = U(k)v(k) \bigg\rvert_{k=p}^{k=N+1} - \sum_{k=p}^N U(k) \Delta v(k)\label{eq:abel}
    \end{equation}
\end{theorem}

\emph{Пример:} требуется найти $\displaystyle \sum_{k=0}^N ka^k$.

Возьмем функции $u(k) = a^{k-1}, v(k) = k-1$. Тогда
\begin{flalign*}
    &U(k) = \sum_{i=0}^k a^{k-1} = \frac{1}{a}\frac{a^{k+1} - 1}{a - 1}\\
    &\Delta v(k) = \Delta k = 1\\
    &\\
    &\sum_{k=0}^N ka^k = \frac{1}{a}\frac{a^{N+2} - 1}{a - 1}\cdot\left( ((N + 1) - 1) - (0 - 1)\right) - \sum_{k=0}^N \frac{1}{a}\frac{a^{k+1} - 1}{a - 1} = \\
    &= \frac{(N+1)(a^{N+2} - 1)}{a(a - 1)} - \frac{1}{a(a-1)} \left( a\sum_{k=0}^{N} a^k - (N+1) \right) = \\
    &=\frac{(N+1)(a^{N+2}-1)}{a(a-1)} - \frac{1}{a(a-1)} \left( a \frac{a^{N+1} - 1}{a - 1} - (N+1) \right) =\\
    &=\frac{(a-1)(N+1)(a^{N+2} - 1) - a^{N+3} + a^2 + (a-1)(N+1)}{a(a-1)^2}=\\
    &=\frac{(a-1)(N+1)a^{N+1} - a^{N+2} + a}{(a-1)^2} = \frac{Na^{N+2} - (N+1)a^{N+1} + a}{(a-1)^2} = \\
    &=\frac{a\left( Na^{N+1} -(N+1)a^N + 1 \right)}{(a-1)^2}
\end{flalign*}
