\section{Решение систем линейных дифференциальных и разностных уравнений с постоянной матрицей.}\label{sec:ch18}

\subsection{Дифференциальные уравнения.}
Рассмотрим систему обыкновенных дифференциальных уравнений первого порядка, разрешенную относительно производных
\begin{equation*}
    \frac{dx^{(i)}(t)}{dt} = f^{(i)}\left( t, x^{(1)}(t), x^{(2)}(t), \dots, x^{(m)}(t) \right), \qquad i=1,2,\ldots,m
\end{equation*}
где $t$ -- независимая переменная, $\displaystyle x^{(i)}(t)$ -- искомые функции, $\displaystyle f^{(i)}$ -- функции,
определенные на некотором множестве $(m+1)$-мерного евклидова пространства переменных $t, x^{(i)}(t)$. Номер компоненты
вектора здесь везде будем писать, как верхний индекс в скобках. Перейдя к векторно-матричным обозначениям
\begin{equation*}
    \bf{x}(t) =
    \begin{pmatrix}
        x^{(1)}(t) \\
        x^{(2)}(t) \\
        \dots      \\
        x^{(m)}(t)
    \end{pmatrix}
    , \qquad
    \bf{f}(t, \bf{x}) =
    \begin{pmatrix}
        f^{(1)}(t, x) \\
        f^{(2)}(t, x) \\
        \dots         \\
        f^{(m)}(t, x) \\
    \end{pmatrix}
\end{equation*}
Исходную систему перепишем в виде
\begin{equation}
    \frac{d\bf{x}(t)}{dt} = \bf{f}(t, \bf{x}) \label{eq:matrix_differential_equation_definition}
\end{equation}
При этом требуется найти решение $\bf{x}(t)$, удовлетворяющее начальным условиям $\displaystyle \bf{x}(t_0) = \bf{x}_0$.
Такая задача называется начальной задачей или \emph{задачей Коши}.

Важным классом дифференциальных систем являются линейные системы с постоянной матрицей или постоянными коэффициентами
\begin{equation}
    \frac{d\bf{x}(t)}{dt} = \bf{A}\bf{x}(t) + \bf{g}(t), \quad \bf{x}(t_0) = \bf{x}_0
\end{equation}
Сначала обратимся к однородной системе
\begin{equation*}
    \frac{d\bf{x}(t)}{dt} = \bf{A}\bf{x}(t)
\end{equation*}
Ее решением является функция $\displaystyle \bf{x}(t) = e^{\bf{A}t} \bf{c}$, где \bf{c} -- вектор произвольных постоянных.
Убедиться в этом можно непосредственной подстановкой решения в уравнение.
\vspace{10pt}

Неоднородная система решается методом Лагранжа вариации произвольных постоянных. При этом полагаем, что элементы вектора
\bf{c} являются функциями независимой переменной $\bf{c} = \bf{c}(t)$. Подставляем искомый вид решения в уравнение
\begin{equation*}
    \bf{A}e^{\bf{A}t}\bf{c}(t) + e^{\bf{A}t}\frac{d\bf{c}(t)}{dt} = \bf{A} e^{\bf{A}t}\bf{c}(t) + \bf{g}(t)
\end{equation*}
Отсюда $\displaystyle e^{\bf{A}t}\frac{d\bf{c}(t)}{dt} = \bf{g}(t)$, и после умножения обеих частей равенства на
$\displaystyle e^{-\bf{A}t}$ получаем:
\begin{equation*}
    \frac{d\bf{c}(t)}{dt} = e^{-\bf{A}t}\bf{g}(t)
\end{equation*}
Интегрируем это уравнение от $t_0$ до $t$
\begin{equation*}
    \bf{c}(t) - \bf{c}(t_0) = \int_{t_0}^{t} e^{-\bf{A}\tau}\bf{g}(\tau)d\tau
\end{equation*}
и, подставив $\bf{c}(t)$ в искомый вид решения, определяем общее решение линейной неоднородной дифференциальной системы
\begin{equation*}
    \bf{x}(t) = e^{\bf{A}t}\bf{c}(t_0) + e^{\bf{A}t}\int_{t_0}^{t}e^{-\bf{A}\tau}\bf{g}(\tau)d\tau
\end{equation*}
Учитывая начальные условия, находим вектор $\displaystyle c(t_0)$: $\displaystyle \bf{x}(t_0) = \bf{x}_0 = e^{\bf{A}t_0}\bf{c}(t_0)$
или $\displaystyle \bf{c}(t_0) = e^{-\bf{A}t_0}\bf{x}_0$ и окончательно получаем
\begin{equation*}
    \bf{x}(t) = e^{\bf{A}(t - t_0)}\bf{x}_0 + \int_{t_0}^{t} e^{\bf{A}(t - \tau)}\bf{g}(\tau)d\tau
\end{equation*}
Без нарушения общности можно считать, что начальным значением независимой переменной является $\displaystyle t_0 = 0$.
Тогда, используя теорему о свертке,
\begin{equation}
    \bf{x}(t) = e^{\bf{A}t}\bf{x}_0 + \int_{0}^{t} e^{\bf{A}(t - \tau)}\bf{g}(\tau)d\tau = e^{\bf{A}t}\bf{x}_0 + \int_{0}^{t} e^{\bf{A}\tau}\bf{g}(t - \tau)d\tau \label{eq:diffeqsol}
\end{equation}
Считая вектор $\bf{g}(t)$ постоянным, упростим полученное равенство:
\begin{equation*}
    \bf{x}(t) = e^{\bf{A}t}\bf{x}_0 + \int_{0}^{t} e^{\bf{A}(t - \tau)}d\tau \cdot \bf{g} = e^{\bf{A}t}\bf{x}_0 + \left( e^{\bf{A}t} - \bf{E} \right)\bf{A}^{-1}\bf{g}
\end{equation*}

\subsection{Разностные уравнения.}
Рассмотрим линейную систему разностных уравнений с постоянной матрицей, где $k$ -- независимая целочисленная переменная
\begin{equation}
    \bf{y}(k+1) = \bf{B}\bf{y}(k) + \bf{g}(k) \label{eq:matrix_sub_equation_definition}
\end{equation}
Будем ее решать так называемым пошаговым методом, последовательно назначая в~\eqref{eq:matrix_sub_equation_definition}
значения $k$ равным $0,1,2,\dots$ и обозначая $\bf{y}_k = \bf{y}(k)$
\begin{flalign*}
    &\bf{y}_1 = \bf{B}\bf{y}_0 + \bf{g}_0\\
    &\bf{y}_2 = \bf{B}\bf{y}_1 + \bf{g}_1 = \bf{B}^2 \bf{y}_0 + \bf{B} \bf{g}_0 + \bf{g}_1\\
    &\bf{y}_3 = \bf{B}\bf{y}_2 + \bf{g}_2 = \bf{B}^3 \bf{y}_0 + \bf{B}^2 \bf{g}_0 + \bf{B} \bf{g}_1 + \bf{g}_2\\
    &\dots
\end{flalign*}
\begin{equation}
    \bf{y}_k = \bf{B}^k \bf{y}_0 + \sum_{i=0}^{k-1} \bf{B}^{k-i-1} \bf{g}_i
\end{equation}
В частном случае, когда $\bf{g}(k) = const = \bf{g}$, этот вектор можно вынести за знак суммы
\begin{equation}
    \bf{y}_k = \bf{B}^k \bf{y}_0 + \left( \sum_{i=0}^{k-1} \bf{B}^i \right)\bf{g} = \bf{B}^k \bf{y}_0 + \left( \bf{B}^k - \bf{E} \right)\left( \bf{B} - \bf{E} \right)^{-1} \bf{g}
\end{equation}
