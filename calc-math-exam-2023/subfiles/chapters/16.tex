\documentclass[../../calc-math-exam-2023.tex]{subfiles}
\begin{document}
    \section{Обратная матрица, собственные числа и векторы. Задачи на матрицы. Норма матрицы, сходимость матричного степенного ряда, функции от матрицы.}\label{sec:ch16}

    \subsection{Обратная матрица.}
    \begin{definition}[Матрица]
        Матрицей называется совокупность $n \times m$ скаляров $a_{ij}$, образующих
        прямоугольную таблицу из $n$ строк и $m$ столбцов.
        \begin{equation*}
            \bf{A} =
            \begin{pmatrix}
                a_{11} & a_{12} & \dots & a_{1m} \\
                a_{21} & a_{22} & \dots & a_{2m} \\
                \dots  & \dots  & \dots & \dots  \\
                a_{n1} & a_{n2} & \dots & a_{nm}
            \end{pmatrix}
        \end{equation*}
        В литературе и, в частности, в этом конспекте матрицы обозначаются \bf{жирным} шрифтом.
    \end{definition}
    \begin{definition}[Обратная матрица]
        Для квадратной матрицы с ненулевым определителем $ \displaystyle \left( \det(\bf{A}) \neq 0 \right)$
        вводится понятие \emph{обратной матрицы} $\bf{X} = \bf{A}^{-1}$, которая существует, является
        единственной и удовлетворяет условиям $ \displaystyle \bf{A} \cdot \bf{X} = \bf{E},\quad \bf{X} \cdot \bf{A} = \bf{E}$
    \end{definition}

    \subsection{Собственные числа и векторы.}
    Поставим задачу: для заданной матрицы \bf{A} найти такие векторы \bf{u},
    которые сохраняют свое направление после линейного преобразования.
    \begin{equation}
        \bf{y} = \bf{A} \bf{u} = \uplambda \bf{u} \Leftrightarrow \left( \bf{A} - \uplambda \bf{E} \right) \bf{u} = 0 \label{eq:lvdef}
    \end{equation}
    \begin{definition}
        Числа $\uplambda$ и векторы \bf{u}, удовлетворяющие уравнению~\eqref{eq:lvdef} получили название
        \emph{собственные значения} и \emph{собственные векторы}.
    \end{definition}

    Если определитель $\det\left( \bf{A} - \uplambda\bf{E} \right)$ не равен нулю, то система имеет
    единственное решение $\bf{u} = 0$. Для того, чтобы существовало решение, отличное от нулевого
    потребуем равенство нулю определителя. Запишем это в покомпонентном виде:
    \begin{flalign*}
        \det \left(
        \begin{pmatrix}
            a_{11} & a_{12} & \dots & a_{1m} \\
            a_{21} & a_{22} & \dots & a_{2m} \\
            \dots  & \dots  & \dots & \dots  \\
            a_{n1} & a_{n2} & \dots & a_{nm}
        \end{pmatrix}
        -
        \begin{pmatrix}
            \uplambda & 0         & \dots & 0         \\
            0         & \uplambda & \dots & 0         \\
            \dots     & \dots     & \dots & \dots     \\
            0         & 0         & \dots & \uplambda
        \end{pmatrix}
        \right) = \\
        =\det
        \begin{pmatrix}
            a_{11} - \uplambda & a_{12}             & \dots & a_{1m}             \\
            a_{21}             & a_{22} - \uplambda & \dots & a_{2m}             \\
            \dots              & \dots              & \dots & \dots              \\
            a_{n1}             & a_{n2}             & \dots & a_{nm} - \uplambda
        \end{pmatrix}
        = 0
    \end{flalign*}

    Вычисление этого определителя дает:
    \begin{equation*}
        \det \left( \bf{A} - \uplambda \bf{E} \right) = (-1)^n \uplambda^n + b_1 \uplambda^{n-1} + \dots + b_{n-1}\uplambda + b_n = 0
    \end{equation*}
    Полином, стоящий в левой части уравнения, называется \emph{характеристическим}, как и само уравнение. Оно
    имеет ровно $n$ корней (с учетом крастности). Это и есть собственные значения матрицы $\displaystyle \uplambda_1, \uplambda_2, \dots, \uplambda_n$.
    Они составляют спектр матрицы \bf{A}, а величина $\rho \left( \bf{A} \right) = \max \left| \uplambda_i \right|$ называется
    \emph{спектральным радиусом}. Для каждого $\uplambda_i$ можно найти решение $\bf{u}_i$, т.е. собственный вектор.

    Поскольку собственные векторы находятся из решения однородной системы, то и неизвестными они оказываются с точностью
    до постоянного (ненулевого) множителя, т.е. собственные векторы однозначно определены по направлению, но их
    длины (нормы) остаются произвольными. Часто бывает удобно приводить их к единичной длине. Без доказательства
    оставим следующую теорему:
    \begin{theorem}
        Если все собственные значения матрицы различны, то отвечающие им собственные векторы -- линейно-независимы.
    \end{theorem}

    \subsection{Задачи на матрицы.}
    Не умаляя общности, будем считать исходные матрицы квадратными. Те же задачи можно доказать
    и для матриц в общем виде, достаточно лишь поменять соответствующие индексы.
    \begin{theorem}
        \begin{equation*}
            \left( \bf{A} \cdot \bf{B} \right)^T = \bf{B}^T \cdot \bf{A}^T
        \end{equation*}
        \underline{Доказательство}.

        Рассмотрим отдельно левую и правую части равенства. Правая:
        \begin{gather*}
            \bf{B}^T \cdot \bf{A}^T =
            \begin{pmatrix}
                b_{11} & \dots & b_{n1} \\
                \dots  & \dots & \dots  \\
                b_{1n} & \dots & b_{nn}
            \end{pmatrix}
            \cdot
            \begin{pmatrix}
                a_{11} & \dots & a_{n1} \\
                \dots  & \dots & \dots  \\
                a_{1n} & \dots & a_{nn}
            \end{pmatrix} =\\
            =\begin{pmatrix}
                 b_{11} \cdot a_{11} + \dots + b_{n1} \cdot a_{1n} & \dots & b_{11} \cdot a_{n1} + \dots + b_{n1} \cdot a_{nn} \\
                 \dots                                             & \dots & \dots                                             \\
                 b_{1n} \cdot a_{11} + \dots + b_{nn} \cdot a_{1n} & \dots & b_{1n} \cdot a_{n1} + \dots + b_{nn} \cdot a_{nn}
            \end{pmatrix} = \\
            = \begin{pmatrix}
                  a_{11} \cdot b_{11} + \dots + a_{1n} \cdot b_{n1} & \dots & a_{n1} \cdot b_{11} + \dots + a_{nn} \cdot b_{n1} \\
                  \dots                                             & \dots & \dots                                             \\
                  a_{11} \cdot b_{1n} + \dots + a_{1n} \cdot b_{nn} & \dots & a_{n1} \cdot b_{1n} + \dots + a_{nn} \cdot b_{nn}
            \end{pmatrix}
        \end{gather*}
        Левая:
        \begin{gather*}
            \left( \bf{A} \cdot \bf{b} \right)^T =
            \left(
            \begin{pmatrix}
                a_{11} & \dots & a_{1n} \\
                \dots  & \dots & \dots  \\
                a_{n1} & \dots & a_{nn}
            \end{pmatrix}
            \cdot
            \begin{pmatrix}
                b_{11} & \dots & b_{1n} \\
                \dots  & \dots & \dots  \\
                b_{n1} & \dots & b_{nn}
            \end{pmatrix}
            \right)^T
            = \\
            = \begin{pmatrix}
                  a_{11} \cdot b_{11} + \dots + a_{1n} \cdot b_{n1} & \dots & a_{11} \cdot b_{1n} + \dots + a_{1n} \cdot b_{nn} \\
                  \dots                                             & \dots & \dots                                             \\
                  a_{n1} \cdot b_{11} + \dots + a_{nn} \cdot b_{n1} & \dots & a_{n1} \cdot b_{1n} + \dots + a_{nn} \cdot b_{nn}
            \end{pmatrix}^T = \\
            = \begin{pmatrix}
                  a_{11} \cdot b_{11} + \dots + a_{1n} \cdot b_{n1} & \dots & a_{n1} \cdot b_{11} + \dots + a_{nn} \cdot b_{n1} \\
                  \dots                                             & \dots & \dots                                             \\
                  a_{11} \cdot b_{1n} + \dots + a_{1n} \cdot b_{nn} & \dots & a_{n1} \cdot b_{1n} + \dots + a_{nn} \cdot b_{nn}
            \end{pmatrix}
        \end{gather*}
        В результате левая и правая части совпадают. Теорема доказана.
    \end{theorem}

    \begin{theorem}
        \begin{equation*}
            \left( \bf{A} \cdot \bf{B} \right)^{-1} = \bf{B}^{-1} \cdot \bf{A}^{-1}
        \end{equation*}
        \underline{Доказательство}.

        Умножим обе части равенства на $\left( \bf{A} \cdot \bf{B} \right)$. Слева, по определению,
        получим единичную матрицу. Для выражения справа получаем
        \begin{equation*}
            \left( \bf{A} \cdot \bf{B} \right) \cdot \bf{B}^{-1} \cdot \bf{A}^{-1} = \bf{A} \cdot \left( \bf{B} \cdot \bf{B}^{-1} \right) \cdot \bf{A}^{-1} =
            \bf{A} \cdot \bf{E} \cdot \bf{A}^{-1} = \bf{A} \cdot \bf{A}^{-1} = \bf{E}
        \end{equation*}
    \end{theorem}

    \begin{theorem}
        \begin{equation*}
            \left( \bf{A}^T \right)^{-1} = \left( \bf{A}^{-1} \right)^T
        \end{equation*}
        \underline{Доказательство}.

        Умножим обе части равенства на $\displaystyle \bf{A}^T$. Слева, по определению,
        получим единичную матрицу. Для выражения справа, используя первую задачу, получаем
        \begin{equation*}
            \left( \bf{A}^{-1} \right)^T \cdot \bf{A}^T = \left( \bf{A} \cdot \bf{A}^{-1} \right)^T = \bf{E}^T = \bf{E}
        \end{equation*}
    \end{theorem}

    \begin{theorem}
        \begin{equation*}
            \det \left( \bf{A}^{-1} \right) = \frac{1}{\det\left( \bf{A} \right)}
        \end{equation*}

        \underline{Доказательство}.

        \bf{TODO}
    \end{theorem}

    \begin{theorem}
        При умножении матрицы \bf{A} на диагональную матрицу \bf{D} \emph{слева} $\bf{B} = \bf{D} \cdot \bf{A}$
        все \emph{строки} матрицы \bf{A} умножаются на соответствующие диагональные элементы.

        При умножении матрицы \bf{A} на диагональную матрицу \bf{D} \emph{справа} $\bf{B} = \bf{A} \cdot \bf{D}$
        все \emph{столбцы} матрицы \bf{A} умножаются на соответствующие диагональные элементы.

        \underline{Доказательство}.

        Очевидно и следует из определения умножения матриц.
    \end{theorem}

    \begin{theorem}
        Собственные значения диагональной матрицы равны ее диагональным элементам.

        \underline{Доказательство}.

        Пускай дана диагональная матрица \bf{D}. Рассмотрим определитель
        \begin{equation*}
            \det
            \begin{pmatrix}
                d_1 - \uplambda_1 & 0                 & \dots & 0                 \\
                0                 & d_2 - \uplambda_2 & \dots & 0                 \\
                \dots             & \dots             & \dots & \dots             \\
                0                 & 0                 & \dots & d_n - \uplambda_n
            \end{pmatrix}
            = 0
        \end{equation*}
        Другими словами
        \begin{equation*}
            \left( d_1 - \uplambda_1 \right) \cdot \left( d_2 - \uplambda_2 \right) \dots \left( d_n - \uplambda_n \right) = 0
        \end{equation*}
        Поочередно получаем $\displaystyle \uplambda_i = d_i$. Теорема доказана.
    \end{theorem}

    \begin{theorem}
        Собственные значения треугольной матрицы равны ее диагональным элементам.

        \underline{Доказательство}.

        Разложив определитель треугольной матрицы по первому столбцу, убедимся, что он равен произведению ее диагональных элементов.
        Применим предыдущую задачу к диагональной матрице, элементами которой являются элементы главной диагонали данной.
        Теорема доказана.
    \end{theorem}

    \begin{theorem}
        При умножении треугольных матриц одного вида получается матрица того же вида.

        \underline{Доказательство}.

        \bf{TODO}
    \end{theorem}

    \begin{theorem}
        Для треугольной матрицы ее обратная матрица имеет тот же вид.

        \underline{Доказательство}.

        \bf{TODO}
    \end{theorem}

    \begin{theorem}
        Для произвольной квадратной матрицы сумма ее собственных значений равна сумме элементов главной диагонали,
        а произведение ее собственных значений равно ее определителю.
        \begin{flalign*}
            &\sum_{k=1}^{N} \uplambda_k = \sum_{k=1}^{N} a_{kk} \\
            &\prod_{k=1}^{N} \uplambda_k = \det \left( \bf{A} \right)
        \end{flalign*}
    \end{theorem}

    \subsection{Нормы матриц.}
    \begin{definition}[Норма матрицы]
        \emph{Нормой} матрицы $\left\| \bf{A} \right\|$ называется число, удовлетворяющее следующим аксиомам:
        \begin{enumerate}
            \item $\displaystyle \left\| \bf{A} \right\| \geq 0$, при этом $\displaystyle \left\| \bf{A} \right\| = 0 \Leftrightarrow \bf{A} = \bf{O}$
            \item $\displaystyle \left\| \alpha \bf{A} \right\| = \left| \alpha \right| \left\| \bf{A} \right\|$
            \item $\left\| \bf{A} + \bf{B} \right\| \leq \left\| \bf{A} \right\| + \left\| \bf{B} \right\|$
            \item $\left\| \bf{A} \cdot \bf{B} \right\| \leq \left\| \bf{A} \right\| \cdot \left\| \bf{B} \right\|$
        \end{enumerate}
    \end{definition}
    Норма также называется \emph{канонической}, если к тому же выполняются следующие аксиомы:
    \begin{enumerate}
        \item[5.] $a_{ik} \leq \left\| \bf{A} \right\| \qquad \forall i, k$
        \item[6.] Если $\forall i, k \, \left| a_{ik} \right| \leq \left| b_{ik} \right|$, то $\left\| \bf{A} \right\| \leq \left\| \bf{B} \right\|$
    \end{enumerate}
    Примерами матричной нормы являются:
    \begin{equation*}
        \left\| \bf{A} \right\|_1 = \max_i \sum_{j=1}^{n} \left| a_{ij} \right|; \qquad \left\| \bf{A} \right\|_2 = \max_j \sum_{i=1}^{n} \left| a_{ij} \right|;
        \qquad \left\| \bf{A} \right\|_3 = \left( \sum_{i=1}^{n} \sum_{j=1}^{n} \left| a_{ij} \right|^2 \right)^{1/2}
    \end{equation*}
    Последняя называется \emph{Евклидовой нормой}. Все эти нормы являются каноническими, т.е. удовлетворяют всем шести аксиомам.

    \subsection{Матричный ряд и матричные функции.}
    Наиболее простыми матричными функциями являются полиномы. Для их построения необходимо образовать степени матрицы,
    которые определены только для квадратных матриц, поэтому далее речь пойдет о них.

    По определению $\displaystyle \bf{A}^k = \prod_{i=1}^k \bf{A}, \, \bf{A}^0 = \bf{E}$. Тогда матричный полином:
    \begin{equation*}
        \bf{P}_n \left( \bf{A} \right) = c_0 \bf{E} + c_1 \bf{A} + c_2 \bf{A}^2 + \dots + c_n \bf{A}^n
    \end{equation*}
    Аргументом является квадратная матрица $m \times m$, и значением будет матрица той же размерности.

    Теперь устремим $n$ к бесконечности, т.е. формально перейдем к бесконечной сумме
    \begin{equation}
        \bf{P}\left( \bf{A} \right) = \sum_{\gamma = 0}^{\infty} c_{\gamma} \bf{A}^{\gamma} \label{eq:matrix_row}
    \end{equation}
    Такая сумма называется степенным матричным рядом относительно матрицы \bf{A}. Матричному ряду естественно
    сопоставить скалярный ряд
    \begin{equation*}
        p(x) = \sum_{\gamma = 0}^{\infty} c_{\gamma} x^{\gamma}
    \end{equation*}
    Матричный ряд будем называться \emph{сходящимся}, если сходятся все $m^2$ скалярных рядов для элементов
    матрицы $\bf{P}\left( \bf{A} \right)$. Введенное понятие нормы позволяет установить достаточное условие
    сходимости матричного ряда. Введем матрицу $\displaystyle \bf{U}^{(\gamma)} = c_{\gamma}\bf{A}^{\gamma}$.
    Обозначим ее элементы за $\displaystyle u_{kj}^{(\gamma)}$, а элементы матрицы $\bf{P} \left( \bf{A} \right)$
    за $\displaystyle p_{kj}$. Тогда с учетом выполнения шести аксиом для канонической нормы имеем цепочку
    неравенств
    \begin{equation*}
        \displaystyle \left| p_{kj} \right| = \left| \sum_{\gamma = 0}^{\infty} u_{kj}^{(\gamma)} \right| \leq \sum_{\gamma = 0}^{\infty} \left| u_{kj}^{(\gamma)} \right|
        \leq \sum_{\gamma = 0}^{\infty} \left\| c_{\gamma} \bf{A}^{\gamma} \right\| = \sum_{\gamma = 0}^{\infty} \left| c_{\gamma} \right| \left\| \bf{A}^{\gamma} \right\|
        \leq \sum_{\gamma = 0}^{\infty} \left| c_{\gamma} \right| \left\| \bf{A} \right\|^{\gamma}
    \end{equation*}
    В результате \emph{достаточным} условием сходимости матричного ряда является выполнение условия
    \begin{equation}
        \left\| \bf{A} \right\| < R
    \end{equation}
    являющегося, в свою очередь, условием абсолютной сходимости скалярного степенного ряда, стоящего последним в цепочке
    неравенств. Здесь $R$ -- радиус сходимости скалярного степенного ряда.

    \begin{definition}[Матричная функция]
        Если матричный ряд сходится, то матрицу $\bf{P} \left( \bf{A} \right)$ называется
        \emph{матричной функцией}.
    \end{definition}
    Примеры матричных функций:
    \begin{gather*}
        e^{\bf{A}} = \sum_{k=0}^{\infty} \frac{\bf{A}^k}{k!} \qquad \cos\left( \bf{A} \right) = \sum_{k=0}^{\infty} \frac{(-1)^k \bf{A}^{2k}}{(2k)!} \\
        \sin \left( \bf{A} \right) = \sum_{k=0}^{\infty} \frac{(-1)^k \bf{A}^{2k + 1}}{(2k+1)!} \qquad \left( \bf{E} - \bf{A} \right)^{-1} = \sum_{k=0}^{\infty} \bf{A}^k
    \end{gather*}
\end{document}