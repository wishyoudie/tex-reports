\documentclass[a4paper,12pt]{article}
%%% Default imports
\usepackage{listings} % Code listings
\usepackage{graphicx} % Images
\usepackage{booktabs} % Better tables
\usepackage{makecell}
\usepackage{enumitem} % Lists
\usepackage{dsfont}
\usepackage{geometry} % Page geometry
\usepackage[utf8]{inputenc} % Encoding
\usepackage[T2A]{fontenc} % Font
\usepackage[english, russian]{babel} % Multi-language support
\usepackage{titling} % Better titles
\usepackage{textcomp} % Old-style numbers? Check difference
\usepackage{mathtext} % Russian text in math expressions? Check difference
\usepackage{amsmath, amsfonts, amssymb, amsthm, mathtools} % Mathematics
\usepackage{bm} % Bold math symbols
\usepackage{icomma} % Better comma in numbers within math mode
\usepackage{xifthen} % Better if-expressions
\usepackage{transparent} % Transparent colors
\usepackage{caption}    % }
\usepackage{subcaption} % } Captioning figures
\usepackage[table,xcdraw]{xcolor} % Colors
\usepackage{textpos} % Absolute positioning
\usepackage{upgreek} % Cool greek letters

\usepackage{fancyvrb}
\usepackage{fvextra}
\usepackage{chngcntr}

%%% Page geometry
% \setlength\parindent{0pt} % No indentation in paragraphs
\setlist{noitemsep} % No spacing between list items

\usepackage{float}
\usepackage{multirow}

\geometry{
    paper=a4paper,
    top=2.5cm,
    bottom=3cm,
    left=2.5cm,
    right=2.5cm,
    headheight=0.75cm,
    footskip=1.5cm,
    headsep=0.75cm,
}

%%% Numeration
\newcommand{\RNumb}[1]{\uppercase\expandafter{\romannumeral #1\relax}}
\newcommand{\thesec}{\arabic{section}}
\renewcommand\thesection{\arabic{section}}
\renewcommand\thesubsection{\thesection.\arabic{subsection}}
\renewcommand\thesubsubsection{\RNumb{\arabic{subsubsection}}}
\renewcommand{\sectionmark}[1]{\markright{\thesection\ #1}}
\renewcommand{\bf}{\textbf}
\renewcommand{\it}{\textit}
\def\hash{\texttt{\#}}
\def\cpp{\C\texttt{++}}

\counterwithin{figure}{section}
\counterwithin{table}{section}
\renewenvironment{titlepage}{\thispagestyle{empty}} % Include titlepage into page numeration

%%% Headers and footers
\usepackage{setspace}
\usepackage{fancyhdr}
\usepackage{lastpage}


%%% Graphics
% Custom colors
\definecolor{myblue}{RGB}{72, 184, 178}
\definecolor{myblue1}{RGB}{0, 109, 167}
\definecolor{commentgreen}{RGB}{2,112,10}
\definecolor{mauve}{rgb}{0.58,0,0.82}
\definecolor{amethyst}{RGB}{153, 102, 203}

\usepackage{pgfplots} % Plots
\usepackage{tikz}
\usetikzlibrary{3d,perspective,decorations.text}
\usetikzlibrary{animations}
\usetikzlibrary{positioning}
\usetikzlibrary{matrix}
\usepackage{tikz-cd}
\usetikzlibrary{cd}
\usetikzlibrary{karnaugh}
\pgfplotsset{width=6cm,compat=newest}
\usepackage{color}

\usepackage[framemethod=TikZ]{mdframed}
\newcommand{\definebox}[2]{\newcounter{#1}\newenvironment{#1}[1][]{\stepcounter{#1}\mdfsetup{frametitle={\tikz[baseline=(current bounding box.east),outer sep=0pt]\node[anchor=east,rectangle,fill=white]{\strut \MakeUppercase#1~\csname the#1\endcsname\ifstrempty{##1}{}{:~##1}};}}\mdfsetup{innertopmargin=1pt,linecolor=#2,linewidth=3pt,topline=true,frametitleaboveskip=\dimexpr-\ht\strutbox\relax,}\begin{mdframed}[]\relax}{\end{mdframed}}}
\definebox{definition}{black!90}
\definebox{theorem}{myblue1!90}
\definebox{demonstration}{amethyst!90}

\newcounter{Theorem}
\def\themytheorem{\thesection.\arabic{Theorem}}
\usepackage[most]{tcolorbox}
\tcbuselibrary{theorems}
\newtcbtheorem{Theorem}{Theorem}
{colframe=myblue!90,coltitle=black,colback=white,fonttitle=\bfseries}{Th}

%%% Code listings
\usepackage{matlab-prettifier}

\lstset{
    extendedchars=\true,
}
\lstdefinestyle{cpp} {
    language=C++,
    frame=tb,
    tabsize=4,
    showstringspaces=false,
    numbers=left,
    captionpos=b,
    columns=flexible,
    upquote=true,
    commentstyle=\color{commentgreen},
    keywordstyle=\color{blue},
    stringstyle=\color{commentgreen},
    basicstyle=\small\ttfamily,
    emph={int,char,double,float,unsigned,void,bool,size\_t},
    emphstyle={\color{blue}},
    escapechar=\&,
    classoffset=1,
    otherkeywords={>,<,.,;,-,!,=,~},
    morekeywords={>,<,.,;,-,!,=,~},
    keywordstyle=\color{black},
    classoffset=0,
}
\lstdefinestyle{py} {
    language=Python,
    frame=tb,
    tabsize=4,
    showstringspaces=false,
    numbers=left,
    captionpos=b,
    columns=flexible,
    upquote=true,
    commentstyle=\color{commentgreen},
    keywordstyle=\color{blue},
    stringstyle=\color{commentgreen},
    basicstyle=\small\ttfamily,
    emph={and,as,assert,break,class,continue,def,del,elif,else,except,False,finally,for,from,global,if,import,in,%
    is,lambda,None,nonlocal,not,or,pass,raise,return,True,try,while,with,yield},
    emphstyle={\color{blue}},
    classoffset=1,
    otherkeywords={>,<,.,;,-,!,=,~},
    morekeywords={>,<,.,;,-,!,=,~},
    keywordstyle=\color{black},
    classoffset=0,
}
\lstdefinestyle{def} {
    frame=tb,
    tabsize=4,
    showstringspaces=false,
    numbers=left,
    captionpos=b,
    columns=flexible,
    upquote=true,
    commentstyle=\color{black},
    keywordstyle=\color{black},
    stringstyle=\color{black},
    basicstyle=\small\ttfamily,
    emph={int,char,double,float,unsigned,void,bool,size\_t},
    emphstyle={\color{black}},
    escapechar=\&,
    classoffset=1,
    otherkeywords={>,<,.,;,-,!,=,~},
    morekeywords={>,<,.,;,-,!,=,~},
    keywordstyle=\color{black},
    classoffset=0,
}

%%% Other
\usepackage[normalem]{ulem} % }
\useunder{\uline}{\ul}{}    % } Underline text
\usepackage[colorlinks,urlcolor=blue,filecolor=blue,citecolor=blue,linkcolor = blue,unicode=true]{hyperref}
\usepackage{titlesec}
\titlelabel{\thetitle.\quad}
\usepackage{secdot}
\sectiondot{subsection}
\usepackage{kvmap} % Karnaugh-maps for logic functions
\usepackage{}
\newcommand{\projectname}[3]{
    \begin{center}
        \Large
        \textbf{#1}\\[10pt]
        \textbf{#2}\\[10pt]
        \normalsize
        #3
        \rule{\linewidth}{0.4pt}
    \end{center}
}

\newcommand{\hfconfiguration}[3]{
    \pagestyle{fancy}
    \fancyhead[LE,RO]{}
    \fancyhead[LO,RE]{#1}
    \renewcommand{\footrulewidth}{0.4pt}
    \fancyfoot[C]{\thepage/\pageref*{LastPage}}
    \fancyfoot[LO,RE]{#2}
    \fancyfoot[LE,RO]{#3}
}

\newcommand{\filename}[2]{
    \pagebreak
    \titleformat{\section}
    [display]
    {\bfseries\Large}
    {}
    {0ex}
    {
        \vspace{-4.5ex}
        % \rule{\textwidth}{1pt}
        #1 \centering
    % \vspace{1ex}
    }
    [
        \normalfont\large
        #2
        \rule{\textwidth}{0.4pt}
        \normalsize
    ]
}

\newcommand{\project}[6]{
    \projectname{#1}{#2}{#3}
    \pagestyle{empty}
    \tableofcontents
    \newpage
    \hfconfiguration{#4}{#5}{#6}
}
%%% Final touch
\usepackage{subfiles}


\begin{document}
    \begin{titlepage}
    \begin{center}
        \large Санкт-Петербургский политехнический университет Петра Великого\\
        \large Институт компьютерных наук и технологий \\
        \large Кафедра компьютерных систем и программных технологий\\[6cm]


        \huge Отчет по лабораторной работе №3\\[0.5cm]
        \large по дисциплине <<Схемотехника операционных устройств>>\\[0.1cm]
        \large\textbf{Триггеры}\\[5cm]
    \end{center}


    \begin{flushright}
        \begin{minipage}{0.25\textwidth}
            \begin{flushleft}

                \large\textbf{Работу выполнил:}\\
                \large Ильин В.П.\\
                \large {Группа:} 35300901/10005\\

                \large \textbf{Преподаватель:}\\
                \large Киселев И.О.

            \end{flushleft}
        \end{minipage}
    \end{flushright}

    \vfill

    \begin{center}
        \large Санкт-Петербург\\
        \large \the\year
    \end{center}
\end{titlepage}

\vfill
\newpage

    % \tableofcontents
    \hfconfiguration{Лабораторная работа №3}{}{}

    \section{Цели работы}
    \begin{itemize}
        \item Закрепление знания характеристик и режимов работы триггеров основных типов;
        \item получение практических навыков тестирования и управления триггерами;
        \item получение навыков ввода проекта в графическом редакторе пакета QP, тестирования
        и отладки проекта и анализа временных характеристик триггеров;
        \item получение навыков отладки цифровых устройств данного класса на физической модели;
        конфигурирование ПЛИС и экспериментальная проверка работы типовых устройств с триггерами
        при использовании лабораторной платы DiLaB.
    \end{itemize}
    \section{Исходные данные}
    \begin{table}[H]
        \centering
        \begin{tabular}{|c|c|c|c|}
            \hline Вариант & Длительность импульса & Фронт/спад & Частота \\
            \hline 8 & 8 нс & Спад & 1.5 Гц \\
            \hline
        \end{tabular}
    \end{table}
    \section{Ход работы}
    \subsection{Асинхронный RS-триггер}

    \begin{figure}[H]
		\centering
		\includegraphics[width=\linewidth]{polytech/scheme/report-lab3/subfiles/images/scheme-1}
		\caption{Разработанная схема триггера}
		\label{fig:scheme-1}
	\end{figure}
    
    \begin{figure}[H]
		\centering
		\includegraphics[width=\linewidth]{polytech/scheme/report-lab3/subfiles/images/tech-mv-1}
		\caption{Схема в Technology Map Viewer}
		\label{fig:tech-mv-1}
	\end{figure}
    
    \begin{table}[H]
        \centering
        \begin{tabular}{|c|cc|cc|c|}
        \hline
        \multirow{2}{*}{\begin{tabular}[c]{@{}c@{}}Дискретное\\ время $t$\\\end{tabular}} & \multicolumn{2}{c|}{Входные переменные} & \multicolumn{2}{c|}{Состояния}    & \multirow{2}{*}{Режим работы} \\ \cline{2-5}
                                                                                        & \multicolumn{1}{c|}{$S(t)$}     & $R(t)$    & \multicolumn{1}{c|}{$Q(t)$} & $nQ(t)$ &                               \\ \hline
        0                                                                               & \multicolumn{1}{c|}{0}        & 1       & \multicolumn{1}{c|}{1}    & 0     & Установка 1                   \\ \hline
        1                                                                               & \multicolumn{1}{c|}{1}        & 1       & \multicolumn{1}{c|}{1}    & 0     & Хранение 1                    \\ \hline
        2                                                                               & \multicolumn{1}{c|}{1}        & 0       & \multicolumn{1}{c|}{0}    & 1     & Установка 0                   \\ \hline
        3                                                                               & \multicolumn{1}{c|}{1}        & 1       & \multicolumn{1}{c|}{0}    & 1     & Хранение 0                    \\ \hline
        4                                                                               & \multicolumn{1}{c|}{0}        & 0       & \multicolumn{1}{c|}{1}    & 1     & Особое состояние                   \\ \hline
        \end{tabular}
        \caption{Таблица переходов триггера}
        \label{tab:tab-1}
    \end{table}
    
    \begin{figure}[H]
		\centering
		\includegraphics[width=\linewidth]{polytech/scheme/report-lab3/subfiles/images/wave-1}
		\caption{Временная диаграмма}
		\label{fig:wave-1}
	\end{figure}

    \begin{figure}[H]
		\centering
		\includegraphics[width=\linewidth]{polytech/scheme/report-lab3/subfiles/images/wave-1-2}
		\caption{Временная диаграмма с импульсами}
		\label{fig:wave-1-2}
	\end{figure}
    По результатам моделирования видно, что минимальная длительность сигнала, переключающего триггер
    составляет $3.6$ нс.
    \subsection{RS-триггер синхронизируемый уровнем}
    \begin{figure}[H]
		\centering
		\includegraphics[width=\linewidth]{polytech/scheme/report-lab3/subfiles/images/scheme-2}
		\caption{Разработанная схема триггера}
		\label{fig:scheme-2}
	\end{figure}
    \begin{table}[H]
        \centering
        \begin{tabular}{|c|ccc|cc|c|}
        \hline
        \multirow{2}{*}{\begin{tabular}[c]{@{}c@{}}Дискретное\\ время $t$\end{tabular}} & \multicolumn{3}{c|}{Входные переменные}                            & \multicolumn{2}{c|}{Состояния}         & \multirow{2}{*}{Режим работы}     \\ \cline{2-6}
                                                                                        & \multicolumn{1}{c|}{$C(t)$} & \multicolumn{1}{c|}{$S(t)$} & $R(t)$ & \multicolumn{1}{c|}{$Q(t)$} & $Q(t+1)$ &                                   \\ \hline
        0                                                                               & \multicolumn{1}{c|}{0}      & \multicolumn{1}{c|}{H}      & H      & \multicolumn{1}{c|}{0}      & 0        & \multirow{2}{*}{Хранение}         \\ \cline{1-6}
        1                                                                               & \multicolumn{1}{c|}{0}      & \multicolumn{1}{c|}{H}      & H      & \multicolumn{1}{c|}{1}      & 1        &                                   \\ \hline
        2                                                                               & \multicolumn{1}{c|}{1}      & \multicolumn{1}{c|}{0}      & 0      & \multicolumn{1}{c|}{0}      & 0        & \multirow{2}{*}{Хранение}         \\ \cline{1-6}
        3                                                                               & \multicolumn{1}{c|}{1}      & \multicolumn{1}{c|}{0}      & 0      & \multicolumn{1}{c|}{1}      & 1        &                                   \\ \hline
        4                                                                               & \multicolumn{1}{c|}{1}      & \multicolumn{1}{c|}{1}      & 0      & \multicolumn{1}{c|}{0}      & 1        & \multirow{2}{*}{Запись 1}         \\ \cline{1-6}
        5                                                                               & \multicolumn{1}{c|}{1}      & \multicolumn{1}{c|}{1}      & 0      & \multicolumn{1}{c|}{1}      & 1        &                                   \\ \hline
        6                                                                               & \multicolumn{1}{c|}{1}      & \multicolumn{1}{c|}{0}      & 1      & \multicolumn{1}{c|}{0}      & 0        & \multirow{2}{*}{Запись 0}         \\ \cline{1-6}
        7                                                                               & \multicolumn{1}{c|}{1}      & \multicolumn{1}{c|}{0}      & 1      & \multicolumn{1}{c|}{1}      & 0        &                                   \\ \hline
        8                                                                               & \multicolumn{1}{c|}{1}      & \multicolumn{1}{c|}{1}      & 1      & \multicolumn{1}{c|}{0}      & H        & \multirow{2}{*}{Особое состояние} \\ \cline{1-6}
        9                                                                               & \multicolumn{1}{c|}{1}      & \multicolumn{1}{c|}{1}      & 1      & \multicolumn{1}{c|}{1}      & H        &                                   \\ \hline
        \end{tabular}
        \caption{Таблица переходов триггера}
        \label{tab:tab-2}
    \end{table}
    \begin{figure}[H]
		\centering
		\includegraphics[width=\linewidth]{polytech/scheme/report-lab3/subfiles/images/wave-2}
		\caption{Временная диаграмма}
		\label{fig:wave-2}
	\end{figure}
    Из результатов моделирования видно, что триггер синхронизируется уровнем, а не перепадом.
    При переходе триггера из особого состояния в состояние хранения на его выходе сохраняется <<1>>.
    
    \subsection{Использование примитива DFFE}
    \begin{figure}[H]
		\centering
		\includegraphics[width=\linewidth]{polytech/scheme/report-lab3/subfiles/images/scheme-3}
		\caption{Разработанная схема триггера}
		\label{fig:scheme-3}
	\end{figure}
    \begin{figure}[H]
		\centering
		\includegraphics[width=\linewidth]{polytech/scheme/report-lab3/subfiles/images/wave-3}
		\caption{Временная диаграмма}
		\label{fig:wave-3}
	\end{figure}
    При одновременной подаче активного уровня на входы PRN и CLRN триггер устанавливается в 0.
    \subsection{Использование примитива JKFFE}
    
    \begin{figure}[H]
		\centering
		\includegraphics[width=\linewidth]{polytech/scheme/report-lab3/subfiles/images/scheme-4}
		\caption{Разработанная схема триггера}
		\label{fig:scheme-4}
	\end{figure}
    \begin{figure}[H]
		\centering
		\includegraphics[width=\linewidth]{polytech/scheme/report-lab3/subfiles/images/wave-4}
		\caption{Временная диаграмма}
		\label{fig:wave-4}
	\end{figure}
    При одновременной подаче активного уровня на входы PRN и CLRN триггер устанавливается в 0.
    \subsection{Генератор коротких импульсов}
    \begin{figure}[H]
		\centering
		\includegraphics[width=\linewidth]{polytech/scheme/report-lab3/subfiles/images/scheme-5}
		\caption{Разработанная схема триггера}
		\label{fig:scheme-5}
	\end{figure}
    
    Устанавливая элементы LCELL, получаем длительность формируемого импульса в 8 нс. Всего потребовалось
    15 элементов. Разница в примерно 8 нс можно увидеть в значении графы <<Master Time Bar>> на рисунках
    ниже.

    \begin{minipage}{0.49\linewidth}
        \begin{figure}[H]
            \centering
            \includegraphics[width=\linewidth]{polytech/scheme/report-lab3/subfiles/images/wave-5-1}
            \caption{Временная диаграмма}
            \label{fig:wave-5-1}
        \end{figure}
    \end{minipage}
    \begin{minipage}{0.49\linewidth}
        \begin{figure}[H]
            \centering
            \includegraphics[width=\linewidth]{polytech/scheme/report-lab3/subfiles/images/wave-5-2}
            \caption{Временная диаграмма}
            \label{fig:wave-5-2}
        \end{figure}
    \end{minipage}

    Используя Chip Planner, посмотрим на расположение данной схемы на кристалле и функциональным преобразователем. 
    \begin{figure}[H]
		\centering
		\includegraphics[width=0.38\linewidth]{polytech/scheme/report-lab3/subfiles/images/crystal}
		\caption{Размещение на кристалле}
		\label{fig:crystal}
	\end{figure}
    \begin{figure}[H]
		\centering
		\includegraphics[width=\linewidth]{polytech/scheme/report-lab3/subfiles/images/func_trans}
		\caption{Функциональный преобразователь}
		\label{fig:func_trans}
	\end{figure}

    \subsection{Устройство удвоения частоты}
    Для создания подобного устройства объединим 2 схемы из предыдущего пункта хода работы.
    Один триггер будет формировать единичный импульс при фронте C, а другой при спаде.
    \begin{figure}[H]
		\centering
		\includegraphics[width=0.9\linewidth]{polytech/scheme/report-lab3/subfiles/images/scheme-7}
		\caption{Разработанная схема устройства}
		\label{fig:scheme-7}
	\end{figure}
    На временной диаграмме видим формирование импульсов как на фронте, так и на спаде.
    \begin{figure}[H]
		\centering
		\includegraphics[width=0.9\linewidth]{polytech/scheme/report-lab3/subfiles/images/wave-7}
		\caption{Временная диаграмма}
		\label{fig:wave-7}
	\end{figure}

    \subsection{Устройство выявления спада}
    Для выявления спада сигнала необходимо смотреть на результат логической функции $\overline{D} \cdot Q$.
    \begin{figure}[H]
		\centering
		\includegraphics[width=\linewidth]{polytech/scheme/report-lab3/subfiles/images/scheme-8}
		\caption{Разработанная схема устройства}
		\label{fig:scheme-8}
	\end{figure}
    \begin{figure}[H]
		\centering
		\includegraphics[width=\linewidth]{polytech/scheme/report-lab3/subfiles/images/wave-8}
		\caption{Временная диаграмма}
		\label{fig:wave-8}
	\end{figure}

    \section{Вывод}
    В ходе выполнения работы были закреплены знания характеристик и режимов работы триггеров основных типов.
    Были получены практические навыки тестирования и управления триггерами. Была проведена экспериментальная
    проверка работы типовых устройств с триггерами.
\end{document}
