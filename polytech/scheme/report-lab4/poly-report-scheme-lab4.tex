\documentclass[a4paper,12pt]{article}
%%% Default imports
\usepackage{listings} % Code listings
\usepackage{graphicx} % Images
\usepackage{booktabs} % Better tables
\usepackage{makecell}
\usepackage{enumitem} % Lists
\usepackage{dsfont}
\usepackage{geometry} % Page geometry
\usepackage[utf8]{inputenc} % Encoding
\usepackage[T2A]{fontenc} % Font
\usepackage[english, russian]{babel} % Multi-language support
\usepackage{titling} % Better titles
\usepackage{textcomp} % Old-style numbers? Check difference
\usepackage{mathtext} % Russian text in math expressions? Check difference
\usepackage{amsmath, amsfonts, amssymb, amsthm, mathtools} % Mathematics
\usepackage{bm} % Bold math symbols
\usepackage{icomma} % Better comma in numbers within math mode
\usepackage{xifthen} % Better if-expressions
\usepackage{transparent} % Transparent colors
\usepackage{caption}    % }
\usepackage{subcaption} % } Captioning figures
\usepackage[table,xcdraw]{xcolor} % Colors
\usepackage{textpos} % Absolute positioning
\usepackage{upgreek} % Cool greek letters

\usepackage{fancyvrb}
\usepackage{fvextra}
\usepackage{chngcntr}

%%% Page geometry
% \setlength\parindent{0pt} % No indentation in paragraphs
\setlist{noitemsep} % No spacing between list items

\usepackage{float}
\usepackage{multirow}

\geometry{
    paper=a4paper,
    top=2.5cm,
    bottom=3cm,
    left=2.5cm,
    right=2.5cm,
    headheight=0.75cm,
    footskip=1.5cm,
    headsep=0.75cm,
}

%%% Numeration
\newcommand{\RNumb}[1]{\uppercase\expandafter{\romannumeral #1\relax}}
\newcommand{\thesec}{\arabic{section}}
\renewcommand\thesection{\arabic{section}}
\renewcommand\thesubsection{\thesection.\arabic{subsection}}
\renewcommand\thesubsubsection{\RNumb{\arabic{subsubsection}}}
\renewcommand{\sectionmark}[1]{\markright{\thesection\ #1}}
\renewcommand{\bf}{\textbf}
\renewcommand{\it}{\textit}
\def\hash{\texttt{\#}}
\def\cpp{\C\texttt{++}}

\counterwithin{figure}{section}
\counterwithin{table}{section}
\renewenvironment{titlepage}{\thispagestyle{empty}} % Include titlepage into page numeration

%%% Headers and footers
\usepackage{setspace}
\usepackage{fancyhdr}
\usepackage{lastpage}


%%% Graphics
% Custom colors
\definecolor{myblue}{RGB}{72, 184, 178}
\definecolor{myblue1}{RGB}{0, 109, 167}
\definecolor{commentgreen}{RGB}{2,112,10}
\definecolor{mauve}{rgb}{0.58,0,0.82}
\definecolor{amethyst}{RGB}{153, 102, 203}

\usepackage{pgfplots} % Plots
\usepackage{tikz}
\usetikzlibrary{3d,perspective,decorations.text}
\usetikzlibrary{animations}
\usetikzlibrary{positioning}
\usetikzlibrary{matrix}
\usepackage{tikz-cd}
\usetikzlibrary{cd}
\usetikzlibrary{karnaugh}
\pgfplotsset{width=6cm,compat=newest}
\usepackage{color}

\usepackage[framemethod=TikZ]{mdframed}
\newcommand{\definebox}[2]{\newcounter{#1}\newenvironment{#1}[1][]{\stepcounter{#1}\mdfsetup{frametitle={\tikz[baseline=(current bounding box.east),outer sep=0pt]\node[anchor=east,rectangle,fill=white]{\strut \MakeUppercase#1~\csname the#1\endcsname\ifstrempty{##1}{}{:~##1}};}}\mdfsetup{innertopmargin=1pt,linecolor=#2,linewidth=3pt,topline=true,frametitleaboveskip=\dimexpr-\ht\strutbox\relax,}\begin{mdframed}[]\relax}{\end{mdframed}}}
\definebox{definition}{black!90}
\definebox{theorem}{myblue1!90}
\definebox{demonstration}{amethyst!90}

\newcounter{Theorem}
\def\themytheorem{\thesection.\arabic{Theorem}}
\usepackage[most]{tcolorbox}
\tcbuselibrary{theorems}
\newtcbtheorem{Theorem}{Theorem}
{colframe=myblue!90,coltitle=black,colback=white,fonttitle=\bfseries}{Th}

%%% Code listings
\usepackage{matlab-prettifier}

\lstset{
    extendedchars=\true,
}
\lstdefinestyle{cpp} {
    language=C++,
    frame=tb,
    tabsize=4,
    showstringspaces=false,
    numbers=left,
    captionpos=b,
    columns=flexible,
    upquote=true,
    commentstyle=\color{commentgreen},
    keywordstyle=\color{blue},
    stringstyle=\color{commentgreen},
    basicstyle=\small\ttfamily,
    emph={int,char,double,float,unsigned,void,bool,size\_t},
    emphstyle={\color{blue}},
    escapechar=\&,
    classoffset=1,
    otherkeywords={>,<,.,;,-,!,=,~},
    morekeywords={>,<,.,;,-,!,=,~},
    keywordstyle=\color{black},
    classoffset=0,
}
\lstdefinestyle{py} {
    language=Python,
    frame=tb,
    tabsize=4,
    showstringspaces=false,
    numbers=left,
    captionpos=b,
    columns=flexible,
    upquote=true,
    commentstyle=\color{commentgreen},
    keywordstyle=\color{blue},
    stringstyle=\color{commentgreen},
    basicstyle=\small\ttfamily,
    emph={and,as,assert,break,class,continue,def,del,elif,else,except,False,finally,for,from,global,if,import,in,%
    is,lambda,None,nonlocal,not,or,pass,raise,return,True,try,while,with,yield},
    emphstyle={\color{blue}},
    classoffset=1,
    otherkeywords={>,<,.,;,-,!,=,~},
    morekeywords={>,<,.,;,-,!,=,~},
    keywordstyle=\color{black},
    classoffset=0,
}
\lstdefinestyle{def} {
    frame=tb,
    tabsize=4,
    showstringspaces=false,
    numbers=left,
    captionpos=b,
    columns=flexible,
    upquote=true,
    commentstyle=\color{black},
    keywordstyle=\color{black},
    stringstyle=\color{black},
    basicstyle=\small\ttfamily,
    emph={int,char,double,float,unsigned,void,bool,size\_t},
    emphstyle={\color{black}},
    escapechar=\&,
    classoffset=1,
    otherkeywords={>,<,.,;,-,!,=,~},
    morekeywords={>,<,.,;,-,!,=,~},
    keywordstyle=\color{black},
    classoffset=0,
}

%%% Other
\usepackage[normalem]{ulem} % }
\useunder{\uline}{\ul}{}    % } Underline text
\usepackage[colorlinks,urlcolor=blue,filecolor=blue,citecolor=blue,linkcolor = blue,unicode=true]{hyperref}
\usepackage{titlesec}
\titlelabel{\thetitle.\quad}
\usepackage{secdot}
\sectiondot{subsection}
\usepackage{kvmap} % Karnaugh-maps for logic functions
\usepackage{}
\newcommand{\projectname}[3]{
    \begin{center}
        \Large
        \textbf{#1}\\[10pt]
        \textbf{#2}\\[10pt]
        \normalsize
        #3
        \rule{\linewidth}{0.4pt}
    \end{center}
}

\newcommand{\hfconfiguration}[3]{
    \pagestyle{fancy}
    \fancyhead[LE,RO]{}
    \fancyhead[LO,RE]{#1}
    \renewcommand{\footrulewidth}{0.4pt}
    \fancyfoot[C]{\thepage/\pageref*{LastPage}}
    \fancyfoot[LO,RE]{#2}
    \fancyfoot[LE,RO]{#3}
}

\newcommand{\filename}[2]{
    \pagebreak
    \titleformat{\section}
    [display]
    {\bfseries\Large}
    {}
    {0ex}
    {
        \vspace{-4.5ex}
        % \rule{\textwidth}{1pt}
        #1 \centering
    % \vspace{1ex}
    }
    [
        \normalfont\large
        #2
        \rule{\textwidth}{0.4pt}
        \normalsize
    ]
}

\newcommand{\project}[6]{
    \projectname{#1}{#2}{#3}
    \pagestyle{empty}
    \tableofcontents
    \newpage
    \hfconfiguration{#4}{#5}{#6}
}
%%% Final touch
\usepackage{subfiles}


\begin{document}
    \begin{titlepage}
		\begin{center}
			\large Санкт-Петербургский политехнический университет Петра Великого\\
			\large Институт компьютерных наук и технологий \\
			\large Кафедра компьютерных систем и программных технологий\\[6cm]


		\huge КУРСОВАЯ РАБОТА\\[0.5cm]
			\large по дисциплине <<Вычислительная математика>>\\[0.1cm]
			\large\textbf{Исследование уравнения движения пузырьков}\\[5cm]
		\end{center}


		\begin{flushright}
			\begin{minipage}{0.25\textwidth}
				\begin{flushleft}

					\large\textbf{Работу выполнил:}\\
					\large Ильин В.П.\\
					\large {Группа:} 3530901/10005\\

					\large \textbf{Преподаватель:}\\
					\large Куляшова З.В.

				\end{flushleft}
			\end{minipage}
		\end{flushright}

		\vfill

		\begin{center}
			\large Санкт-Петербург\\
			\large \the\year
		\end{center}
	\end{titlepage}

	\vfill
	\newpage


    \tableofcontents
    \hfconfiguration{Лабораторная работа №4}{}{}

    \section{Цели работы}
    \begin{itemize}
        \item Закрепление навыков структурного синтеза конечных автоматов;
        \item Закрепление знаний о характеристиках и режимах работы триггеров основных типов;
        \item Получение практических навыков тестирования и управления КА;
        \item Получение навыков ввода прокта в графическом редакторе пакета QP,
        тестирования и отладки проекта и анализа временных характеристик КА;
        \item Знакомство с редактором КА пакета QP  и анализ результатов синтеза;
        \item Получение навыков отладки цифровых устройств класса КА на физической модели:
        конфигурирование ПЛИС и экспериментальная проверка работы КА при использовании
        лабораторного стенда.
    \end{itemize}
    \section{Исходные данные}
    Вариант исходного задания -- 8. Тип триггера -- JK.

    \begin{minipage}{0.5\linewidth}
        \begin{table}[H]
            \centering
            \begin{tabular}{|cl|cccc|}
            \hline
            \multicolumn{2}{|c|}{\multirow{2}{*}{$(x_2, x_1)^t$}} & \multicolumn{4}{c|}{$r^t$}                                                   \\ \cline{3-6} 
            \multicolumn{2}{|c|}{}                                & \multicolumn{1}{c|}{0} & \multicolumn{1}{c|}{1} & \multicolumn{1}{c|}{2} & 3 \\ \hline
            \multicolumn{2}{|c|}{0 0}                             & \multicolumn{1}{c|}{2} & \multicolumn{1}{c|}{H} & \multicolumn{1}{c|}{0} & 1 \\ \hline
            \multicolumn{2}{|c|}{0 1}                             & \multicolumn{1}{c|}{H} & \multicolumn{1}{c|}{H} & \multicolumn{1}{c|}{H} & H \\ \hline
            \multicolumn{2}{|c|}{1 0}                             & \multicolumn{1}{c|}{3} & \multicolumn{1}{c|}{0} & \multicolumn{1}{c|}{H} & 2 \\ \hline
            \multicolumn{2}{|c|}{1 1}                             & \multicolumn{1}{c|}{H} & \multicolumn{1}{c|}{2} & \multicolumn{1}{c|}{3} & 1 \\ \hline
            \end{tabular}
            \caption{Таблица переходов}
            \label{tab:transtab}
        \end{table}
    \end{minipage}
    \begin{minipage}[b]{0.5\linewidth}
        \begin{table}[H]
            \centering
            \begin{tabular}{|c|c|c|c|c|}
            \hline
            $r^t$          & 0  & 1  & 2  & 3  \\ \hline
            $(y_2, y_1)^t$ & 01 & 10 & 10 & 01 \\ \hline
            \end{tabular}
            \caption{Таблица выходов}
        \end{table}
    \end{minipage}

    \newpage
    \section{Ход работы}
    \subsection{Структурный синтез КА}
    При помощи таблицы~\ref{tab:transtab} составим граф переходов.

    \begin{figure}[H]
    \begin{center}
        \begin{tikzpicture}[
            nodecirc/.style={circle, draw=black, fill=gray!30, thick},
            nofill/.style={circle, draw=cyan},
            filled/.style={circle, draw=cyan, fill=cyan}
        ]
        \matrix (m) [ matrix of nodes, row sep=2cm, column sep=4cm,
                    nodes = {anchor=center}] {
                    |[nodecirc]{$r_0$};| & |[nodecirc]{$r_1$};| \\
                    |[nodecirc]{$r_3$};| & |[nodecirc]{$r_2$};| \\
        };
        \draw[->, thick] (m-1-1) to node[left] {10} (m-2-1);
        \draw[->, thick] (m-1-2) to node[above] {10} (m-1-1);
        \draw[->, thick] (m-1-2) to node[right] {11} (m-2-2);
        \draw[->, thick] (m-2-2) to node[above] {11} (m-2-1);
        \def\myshift#1{\raisebox{-2.5ex}}
        \draw [->,thick,postaction={decorate,decoration={text along path,text align=center,text={|\myshift|10}}}] (m-2-1) to [bend right=15]  (m-2-2);
        \draw [->,thick,postaction={decorate,decoration={text along path,text align=center,text={|\myshift|00}}}] (m-2-1) to [bend right=5]  (m-1-2);
        \def\myshift#1{\raisebox{1ex}}
        \draw [->,thick,postaction={decorate,decoration={text along path,text align=center,text={|\myshift|11}}}] (m-2-1) to [bend right=-5]  (m-1-2);
        \draw[<->, thick, red] (m-2-2) to node[right] {00} (m-1-1);
        \end{tikzpicture}
        \caption{Граф переходов синтезируемого автомата}
    \end{center}
\end{figure}
    Всего автомат имеет 4 различных состояния, значит, минимальное необходиоме число триггеров $m = \log_2 4 = 2$.
    Воспользовавшись экономичным кодированием внутренних состояний, получим необходимые коды:
    \begin{table}[H]
        \centering
        \begin{tabular}{|c|c|c|c|c|}
        \hline
        $r^t$          & 0  & 1  & 2  & 3  \\ \hline
        $(Q_2, Q_1)^t$ & 00 & 01 & 11 & 10 \\ \hline
        \end{tabular}
        \caption{Коды состояний автомата}
        \label{tab:statecodes}
    \end{table}
    Воспользовавшись таблицами~\ref{tab:transtab} и~\ref{tab:statecodes} получим закодированную таблицу
    переходов синтезируемого КА:
    \begin{table}[H]
        \centering
        \begin{tabular}{|cl|cccc|}
        \hline
        \multicolumn{2}{|c|}{\multirow{2}{*}{$(x_2, x_1)^t$}} & \multicolumn{4}{c|}{$(Q_2, Q_1)^t$}                                                   \\ \cline{3-6} 
        \multicolumn{2}{|c|}{}                                & \multicolumn{1}{c|}{00} & \multicolumn{1}{c|}{01} & \multicolumn{1}{c|}{11} & 10 \\ \hline
        \multicolumn{2}{|c|}{0 0}                             & \multicolumn{1}{c|}{11} & \multicolumn{1}{c|}{H} & \multicolumn{1}{c|}{00} & 01 \\ \hline
        \multicolumn{2}{|c|}{0 1}                             & \multicolumn{1}{c|}{H} & \multicolumn{1}{c|}{H} & \multicolumn{1}{c|}{H} & H \\ \hline
        \multicolumn{2}{|c|}{1 0}                             & \multicolumn{1}{c|}{10} & \multicolumn{1}{c|}{00} & \multicolumn{1}{c|}{H} & 11 \\ \hline
        \multicolumn{2}{|c|}{1 1}                             & \multicolumn{1}{c|}{H} & \multicolumn{1}{c|}{11} & \multicolumn{1}{c|}{10} & 01 \\ \hline
        \end{tabular}
        \caption{Таблица переходов}
        \label{tab:transtab_coded}
    \end{table}    
    Используя таблицу истинности для JK-триггера, построим таблицу управления триггером:
    \begin{table}[H]
        \centering
        \begin{tabular}{|c|c|c|c|c|c|c|}
        \hline
        $(x_2, x_1)^t$ & $(Q_2, Q_1)^t$ & $(Q_2, Q_1)^{t+1}$ & $J_2$ & $K_2$ & $J_1$ & $K_1$ \\ \hline
        00             & 00             & 11                 & 1     & H     & 1     & H     \\ \hline
        00             & 01             & H                  & H     & H     & H     & H     \\ \hline
        00             & 11             & 00                 & H     & 1     & H     & 1     \\ \hline
        00             & 10             & 01                 & H     & 1     & 1     & H     \\ \Xhline{4\arrayrulewidth}
        01             & 00             & H                  & H     & H     & H     & H     \\ \hline
        01             & 01             & H                  & H     & H     & H     & H     \\ \hline
        01             & 11             & H                  & H     & H     & H     & H     \\ \hline
        01             & 10             & H                  & H     & H     & H     & H     \\ \Xhline{4\arrayrulewidth}
        10             & 00             & 10                 & 1     & H     & 0     & H     \\ \hline
        10             & 01             & 00                 & 0     & H     & H     & 1     \\ \hline
        10             & 11             & H                  & H     & H     & H     & H     \\ \hline
        10             & 10             & 11                 & H     & 0     & 1     & H     \\ \Xhline{4\arrayrulewidth}
        11             & 00             & H                  & H     & H     & H     & H     \\ \hline
        11             & 01             & 11                 & 1     & H     & H     & 0     \\ \hline
        11             & 11             & 10                 & H     & 0     & H     & 1     \\ \hline
        11             & 10             & 01                 & H     & 1     & 1     & H     \\ \hline
        \end{tabular}
        \caption{Таблица функций возбуждения триггеров}
        \label{tab:erection}
    \end{table}
    Произведем минимизацию полученных функций при помощи карт Карно:

    \begin{minipage}{0.5\linewidth}
        \begin{center}
            \begin{tikzpicture}[karnaugh,
                thick,
                grp/.style n args={3}{#1,fill=#1!30,
                    minimum width=#2\kmunitlength,
                    minimum height=#3\kmunitlength,
                    rounded corners=0.2\kmunitlength,
                    fill opacity=0.6,
                    rectangle,draw}]]
                \karnaughmaptab{4}{$J_2$}{{$x_2$}{$x_1$}{$Q_2$}{$Q_1$}}%
                {1HHH HHHH H1HH 10HH}%
                {
                    \node[grp={blue}{0.9}{3.9}](n000) at (0.5, 2.0) {};
                    \node[grp={red}{3.9}{1.9}](n001) at (2.0, 2.0) {};
                }
            \end{tikzpicture}
        
            $J_2 = \color{blue}\overline{Q}_1 \overline{Q_2} \color{black} + \color{red} x_1 \color{black} = \overline{\overline{x}_1 \overline{\overline{Q}_1 \overline{Q}_2}}$
        \end{center}
    \end{minipage}
    \begin{minipage}{0.5\linewidth}
        \begin{center}
            \begin{tikzpicture}[karnaugh,
                thick,
                grp/.style n args={3}{#1,fill=#1!30,
                    minimum width=#2\kmunitlength,
                    minimum height=#3\kmunitlength,
                    rounded corners=0.2\kmunitlength,
                    fill opacity=0.6,
                    rectangle,draw}]]
                \karnaughmaptab{4}{$K_2$}{{$x_2$}{$x_1$}{$Q_2$}{$Q_1$}}%
                {HH11 HHHH HH01 HHH0}%
                {
                    \node[grp={blue}{3.9}{1.9}](n001) at (2.0, 3.0) {};
                    \node[grp={red}{0.9}{1.9}](n000) at (3.5, 2.0) {};
                    \node[grp={red}{0.9}{1.9}](n002) at (0.5, 2.0) {};
                }
            \end{tikzpicture}
        
            $K_2 = \color{blue}\overline{x}_2 \color{black} + \color{red} x_1\overline{Q}_1 \color{black} = \overline{x_2 \overline{x_1 \overline{Q}_1}}$
        \end{center}
    \end{minipage}

    \begin{minipage}{0.5\linewidth}
        \begin{center}
            \begin{tikzpicture}[karnaugh,
                thick,
                grp/.style n args={3}{#1,fill=#1!30,
                    minimum width=#2\kmunitlength,
                    minimum height=#3\kmunitlength,
                    rounded corners=0.2\kmunitlength,
                    fill opacity=0.6,
                    rectangle,draw}]]
                \karnaughmaptab{4}{$J_1$}{{$x_2$}{$x_1$}{$Q_2$}{$Q_1$}}%
                {1HH1 HHHH HHH1 0HH1}%
                {
                    \node {};
                    \node[grp={blue}{3.9}{1.9}](n001) at (2.0, 3.0) {};
                    \node[grp={red}{1.9}{3.9}](n000) at (3.0, 2.0) {};
                }
            \end{tikzpicture}
        
            $J_1 = \color{blue}\overline{x}_2 \color{black} + \color{red} Q_2 \color{black} = \overline{x_2 \overline{Q}_2}$
        \end{center}
    \end{minipage}
    \begin{minipage}{0.5\linewidth}
        \begin{center}
            \begin{tikzpicture}[karnaugh,
                thick,
                grp/.style n args={3}{#1,fill=#1!30,
                    minimum width=#2\kmunitlength,
                    minimum height=#3\kmunitlength,
                    rounded corners=0.2\kmunitlength,
                    fill opacity=0.6,
                    rectangle,draw}]]
                \karnaughmaptab{4}{$K_1$}{{$x_2$}{$x_1$}{$Q_2$}{$Q_1$}}%
                {HH1H HHHH H01H H1HH}%
                {
                    \node {};
                    \node[grp={blue}{3.9}{0.9}](n001) at (2.0, 3.5) {};
                    \node[grp={blue}{3.9}{0.9}](n002) at (2.0, 0.5) {};
                    \node[grp={red}{1.9}{3.9}](n000) at (3.0, 2.0) {};
                }
            \end{tikzpicture}
        
            $K_1 = \color{blue}\overline{x}_1 \color{black} + \color{red} Q_2 \color{black} = \overline{x_1 \overline{Q}_2}$
        \end{center}
    \end{minipage}

    Также составим таблицу для выходных сигналов, как функций состояния автоматов, и проведем минимизацию.
    \begin{table}[H]
        \centering
        \begin{tabular}{|c|c|}
        \hline
        $(Q_2, Q_1)$ & $(y_2, y_1)$ \\ \hline
        0 0          & 0 1          \\ \hline
        0 1          & 1 0          \\ \hline
        1 1          & 1 0          \\ \hline
        1 0          & 0 1          \\ \hline
        \end{tabular}
    \end{table}
    \[y_1 = \overline{Q}_1, \, y_2 = Q_2 \overline{Q}_1 + \overline{Q}_2 Q_1 = \overline{\overline{Q_2 \overline{Q}_1} \cdot \overline{\overline{Q}_2 Q_1}}\]

    \subsection{Исследование синтезированного автомата}
    Введем схему синтезированного автомата в Quartus Prime.
    \begin{figure}[H]
        \centering
        \includegraphics[width=\linewidth]{polytech/scheme/report-lab4/subfiles/images/scheme}
        \caption{Синтезированная схема}
        \label{fig:scheme}
    \end{figure}
    \begin{figure}[H]
        \centering
        \includegraphics[width=\linewidth]{polytech/scheme/report-lab4/subfiles/images/tmv1}
        \caption{Technology Map Viewer}
        \label{fig:tmv1}
    \end{figure}
    \begin{figure}[H]
        \centering
        \includegraphics[width=.7\linewidth]{polytech/scheme/report-lab4/subfiles/images/app}
        \caption{Аппаратные затраты}
        \label{fig:app}
    \end{figure}
    \begin{figure}[H]
        \centering
        \includegraphics[width=\linewidth]{polytech/scheme/report-lab4/subfiles/images/wave}
        \caption{Моделирование работы}
        \label{fig:wave}
    \end{figure}

    Сравнение выходных результатов для Q и Y подтверждает правильность работы устройства.

    \subsection{Синтез конечного автомата средствами Quartus Prime}

    Теперь создадим автомат при помощи встроенных средств среды Quartus.

    \begin{figure}[H]
        \centering
        \includegraphics[width=0.25\linewidth]{polytech/scheme/report-lab4/subfiles/images/smw1}
        \includegraphics[width=0.4\linewidth]{polytech/scheme/report-lab4/subfiles/images/smw2}
        \includegraphics[width=0.15\linewidth]{polytech/scheme/report-lab4/subfiles/images/smw3}

        \begin{minipage}{0.45\linewidth}
            \includegraphics[width=\linewidth]{polytech/scheme/report-lab4/subfiles/images/smw4}
        \end{minipage}
        \begin{minipage}{0.45\linewidth}
            \includegraphics[width=\linewidth]{polytech/scheme/report-lab4/subfiles/images/smw5}
        \end{minipage}
        \caption{Настройки создания автомата}
    \end{figure}

    \begin{figure}[H]
        \centering
        \includegraphics[width=\linewidth]{polytech/scheme/report-lab4/subfiles/images/scheme_machine}
        \caption{Синтезированная схема}
    \end{figure}
    \begin{figure}[H]
        \centering
        \includegraphics[width=0.7\linewidth]{polytech/scheme/report-lab4/subfiles/images/compile_machine}
        \caption{Отчет о компиляции}
    \end{figure}
    \begin{figure}[H]
        \centering
        \includegraphics[width=\linewidth]{polytech/scheme/report-lab4/subfiles/images/smv}
        \caption{State Machine Viewer}
    \end{figure}
    \section{Вывод}
    В ходе работы были закреплены знания характеристик и режимов работы триггеров. Были получены навыки
    структурного синтеза, тестирования и управления конечными автоматами. Конечный автомат на основе заданных
    данных был синтезирован вручную, а также при помощи встроенных средств Quartus Prime. Автомат, полученный
    вручную работает медленее и содержит большее число элементов, чем созданный автоматически. Помимо оптимизаций,
    производимых Quartus это также связано с тем, что для тестирования <<ручной>> автомат выводил промежуточные значения.
\end{document}
