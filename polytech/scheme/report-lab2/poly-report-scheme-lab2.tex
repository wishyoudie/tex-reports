\documentclass[a4paper,12pt]{article}
%%% Default imports
\usepackage{listings} % Code listings
\usepackage{graphicx} % Images
\usepackage{booktabs} % Better tables
\usepackage{makecell}
\usepackage{enumitem} % Lists
\usepackage{dsfont}
\usepackage{geometry} % Page geometry
\usepackage[utf8]{inputenc} % Encoding
\usepackage[T2A]{fontenc} % Font
\usepackage[english, russian]{babel} % Multi-language support
\usepackage{titling} % Better titles
\usepackage{textcomp} % Old-style numbers? Check difference
\usepackage{mathtext} % Russian text in math expressions? Check difference
\usepackage{amsmath, amsfonts, amssymb, amsthm, mathtools} % Mathematics
\usepackage{bm} % Bold math symbols
\usepackage{icomma} % Better comma in numbers within math mode
\usepackage{xifthen} % Better if-expressions
\usepackage{transparent} % Transparent colors
\usepackage{caption}    % }
\usepackage{subcaption} % } Captioning figures
\usepackage[table,xcdraw]{xcolor} % Colors
\usepackage{textpos} % Absolute positioning
\usepackage{upgreek} % Cool greek letters

\usepackage{fancyvrb}
\usepackage{fvextra}
\usepackage{chngcntr}

%%% Page geometry
% \setlength\parindent{0pt} % No indentation in paragraphs
\setlist{noitemsep} % No spacing between list items

\usepackage{float}
\usepackage{multirow}

\geometry{
    paper=a4paper,
    top=2.5cm,
    bottom=3cm,
    left=2.5cm,
    right=2.5cm,
    headheight=0.75cm,
    footskip=1.5cm,
    headsep=0.75cm,
}

%%% Numeration
\newcommand{\RNumb}[1]{\uppercase\expandafter{\romannumeral #1\relax}}
\newcommand{\thesec}{\arabic{section}}
\renewcommand\thesection{\arabic{section}}
\renewcommand\thesubsection{\thesection.\arabic{subsection}}
\renewcommand\thesubsubsection{\RNumb{\arabic{subsubsection}}}
\renewcommand{\sectionmark}[1]{\markright{\thesection\ #1}}
\renewcommand{\bf}{\textbf}
\renewcommand{\it}{\textit}
\def\hash{\texttt{\#}}
\def\cpp{\C\texttt{++}}

\counterwithin{figure}{section}
\counterwithin{table}{section}
\renewenvironment{titlepage}{\thispagestyle{empty}} % Include titlepage into page numeration

%%% Headers and footers
\usepackage{setspace}
\usepackage{fancyhdr}
\usepackage{lastpage}


%%% Graphics
% Custom colors
\definecolor{myblue}{RGB}{72, 184, 178}
\definecolor{myblue1}{RGB}{0, 109, 167}
\definecolor{commentgreen}{RGB}{2,112,10}
\definecolor{mauve}{rgb}{0.58,0,0.82}
\definecolor{amethyst}{RGB}{153, 102, 203}

\usepackage{pgfplots} % Plots
\usepackage{tikz}
\usetikzlibrary{3d,perspective,decorations.text}
\usetikzlibrary{animations}
\usetikzlibrary{positioning}
\usetikzlibrary{matrix}
\usepackage{tikz-cd}
\usetikzlibrary{cd}
\usetikzlibrary{karnaugh}
\pgfplotsset{width=6cm,compat=newest}
\usepackage{color}

\usepackage[framemethod=TikZ]{mdframed}
\newcommand{\definebox}[2]{\newcounter{#1}\newenvironment{#1}[1][]{\stepcounter{#1}\mdfsetup{frametitle={\tikz[baseline=(current bounding box.east),outer sep=0pt]\node[anchor=east,rectangle,fill=white]{\strut \MakeUppercase#1~\csname the#1\endcsname\ifstrempty{##1}{}{:~##1}};}}\mdfsetup{innertopmargin=1pt,linecolor=#2,linewidth=3pt,topline=true,frametitleaboveskip=\dimexpr-\ht\strutbox\relax,}\begin{mdframed}[]\relax}{\end{mdframed}}}
\definebox{definition}{black!90}
\definebox{theorem}{myblue1!90}
\definebox{demonstration}{amethyst!90}

\newcounter{Theorem}
\def\themytheorem{\thesection.\arabic{Theorem}}
\usepackage[most]{tcolorbox}
\tcbuselibrary{theorems}
\newtcbtheorem{Theorem}{Theorem}
{colframe=myblue!90,coltitle=black,colback=white,fonttitle=\bfseries}{Th}

%%% Code listings
\usepackage{matlab-prettifier}

\lstset{
    extendedchars=\true,
}
\lstdefinestyle{cpp} {
    language=C++,
    frame=tb,
    tabsize=4,
    showstringspaces=false,
    numbers=left,
    captionpos=b,
    columns=flexible,
    upquote=true,
    commentstyle=\color{commentgreen},
    keywordstyle=\color{blue},
    stringstyle=\color{commentgreen},
    basicstyle=\small\ttfamily,
    emph={int,char,double,float,unsigned,void,bool,size\_t},
    emphstyle={\color{blue}},
    escapechar=\&,
    classoffset=1,
    otherkeywords={>,<,.,;,-,!,=,~},
    morekeywords={>,<,.,;,-,!,=,~},
    keywordstyle=\color{black},
    classoffset=0,
}
\lstdefinestyle{py} {
    language=Python,
    frame=tb,
    tabsize=4,
    showstringspaces=false,
    numbers=left,
    captionpos=b,
    columns=flexible,
    upquote=true,
    commentstyle=\color{commentgreen},
    keywordstyle=\color{blue},
    stringstyle=\color{commentgreen},
    basicstyle=\small\ttfamily,
    emph={and,as,assert,break,class,continue,def,del,elif,else,except,False,finally,for,from,global,if,import,in,%
    is,lambda,None,nonlocal,not,or,pass,raise,return,True,try,while,with,yield},
    emphstyle={\color{blue}},
    classoffset=1,
    otherkeywords={>,<,.,;,-,!,=,~},
    morekeywords={>,<,.,;,-,!,=,~},
    keywordstyle=\color{black},
    classoffset=0,
}
\lstdefinestyle{def} {
    frame=tb,
    tabsize=4,
    showstringspaces=false,
    numbers=left,
    captionpos=b,
    columns=flexible,
    upquote=true,
    commentstyle=\color{black},
    keywordstyle=\color{black},
    stringstyle=\color{black},
    basicstyle=\small\ttfamily,
    emph={int,char,double,float,unsigned,void,bool,size\_t},
    emphstyle={\color{black}},
    escapechar=\&,
    classoffset=1,
    otherkeywords={>,<,.,;,-,!,=,~},
    morekeywords={>,<,.,;,-,!,=,~},
    keywordstyle=\color{black},
    classoffset=0,
}

%%% Other
\usepackage[normalem]{ulem} % }
\useunder{\uline}{\ul}{}    % } Underline text
\usepackage[colorlinks,urlcolor=blue,filecolor=blue,citecolor=blue,linkcolor = blue,unicode=true]{hyperref}
\usepackage{titlesec}
\titlelabel{\thetitle.\quad}
\usepackage{secdot}
\sectiondot{subsection}
\usepackage{kvmap} % Karnaugh-maps for logic functions
\usepackage{}
\newcommand{\projectname}[3]{
    \begin{center}
        \Large
        \textbf{#1}\\[10pt]
        \textbf{#2}\\[10pt]
        \normalsize
        #3
        \rule{\linewidth}{0.4pt}
    \end{center}
}

\newcommand{\hfconfiguration}[3]{
    \pagestyle{fancy}
    \fancyhead[LE,RO]{}
    \fancyhead[LO,RE]{#1}
    \renewcommand{\footrulewidth}{0.4pt}
    \fancyfoot[C]{\thepage/\pageref*{LastPage}}
    \fancyfoot[LO,RE]{#2}
    \fancyfoot[LE,RO]{#3}
}

\newcommand{\filename}[2]{
    \pagebreak
    \titleformat{\section}
    [display]
    {\bfseries\Large}
    {}
    {0ex}
    {
        \vspace{-4.5ex}
        % \rule{\textwidth}{1pt}
        #1 \centering
    % \vspace{1ex}
    }
    [
        \normalfont\large
        #2
        \rule{\textwidth}{0.4pt}
        \normalsize
    ]
}

\newcommand{\project}[6]{
    \projectname{#1}{#2}{#3}
    \pagestyle{empty}
    \tableofcontents
    \newpage
    \hfconfiguration{#4}{#5}{#6}
}
%%% Final touch
\usepackage{subfiles}


\begin{document}
    \begin{titlepage}
		\begin{center}
			\large Санкт-Петербургский политехнический университет Петра Великого\\
			\large Институт компьютерных наук и технологий \\
			\large Кафедра компьютерных систем и программных технологий\\[6cm]


		\huge КУРСОВАЯ РАБОТА\\[0.5cm]
			\large по дисциплине <<Вычислительная математика>>\\[0.1cm]
			\large\textbf{Исследование уравнения движения пузырьков}\\[5cm]
		\end{center}


		\begin{flushright}
			\begin{minipage}{0.25\textwidth}
				\begin{flushleft}

					\large\textbf{Работу выполнил:}\\
					\large Ильин В.П.\\
					\large {Группа:} 3530901/10005\\

					\large \textbf{Преподаватель:}\\
					\large Куляшова З.В.

				\end{flushleft}
			\end{minipage}
		\end{flushright}

		\vfill

		\begin{center}
			\large Санкт-Петербург\\
			\large \the\year
		\end{center}
	\end{titlepage}

	\vfill
	\newpage


    \tableofcontents

    \section{Цели работы}
    \begin{itemize}
        \item Закрепление знаний и получение практических навыков синтеза комбинационных схем в заданном
        элементном базисе.
        \item Получение навыков ввода проекта в графическом редакторе пакета QП, тестирования и отладки проекта и
        исследования на модели рисков сбоев.
        \item Получение навыков отладки цифровых устройств данного класса на физической модели: конфигурирование ПЛИС и
        экспериментальная проверка работы комбинационной схемы при использовании лабораторной платы miniDiLab.
    \end{itemize}

    \section{Задача}
	В соответствии с вариантом, составить таблицу истинности для логической функции 5 переменных. Минимизировать функцию при помощи карт
	Карно. Синтезировать комбинационную схему и исследовать ее при помощи САПР Quartus Prime. Реализовать полученную
	комбинационную схему на физической модели.
	
	Исходная функция задана таблично:
	\begin{table}[H]
		\centering
		\begin{tabular}{|c|c|}
		\hline Для значений $y = 1$ & Для значений $y = \textit{н}$ \\
		\hline $1, 6-10, 13, 15, 20-25, 28, 31$ & $3, 4, 12, 17, 26$ \\
		\hline
		\end{tabular}
	\end{table}
    \section{Ход работы}
    \subsection{Получение таблицы истинности}
    Для построения таблицы истинности необходимо перебрать все возможные
    наборы значений аргументов и указать данное значение $y_{\text{исх}}$ для каждого набора.
    Столбец $y_{\text{теор}}$ соответствует выбранным значениям для наборов <<н>>, столбец
    $y_{\text{эксп}}$ -- значениям, полученным в процессе тестирования комбинационной схемы,
    $y_{\text{физ}}$ -- значениям, полученным в процессе тестирования физической схемы.
    \begin{table}[H]
		\centering
		\begin{tabular}{|c|c|c|c|c|c|c|c|c|c|}
		\hline
		№ & $x_1$ & $x_2$ & $x_3$ & $x_4$ & $x_5$ & $y_{\text{исх}}$ & $y_{\text{теор}}$ & $y_{\text{эксп}}$ & $y_{\text{физ}}$ \\
		\hline 
		0 & 0 & 0 & 0 & 0 & 0 & 0 & 0 & 0 & 0 \\
		\hline 
		1 & 0 & 0 & 0 & 0 & 1 & 1 & 1 & 1 & 1 \\
		\hline 
		2 & 0 & 0 & 0 & 1 & 0 & 0 & 0 & 0 & 0 \\
		\hline 
		3 & 0 & 0 & 0 & 1 & 1 & н & 0 & 0 & 0 \\
		\hline 
		4 & 0 & 0 & 1 & 0 & 0 & н & 0 & 0 & 0 \\
		\hline 
		5 & 0 & 0 & 1 & 0 & 1 & 0 & 0 & 0 & 0 \\
		\hline 
		6 & 0 & 0 & 1 & 1 & 0 & 1 & 1 & 1 & 1 \\
		\hline 
		7 & 0 & 0 & 1 & 1 & 1 & 1 & 1 & 1 & 1 \\
		\hline 
		8 & 0 & 1 & 0 & 0 & 0 & 1 & 1 & 1 & 1 \\
		\hline 
		9 & 0 & 1 & 0 & 0 & 1 & 1 & 1 & 1 & 1 \\
		\hline 
		10 & 0 & 1 & 0 & 1 & 0 & 1 & 1 & 1 & 1 \\
		\hline 
		11 & 0 & 1 & 0 & 1 & 1 & 0 & 0 & 0 & 0 \\
		\hline 
		12 & 0 & 1 & 1 & 0 & 0 & н & 1 & 1 & 1 \\
		\hline 
		13 & 0 & 1 & 1 & 0 & 1 & 1 & 1 & 1 & 1 \\
		\hline 
		14 & 0 & 1 & 1 & 1 & 0 & 0 & 0 & 0 & 0 \\
		\hline 
		15 & 0 & 1 & 1 & 1 & 1 & 1 & 1 & 1 & 1 \\
		\hline 
		16 & 1 & 0 & 0 & 0 & 0 & 0 & 0 & 0 & 0 \\
		\hline 
		17 & 1 & 0 & 0 & 0 & 1 & н & 1 & 1 & 1 \\
		\hline 
		18 & 1 & 0 & 0 & 1 & 0 & 0 & 0 & 0 & 0 \\
		\hline 
		19 & 1 & 0 & 0 & 1 & 1 & 0 & 0 & 0 & 0 \\
		\hline 
		20 & 1 & 0 & 1 & 0 & 0 & 1 & 1 & 1 & 1 \\
		\hline 
		21 & 1 & 0 & 1 & 0 & 1 & 1 & 1 & 1 & 1 \\
		\hline 
		22 & 1 & 0 & 1 & 1 & 0 & 1 & 1 & 1 & 1 \\
		\hline 
		23 & 1 & 0 & 1 & 1 & 1 & 1 & 1 & 1 & 1 \\
		\hline 
		24 & 1 & 1 & 0 & 0 & 0 & 1 & 1 & 1 & 1 \\
		\hline 
		25 & 1 & 1 & 0 & 0 & 1 & 1 & 1 & 1 & 1 \\
		\hline 
		26 & 1 & 1 & 0 & 1 & 0 & н & 1 & 1 & 1 \\
		\hline 
		27 & 1 & 1 & 0 & 1 & 1 & 0 & 0 & 0 & 0 \\
		\hline 
		28 & 1 & 1 & 1 & 0 & 0 & 1 & 1 & 1 & 1 \\
		\hline 
		29 & 1 & 1 & 1 & 0 & 1 & 0 & 0 & 0 & 0 \\
		\hline 
		30 & 1 & 1 & 1 & 1 & 0 & 0 & 0 & 0 & 0 \\
		\hline 
		31 & 1 & 1 & 1 & 1 & 1 & 1 & 1 & 1 & 1 \\
		\hline
		\end{tabular}
		\label{tab:truth}
		\caption{Составленная таблица истинности}
    \end{table}
    
    \subsection{Минимизация функции}
    По полученной таблице истинности построим карту Карно и произведем склеивания.
    \begin{center}
    \begin{kvmap}
		\begin{kvmatrix}{x_3,x_2,x_1,x_5,x_4}
			0        & 0        & 1        & 1 & 1 & 1 & 1 & \text{н}\\
			0        & 0        & \text{н} & 1 & 0 & 0 & 1 & 1       \\
			\text{н} & 0        & 0        & 0 & 1 & 1 & 1 & 1       \\
			1        & \text{н} & 1        & 1 & 1 & 0 & 1 & 0
		\end{kvmatrix}
		\bundle{2}{0}{3}{1}
		\bundle{4}{0}{5}{0}
		\bundle{6}{0}{6}{3}
		\bundle{7}{1}{7}{2}
		\bundle[color=blue]{4}{2}{7}{2}
		\bundle{4}{2}{4}{3}
		\bundle{0}{3}{3}{3}
    \end{kvmap}
    \end{center}
	По результатам склеивания запишем ЛФ в минимальной ДНФ:
	\begin{equation*}
		y = x_2 \overline{x}_3 \overline{x}_5 + x_2 x_3 \overline{x}_4 \overline{x}_5 + x_1 \overline{x}_2 x_3 + \overline{x}_1 \overline{x}_2 + x_3 x_4 + x_3 x_4 x_5 + \overline{x}_1 x_2 x_3 x_5 + \overline{x}_3 \overline{x}_4 x_5
	\end{equation*}
	
	Получив логическую функцию в минимальной форме, синтезируем ее схему в базисе И-ИЛИ-НЕ, используя
	графический редактор Quartus Prime, а также выполним анализ и синтез и назначим входные и выходные
	сигналы на выводы микросхемы.
	\begin{figure}[H]
		\centering
		\includegraphics[width=\linewidth]{subfiles/images/scheme}
		\caption{Полученная схема}
		\label{fig:scheme}
	\end{figure}
	
	\begin{figure}[H]
		\centering
		\includegraphics[width=\linewidth]{subfiles/images/pins}
		\caption{Назначенные сигналы}
		\label{fig:pins}
	\end{figure}
	
	\subsection{Анализ синтеза комбинационной схемы}
	Для начала воспользуемся RTL Viewer для рассмотрения преобразования
	логики в процессе синтеза. На рисунке видно, что существенных изменений
	не произошло, за исключением того, что теперь инверсия переменных
	реализуется на входах логических элементов.
	\begin{figure}[H]
		\centering
		\includegraphics[width=0.5\linewidth]{subfiles/images/rtl}
		\caption{Логическое представление устройства в RTL Viewer}
		\label{fig:rtl}
	\end{figure}
	На основе полученного RTL описания в процессе синтеза в элементном
	базисе выбранной для проекта ПЛИС синтезируется новая схема (Netlist)
	с той же функциональностью, что и исходная схема. Синтезированное
	представление схемы в базисе целевой ПЛИС доступно в Technology Map Viewer.
	\begin{figure}[H]
		\centering
		\includegraphics[width=\linewidth]{subfiles/images/tmv}
		\caption{Схема устройства в Technology Map Viewer}
		\label{fig:rmv}
	\end{figure}
	
	В редакторе временных диаграмм САПР QP создадим тест, в котором работа
	КС проверяется на всех наборах входных сигналов.
	\begin{figure}[H]
		\centering
		\includegraphics[width=\linewidth]{subfiles/images/wave}
		\caption{Временная диаграмма}
		\label{fig:wave}
	\end{figure}
	На диаграмме видно, что полученные выходные сигналы совпадают с ожидаемыми,
	следовательно, схема реализована верно. 
	
	Последним шагом в работе была реализация КС на физической модели.
	При помощи Quartus Prime, была запрограммирована микросхема
	miniDiLaB-CIV c ПЛИС Cyclone IV EP4CE6E22C8N, на которой при помощи переключателей и светодиодов 
	еще раз была проверена правильность работы логической функции.
	\begin{figure}[H]
		\centering
		\includegraphics[width=0.5\linewidth]{subfiles/images/chip}
		\caption{Пример работы физической модели}
		\label{fig:chip}
	\end{figure}
	
	\section{Вывод}
	В результате работы заданная таблично логическая функция была минимизирована при помощи карт
	Карно. Была синтезирована и исследована комбинационная схема. В заключение, она была 
	реализована на физической модели.
\end{document}
