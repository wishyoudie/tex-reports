\section{Устойчивость решений дифференциальных и разностных уравнений.}\label{sec:ch19}
Обратимся к системе нелинейных дифференциальных уравнений
\begin{equation}
    \frac{d\bf{x}(t)}{dt} = \bf{f}(t, \bf{x}), \quad \bf{x}(t_0) = \bf{x}_0, \quad t \in [a, b] \label{eq:nonlinear_differential_equation}
\end{equation}
где $t$ -- независимая переменная, \bf{x} -- вектор решения; $\bf{f}(t, \bf{x})$ -- вектор-функция,
непрерывная по $t$ и имеющая непрерывные частные производные первого порядка по компонентам вектора \bf{x}.

Большой интерес представляет исследование зависимости решения задачи Коши от начальных условий. Если
незначительные изменения в $\bf{x}_0$ могут существенно изменить решение, то в прикладном отношении
такое решение часто неприемлемо. На конечном промежутке $[a, b]$ для систем~\eqref{eq:nonlinear_differential_equation}
с непрерывной функцией $\bf{f}(t, \bf{x})$ и свойством единственности решения имеет место
\emph{интегральная непрерывность решений}. Иными словами,
\begin{equation*}
    \forall \varepsilon > 0 \quad \exists \, \delta > 0: \quad \left\| \bf{x}(t_0) - \bf{z}(t_0) \right\| < \delta \Rightarrow \left\| \bf{x}(t) - \bf{z}(t) \right\| < \varepsilon
\end{equation*}
Иначе обстоит дело при $t \rightarrow \infty$. Изучением этих вопросов занимается теория устойчивости.
\begin{definition}
    Решение $\bf{x}(t)$ называется \emph{устойчивым по Ляпунову}, если
    \begin{equation*}
        \forall \varepsilon > 0 \quad \exists \, \delta > 0: \quad \forall \bf{z}(t) \quad \forall t \in [t_0; \infty) \quad \left\| \bf{x}(t_0) - \bf{z}(t_0) \right\| < \delta \Rightarrow \left\| \bf{x}(t) - \bf{z}(t) \right\| < \varepsilon
    \end{equation*}
    Иными словами, решение $\bf{x}(t)$ называется устойчивым, если другие достаточно близкие к нему
    в момент времени $t_0$ решения $\bf{z}(t)$ целиком находятся в узкой $\varepsilon$-трубке,
    построенной вокруг $\bf{x}(t)$.
\end{definition}

\begin{definition}
    Решение $\bf{x}(t)$ системы~\eqref{eq:nonlinear_differential_equation} называется \emph{асимптотически устойчивым}
    по Ляпунову, если
    \begin{enumerate}
        \item Оно устойчиво
        \item Выполняется условие
        \begin{equation*}
            \exists \Delta > 0: \quad \left\| \bf{x}(t_0) - \bf{z}(t_0) \right\| < \Delta \Rightarrow \lim_{t \rightarrow \infty} \left\| \bf{x}(t) - \bf{z}(t) \right\| = 0
        \end{equation*}
    \end{enumerate}
    В случае асимптотической устойчивости близкие решения не только остаются близкими друг к другу, но и
    неограниченно сближаются при возрастании $t$.
\end{definition}
\vspace{10pt}

Для систем разностных уравнений
\begin{equation}
    \bf{y}(n+1) = \bf{g}\left( n, \bf{y}(n) \right), \quad \bf{y}(n_0) = \bf{y}_0, \quad n \in [n_0; \infty) \label{eq:sub_equation}
\end{equation}
понятие устойчивости вводится аналогичным образом.
\begin{definition}
    Решение $\bf{y}(n)$ называется \emph{устойчивым}, если
    \begin{equation*}
        \forall \varepsilon > 0 \quad \exists \, \delta > 0: \, \forall \bf{w}(n) \, \forall n \in [n_0; \infty) \quad \left\| \bf{y}(n_0) - \bf{w}(n_0) \right\| < \delta \Rightarrow \left\| \bf{y}(n) - \bf{w}(n) \right\| < \varepsilon
    \end{equation*}
\end{definition}
\vspace{10pt}

Сформулированные определения позволяют сделать суждение об устойчивости после анализа уже полученных решений.
С практической точки зрения важно судить об устойчивости, не решая систему. Это возможно, в частности, для
линейных систем с постоянной матрицей
\begin{equation*}
    \frac{d\bf{x}(t)}{dt} = \bf{A} \bf{x}(t) + \bf{g}(t), \qquad \bf{x}(t_0) = \bf{x}_0
\end{equation*}
Будем называть их устойчивыми (асимптотически устойчивыми, неустойчивыми), если все их решения устойчивы
(асимптотически устойчивы, неустойчивы).

Пусть $\bf{x}(t)$ и $\bf{z}(t)$ -- два различных решения~\eqref{eq:diffeqsol}, отличающиеся начальными
условиями. В соответствии с~\eqref{eq:diffeqsol} они имеют вид
\begin{gather*}
    \bf{x}(t) = e^{\bf{A}t}\bf{x}_0 + \int_{0}^{t} e^{\bf{A}(t - \tau)}\bf{g}(\tau)d\tau = e^{\bf{A}t}\bf{x}_0 + \int_{0}^{t} e^{\bf{A}\tau}\bf{g}(t - \tau)d\tau\\
    \bf{z}(t) = e^{\bf{A}t}\bf{z}_0 + \int_{0}^{t} e^{\bf{A}(t - \tau)}\bf{g}(\tau)d\tau = e^{\bf{A}t}\bf{z}_0 + \int_{0}^{t} e^{\bf{A}\tau}\bf{g}(t - \tau)d\tau
\end{gather*}
Вычтем из первой формулы вторую. После сокращения интегралов получаем
\begin{equation*}
    \bf{x}(t) - \bf{z}(t) = e^{\bf{A}t}\left( \bf{x}_0 - \bf{z}_0 \right)
\end{equation*}
Пусть первоначально собственные значения матрицы \bf{A} различны. Тогда, используя для матричной
экспоненты формулу Лагранжа-Сильвестра, имеем
\begin{equation*}
    \bf{x}(t) - \bf{z}(t) = e^{\bf{A}t}\left( \bf{x}_0 - \bf{z}_0 \right) = \sum_{k=1}^{m} e^{\uplambda_k t}\bf{T}_k \left( \bf{x}_0 - \bf{z}_0 \right)
\end{equation*}

Обращаясь к определениям устойчивости, приходим к выводу о том, что для обеспечения неравенства
$\displaystyle \left( \left\| \bf{x}(t) - \bf{z}(t) \right\| < \varepsilon \right)$ элементы матричной
экспоненты при $t \rightarrow \infty$ должны быть ограничены. А это, в свою очередь, требует, чтобы
вещественные части $\displaystyle \Re(\uplambda_k)$ собственных значений были бы неположительные.
Для асимптотической устойчивости условие $\displaystyle \lim_{t \rightarrow \infty} \left\| \bf{x}(t) - \bf{z}(t) \right\| = 0$
выполняется, когда элементы матричной экспоненты при $t \rightarrow \infty$ стремятся к нулю, а
вещественные части собственных значений соответственно отрицательные.

Если среди собственных значений есть кратные, условия несколько корректируются. Пусть, например,
собственное значение $\uplambda_k$ имеет кратность $s$. Тогда в решении этой группе собственных
значений отвечает слагаемое $\displaystyle P_{s-1}(t)e^{\uplambda_k t}$. Если для $\Re(\uplambda_k) < 0$
асимптотическая устойчивость обеспечивается независимо от кратности корня
\begin{equation*}
    P_{s-1}(t) e^{\uplambda_k t} \rightarrow 0 \text{ при } t \rightarrow \infty
\end{equation*}
то при нулевой вещественной части $P_{s-1}(t) \rightarrow \pm \infty$ при $t \rightarrow \infty$ и не выполняется
условие $\displaystyle \left( \left\| \bf{x}(t) - \bf{z}(t) \right\| < \varepsilon \right)$.

Подведем итоги.
\begin{enumerate}
    \item Для асимптотической устойчивости необходимо и достаточно, чтобы для всех собственных значений
    выполнялись условия $\Re(\uplambda_k) < 0$.
    \item Для устойчивости необходимо, чтобы $\Re(\uplambda_k) \leq 0$. При этом достаточно, чтобы
    среди собственных значений с нулевой вещественной частью не было бы кратных.
    \item Для неустойчивости необходимо наличие хотя бы одного собственного значения с $\Re(\uplambda_k) > 0$
    или кратных собственных значений с $\Re(\uplambda_k) = 0$.
\end{enumerate}
\vspace{10pt}

Теперь обратимся к системе разностных уравнений с постоянной матрицей
\begin{equation*}
    \bf{y}(n+1) = \bf{B}\bf{y}(n) + \bf{g}(n)
\end{equation*}
Пусть \bf{y}(n) и \bf{w}(n) -- ее два различных решения, отличающиеся начальными условиями. Они имеют вид
\begin{gather*}
    \bf{y}(n) = \bf{B}^n\bf{y}(0) + \sum_{k=0}^{n-1} \bf{B}^{n-k-1}\bf{g}(k) = \bf{B}^n \bf{y}(0) + \sum_{k=0}^{n-1} \bf{B}^k \bf{g}(n - k - 1)\\
    \bf{w}(n) = \bf{B}^n\bf{w}(0) + \sum_{k=0}^{n-1} \bf{B}^{n-k-1}\bf{g}(k) = \bf{B}^n \bf{w}(0) + \sum_{k=0}^{n-1} \bf{B}^k \bf{g}(n - k - 1)
\end{gather*}
Вычитая из первой формулы вторую, после сокращения сумм получаем
\begin{equation*}
    \bf{y}(n) - \bf{w}(n) = \bf{B}^n \left( \bf{y}_0 - \bf{w}_0 \right)
\end{equation*}
Если все собственные значения $\mu_k$ матрицы \bf{B} различны, то, воспользовавшись формулой
Лагранжа-Сильвестра для $\displaystyle \bf{B}^n$, имеем
\begin{equation*}
    \bf{y}(n) - \bf{w}(n) = \sum_{k=1}^{m} \mu_k^n \bf{T}_k \left( \bf{y}_0 - \bf{w}_0 \right)
\end{equation*}
Аналогично предыдущему для обеспечения неравенства $\displaystyle \left( \left\| \bf{y}(n) - \bf{w}(n) \right\| < \varepsilon \right)$
элементы матрицы $\displaystyle \bf{B}^n \text{ при } n \rightarrow \infty$ должны быть ограничены.
А это, в свою очередь, требует выполнения условий $|\mu_k| \leq 1$ для всех собственных значений.
Для асимптотической устойчивости неравенства должны быть строгими: $|\mu_k| < 1$. Если собственное
значение $\mu_k$ имеет кратность $s$, то, как и для дифференциальных уравнений, в решении появляется
слагаемое $\displaystyle P_{s-1}(n)\cdot \mu_k^n$. Для $|\mu_k| < 1$ этот факт не оказывает влияния
на условие устойчивости, но для $|\mu_k| = 1$ условие устойчивости нарушается, если $\displaystyle P_{s-1}(n) \rightarrow \pm \infty \text{ при } n \rightarrow \infty$.
Как результат, сформулируем условия устойчивости.
\begin{enumerate}
    \item Для асимптотической устойчивости необходимо и достаточно, чтобы для всех собственных
    значений выполнялись условия $\displaystyle |\mu_k| < 1$.
    \item Для устойчивости необходимо, чтобы $\displaystyle |\mu_k| \leq 1$. При этом достаточно, чтобы
    среди собственных значений с единичными модулями не было бы кратных.
    \item Для неустойчивости необходимо наличие хотя бы одного собственного значения с $\displaystyle |\mu_k| > 1$
    или кратных собственных значений с $\displaystyle |\mu_k| = 1$.
\end{enumerate}
